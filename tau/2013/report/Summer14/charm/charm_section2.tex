\subsection{Excited \emph{$D_{(s)}$} mesons}
Excited charm meson states have received increased attention since the first observation of states that could not be accommodated by QCD predictions~\cite{Aubert:2003fg,
Besson:2003cp,Abe:2003jk,Aubert:2003pe}. Tables \ref{table:charm:spect:1} and \ref{tabel:charm:spect:2} summarize recent measurements of the masses and widths of excited $D$ and $D_{s}$ mesons, respectively. If a preferred assignment of spin and parity was measured it is listed in the column $J^{P}$, where the label natural denotes $J^{P}=0^{-},1^{+},2^{-}\ldots$ and unnatural $J^{P}=0^{+},1^{-},2^{+}\ldots$. If possible an average mass and width was calculated, which is listed in the gray shaded row. The calculation of the averages assumes no correlation between individual measurements. A summary of the averaged masses and widths is shown in Figure~\ref{fig:charm:spect:1}. 
\begin{figure}[htb!]
\begin{centering}
\includegraphics[width=0.45\textwidth]{./figures/charm/Dsmasses}
\quad
\includegraphics[width=0.45\textwidth]{./figures/charm/Dmasses}\\
\includegraphics[width=0.45\textwidth]{./figures/charm/Dswidths}
\quad
\includegraphics[width=0.45\textwidth]{./figures/charm/Dwidths}

\caption{\label{fig:charm:spect:1}  Summary of averaged masses for $D_{s}$ mesons are shown in subfigure (a) and for $D$ mesons in subfigure (b). The average widths for $D_{s}$ mesons are shown in subfigure (c) and for $D$ mesons in subfigure (d). The vertical shaded regions distinguish between different spin parity states.}
\end{centering}
\end{figure}

In the study of $B_{s}^{0}\to \overline{D}{}^{0}K^{-}\pi^{+}$ decays the LHCb collaboration searched for excited excited  $D_{s}$ mesons~\cite{Aaij:2014xza}.  Previous measurements by \babar{}~\cite{Aubert:2009ah} and LHCb~\cite{Aaij:2012pc} indicated the existence of a strange-charm $D^{*}_{sJ}(2860)^{-}$ meson. The new measurement of LHCb showed with $10\sigma$ significance that this state is comprised of two different particles,  one of spin 1 and and one of spin 3. This represents the first measurement of a heavy flavored spin-3 particle, and the first observation $B$ mesons decays to spin 3 particles.

The masses and widths of narrow ($\Gamma<50$~MeV) orbitally excited $D$ mesons (denoted $D^{\ast\ast}$), both neutral and charged, are well established. Measurements of broad states ($\Gamma\sim$ 200--400~MeV) are less abundant, as identifying the signal is more challenging. There is a slight discrepancy between the 
$D_0^\ast(2400)^0$ masses measured by the Belle~\cite{Abe:2003zm} and  FOCUS~\cite{Link:2003bd} experiments. No data exist yet for the $D_1(2430)^{\pm}$ state. Dalitz plot analyses of $B\to D^{(\ast)}\pi\pi$ decays strongly favor the assignments $0^+$ and $1^+$ for the spin-parity quantum numbers of the $D_0^\ast(2400)^0/D_0^\ast(2400)^\pm$ and $D_1(2430)^{0}$ states, respectively. The measured masses and widths, as well as the $J^P$ values, are in agreement with theoretical predictions based on potential models~\cite{Godfrey:1985xj, Godfrey:1986wj, Isgur:1991wq, Schweitzer:2002nm}. 


Tables~\ref{table:charm:spect:3} and \ref{table:charm:spect:4} summarizes the branching fractions of $B$ mesons decays to excited $D$ and $D_{s}$ states, respectively. It can be noted that the branching fractions for $B$ mesons decaying to a narrow $D^{\ast\ast}$ state and a pion are similar for charged and neutral $B$ initial states, the branching fractions to a broad $D^{\ast\ast}$ state and $\pi^+$ are much larger for $B^+$ than for $B^0$. This may be due to the fact that color-suppressed amplitudes contribute only to the $B^+$ decay and not to the $B^0$ decay (for a theoretical discussion, see Ref.~\citep{Jugeau:2005yr,Colangelo:2004vu}). Measurements of individual branching fractions of $D$ mesons are difficult due to the unknown fragmentation of $c\bar c \to D^{\ast\ast}$ or due to the unknown $B \to D^{\ast\ast} X$ branching fractions.

The discoveries of the $D_{s0}^\ast(2317)^{\pm}$ and $D_{s1}(2460)^{\pm}$ have triggered increased interest in properties of, and searches for, excited $D_s$ mesons (here generically denoted $D_s^{\ast\ast}$). While the masses and widths of $D_{s1}(2536)^{\pm}$ and $D_{s2}(2573)^{\pm}$ states are in relatively good agreement with potential model predictions, the masses of $D_{s0}^\ast(2317)^{\pm}$ and $D_{s1}(2460)^{\pm}$ states are significantly lower than expected (see Ref.~\cite{Cahn:2003cw} for a discussion of $c\bar{s}$ models). Moreover, the mass splitting between these two states greatly exceeds that between the $D_{s1}(2536)^{\pm}$ and $D_{s2}(2573)^{\pm}$. These unexpected properties have led to interpretations of the $D_{s0}^\ast(2317)^{\pm}$ and $D_{s1}(2460)^{\pm}$ as exotic four-quark states~\cite{Barnes:2003dj,Lipkin:2003zk}.

While there are few measurements of the $J^P$ values of $D_{s0}^\ast(2317)^{\pm}$ and $D_{s1}(2460)^{\pm}$, the available data favor $0^+$ and $1^+$, respectively. A molecule-like ($DK$) interpretation of the $D_{s0}^\ast(2317)^{\pm}$ and $D_{s1}(2460)^{\pm}$~\cite{Barnes:2003dj,Lipkin:2003zk} that can account for their low masses and isospin-breaking decay modes is tested by searching for charged and neutral isospin partners of these states; thus far such searches have yielded negative results. Therefore the subset of models that predict equal production rates for different charged states is excluded. The molecular picture can also be tested by measuring the rates for the radiative processes $D_{s0}^\ast(2317)^{\pm}/D_{s1}(2460)^{\pm}\to D_s^{(\ast)}\gamma$ and comparing to theoretical predictions. The predicted rates, however, are below the sensitivity of current experiments. 

Another model successful in explaining the total widths and the $D_{s0}^\ast(2317)^{\pm}$ -- $D_{s1}(2460)^{\pm}$ mass splitting is based on the assumption that these states are chiral partners of the ground states $D_{s}^{+}$ and~$D_{s}^{*}$~\cite{Bardeen:2003kt}. While some measured branching fraction ratios agree with predicted
values, further experimental tests with better sensitivity are needed to confirm or refute this scenario. A summary of the mass difference measurements is given in Table~\ref{table:charm:spect:5}.

In addition to the $D_{s0}^\ast(2317)^{\pm}$ and $D_{s1}(2460)^{\pm}$ states, other excited $D_s$ states may have been observed. SELEX has reported a $D_{sJ}(2632)^{\pm}$ candidate~\cite{Evdokimov:2004iy}, but this has not been confirmed by other experiments. Recently, Belle, \babar{} and LHCb have observed $D_{s1}(2700)^{\pm}$ which may be radial excitations of the $D_s^{\ast\pm}$. Equally the $D_{s1}(2860)^{\pm}$ measured by LHCb and $D_{sJ}(3040)^{\pm}$ measured by \babar{} could be excitations of $D_{s0}^\ast(2317)^{\pm}$ and $D_{s1}(2460)^{\pm}$ or $D_{s1}(2536)^{\pm}$, respectively (for a theoretical discussion, see Ref~\cite{Matsuki:2006rz}).

Table~\ref{table:charm:spect:6} summarizes measurements of the polarization amplitude $A_{D}$ (sometimes also referred as helicity parameter), which describes the initial polarization of the $D$ meson. In $D^{\ast\ast}$ meson decay the helicity distribution varies like $1 + A_{D}\cos^{2}\theta_{H}$, where $\theta_{H}$ is the angle in the $D^{\ast}$ rest frame between the two pions emitted by decay $D^{\ast\ast} \to D^{\ast}\pi$ and the $D^{\ast} \to D \pi$. The parameter is sensitive to possible S-wave contributions in the decay. In the case of an unpolarized $D$ meson decay decaying purely via D-wave the polarization amplitude is predicted to give $A_{D}=3$. 
Studies of the $D_{1}(2420)^{0}$ meson by the ZUES and \babar{} collaborations suggest that there is an S-wave admixture in the decay, which is  contrary to Heavy Quark Effective Theory calculations~\cite{Isgur:1989vq,Neubert:1993mb}.

\begin{table}[htb!]
\begin{adjustbox}{width=\textwidth,center}
{\setlength\tabcolsep{0pt}
	\begin{tabular}{cp{5pt}cp{5pt}cp{5pt}r@{}lp{5pt}r@{}lp{5pt}cp{5pt}c}
	\toprule
	\rowcolor{Gray} Resonance                 && $J^{P}$                   && Decay mode                                                         && \multicolumn{2}{c}{Mass [MeV$/c^{2}$]} && \multicolumn{2}{c}{Width [MeV]}      && \multicolumn{1}{c}{Measured by} && \multicolumn{1}{c}{Reference} 
	\\ \midrule
	\multirow{3}{*}{$D_{s0}^{*}(2317)^{\pm}$}	&&	\multirow{3}{*}{$0^{+}$}	&&	$D_{s}^{+}\pi^{0}$	&&$	2319$&$.6\pm0.2\pm1.4	$&&$	$&$	$&&	\babar{}	&&	\cite{Aubert:2006bk}          \\
		&&		&&	$D_{s}^{+}\pi^{0}$	&&$	2317$&$.3\pm0.4\pm0.8	$&&$	$&$	$&&	\babar{}	&&	\cite{Aubert:2003pe}          \\  \cmidrule{6-14}
		&&		&&		&\cellcolor{Gray}&$	\cellcolor{Gray}  2318$&\cellcolor{Gray}$.0 \pm 0.8	$&\cellcolor{Gray}&	\cellcolor{Gray}&\cellcolor{Gray}	&\cellcolor{Gray}&	\cellcolor{Gray} Our average	&\cellcolor{Gray}&	\\ \midrule
	%												
													
	\multirow{3}{*}{$D_{s1}(2460)^{\pm}$}	&&	\multirow{3}{*}{$1^{+}$}	&&	$D_{s}^{+}\pi^{0}\gamma, D_{s}^{+}\gamma, D_{s}^{+}\pi^{+}\pi^{-}$	&&$	2460$&$.1\pm0.2\pm0.8	$&&$	$&$	$&&	\babar{}	&&	\cite{Aubert:2006bk}          \\
		&&		&&	$D_{s}^{+}\pi^{0}\gamma$	&&$	2458$&${}\pm1.0\pm1.0	$&&$	$&$	$&&	\babar{}	&&	\cite{Aubert:2003pe}          \\ \cmidrule{6-14}
		&&		&&		&\cellcolor{Gray}&$	\cellcolor{Gray}  2459$&\cellcolor{Gray}$.6 \pm 0.7	$&\cellcolor{Gray}&	\cellcolor{Gray}&\cellcolor{Gray}	&\cellcolor{Gray}&	\cellcolor{Gray} Our average	&\cellcolor{Gray}&	\\ \midrule
	%												
	\multirow{11}{*}{$D_{s1}(2536)^{\pm}$}	&&	\multirow{11}{*}{$1^{+}$}	&&	$D^{*}{}^{+}K_{S}^{0}$	&&$	2535$&$.7\pm0.6\pm0.5	$&&$	$&$	$&&	D\O\ &&\cite{Abazov:2007wg}		\\
		&&		&&	$D^{*}{}^{+}K_{S}^{0}, D^{*}{}^{0}K^{+}$	&&$	2534$&$.78\pm0.31\pm0.40	$&&$	$&$	$&&	\babar{}	&&	\cite{Aubert:2007rva}         \\
		&&		&&	$D_{s}^{+}\pi^{+}\pi^{-}$	&&$	2534$&$.6\pm0.3\pm0.7	$&&$	$&$	$&&	\babar{}	&&	\cite{Aubert:2006bk}          \\
		&&		&&	$D^{*}{}^{+}K_{S}^{0}, D^{*}{}^{0}K^{+}$	&&$	2535$&$.0\pm0.6\pm1.0	$&&$	$&$	$&&	E687	&&	\cite{Frabetti:1993vv}        \\
		&&		&&	$D^{*}{}^{0}K^{+}$	&&$	2535$&$.3\pm0.2\pm0.5	$&&$	$&$	$&&	CLEO	&&	\cite{Alexander:1993nq}       \\
		&&		&&	$D^{*}{}^{+}K_{S}^{0}$	&&$	2534$&$.8\pm0.6\pm0.6	$&&$	$&$	$&&	CLEO	&&	\cite{Alexander:1993nq}       \\
		&&		&&	$D^{*}{}^{0}K^{+}$	&&$	2535$&$.2\pm0.5\pm1.5	$&&$	$&$	$&&	ARGUS	&&	\cite{Albrecht:1992zh}        \\
		&&		&&	$D^{*}{}^{+}K_{S}^{0}$	&&$	2535$&$.6\pm0.7\pm0.4	$&&$	$&$	$&&	CLEO	&&	\cite{Avery:1989ui}           \\
		&&		&&	$D^{*}{}^{+}K_{S}^{0}$	&&$	2535$&$.9\pm0.6\pm2.0	$&&$	$&$	$&&	ARGUS	&&	\cite{Albrecht:1989yi}        \\
		&&		&&	$D^{*}{}^{+}K_{S}^{0}$	&&$	$&$	$&&$	0$&$.92\pm0.03\pm0.04	$&&	\babar{}	&&	\cite{Lees:2011um}            \\ \cmidrule{6-14}
		&&		&&		&\cellcolor{Gray}&$	\cellcolor{Gray}2535$&\cellcolor{Gray}$.10 \pm 0.26	$&\cellcolor{Gray}&$	\cellcolor{Gray}0$&\cellcolor{Gray}$.92\pm0.05	$&\cellcolor{Gray}&	\cellcolor{Gray} Our average	&\cellcolor{Gray}&	\\ \midrule
													
	\multirow{5}{*}{$D_{s2}^{*}(2573)^{\pm}$}	&&	\multirow{5}{*}{$2^{+}$}	&&	$D^{0}K^{+}$	&&$	2568$&$.39\pm0.29\pm0.26	$&&$	16$&$.9\pm0.5\pm0.6	$&&	LHCb	&&	\cite{Aaij:2014baa}           \\
		&&		&&	$D^{+}K_{S}^{0}, D^{0}K^{+}$	&&$	2569$&$.4\pm1.6\pm0.5	$&&$	12$&$.1\pm4.5\pm1.6	$&&	LHCb	&&	\cite{Aaij:2011ju}            \\
		&&		&&	$D^{+}K_{S}^{0}, D^{0}K^{+}$	&&$	2572$&$.2\pm0.3\pm1.0	$&&$	27$&$.1\pm0.6\pm5.6	$&&	\babar{}	&&	\cite{Aubert:2006mh}          \\
		&&		&&	$D^{0}K^{+}$	&&$	2574$&$.25\pm3.3\pm1.6	$&&$	10$&$.4\pm8.3\pm3.0	$&&	ARGUS	&&	\cite{Albrecht:1995qx}        \\
		&&		&&	$D^{0}K^{+}$	&&$	2573$&$.2 _{-1.6}^{+1.7}\pm0.9	$&&$	16$&${}_{-4}^{+5}\pm3	$&&	CLEO	&&	\cite{Kubota:1994gn}          \\ \cmidrule{6-14}
		&&		&&		&\cellcolor{Gray}&$	\cellcolor{Gray}2569$&\cellcolor{Gray}$.08 \pm 0.35	$&\cellcolor{Gray}&$	\cellcolor{Gray} 16$&\cellcolor{Gray}$.9 \pm 0.8	$&\cellcolor{Gray}&	\cellcolor{Gray}Our average	&\cellcolor{Gray}&	\\ \midrule
													
	\multirow{4}{*}{$D_{s1}^{*}(2700)^{\pm}$}	&&	\multirow{4}{*}{$1^{-}$}	&&	$D^{*}{}^{+}K_{S}^{0}, D^{*}{}^{0}K^{+}$	&&$	2709$&$.2\pm1.9\pm4.5	$&&$	115$&$.8\pm7.3\pm12.1	$&&	LHCb	&&	\cite{Aaij:2012pc}            \\
		&&		&&	$DK, D^{*}K$	&&$	2710$&${}\pm 2 _{-7}^{+12}	$&&$	149$&${}\pm7 _{-52}^{+39}	$&&	\babar{}	&&	\cite{Aubert:2009ah}          \\
		&&		&&	$D^{0}K^{+}$	&&$	2708$&${}\pm9 _{-10}^{+11}	$&&$	108$&${}\pm2. _{-31}^{+36}	$&&	Belle	&&	\cite{Brodzicka:2007aa}       \\ \cmidrule{6-14}
		&&		&&		&\cellcolor{Gray}&$	\cellcolor{Gray} 2709$&\cellcolor{Gray}$.2 \pm 4.2	$&\cellcolor{Gray}&$	\cellcolor{Gray} 117$&\cellcolor{Gray}$.2 \pm 12.5	$&\cellcolor{Gray}&	\cellcolor{Gray} Our average	&\cellcolor{Gray}&	\\ \midrule
	%												
	\multirow{1}{*}{$D_{s1}^{*}(2860)^{\pm}$}	&&	\multirow{1}{*}{1}	&&	$ D^{0}K^{+}$	&\cellcolor{LightGray} &$	\cellcolor{LightGray} 2859$&\cellcolor{LightGray}${}\pm12\pm24	$&\cellcolor{LightGray}&$	\cellcolor{LightGray} 159$&\cellcolor{LightGray}${}\pm23\pm77	$&\cellcolor{LightGray}&	\cellcolor{LightGray} LHCb	&\cellcolor{LightGray}&	\cite{Aaij:2014xza}           \\ \midrule
	%												
	\multirow{1}{*}{$D_{s3}^{*}(2860)^{\pm}$}	&&	\multirow{1}{*}{3}	&&	$ D^{0}K^{+}$	&\cellcolor{LightGray}&$	\cellcolor{LightGray} 2860$&\cellcolor{LightGray}$.5\pm2.6\pm6.5	$&\cellcolor{LightGray}&$	\cellcolor{LightGray} 53$&\cellcolor{LightGray}${}\pm7\pm7	$&\cellcolor{LightGray}&	\cellcolor{LightGray} LHCb	&\cellcolor{LightGray}&	\cite{Aaij:2014xza}           \\ \midrule
	%												
	\multirow{1}{*}{$D_{sJ}(3040)^{\pm}$}	&&	\multirow{1}{*}{}	&&	$D^{*}K$	&\cellcolor{LightGray}&$	\cellcolor{LightGray} 3044$&\cellcolor{LightGray}${}\pm8 _{-5}^{+30}	$&\cellcolor{LightGray}&$	\cellcolor{LightGray}239$&\cellcolor{LightGray}${}\pm35 _{-42}^{+46}	$&\cellcolor{LightGray}&	\cellcolor{LightGray} \babar{}	&\cellcolor{LightGray}&	\cite{Aubert:2009ah}          \\ \bottomrule
\end{tabular}
}
\end{adjustbox}
\caption{\label{table:charm:spect:1} Summary of recent measurements of mass and width for different excited $D_{s}$ mesons. The column $J^{P}$ list the most significant assignment of spin and parity. If possible an average mass or width is calculated.}
\end{table} 

\begin{table}[htb!]
\begin{adjustbox}{width=0.76\textwidth,center}
{\setlength\tabcolsep{0pt}
	\begin{tabular}{cp{5pt}cp{5pt}cp{5pt}r@{}lp{5pt}r@{}lp{5pt}cp{5pt}c}
		\toprule
		\rowcolor{Gray} Resonance	&&	$J^{P}$	&&	Decay mode	&&	\multicolumn{2}{c}{Mass [MeV$/c^{2}$]}	& &	\multicolumn{2}{c}{Width [MeV]}	& &	\multicolumn{1}{c}{Measured by}	&&	\multicolumn{1}{c}{Reference}
		\\ \midrule												
		%D*0(2400)0												
		\multirow{4}{*}{$D_{0}^{*}(2400)^{0}$}   &                  & \multirow{4}{*}{$0^{+}$}     &                  & $D^{+}\pi^{-}$                  &                       & $	2297$                       & ${}\pm8\pm20	$                                &                       & $	273$                                           & ${}\pm12\pm48	$                              &                       & \babar{}                                                      &                       & \cite{Aubert:2009wg}                      \\
		                                         &                  &                              &                  & $D^{+}\pi^{-}$                  &                       & $	2308$                       & ${}\pm17\pm32	$                               &                       & $	276$                                           & ${}\pm21\pm63	$                              &                       & Belle                                                         &                       & \cite{Abe:2003zm}                         \\
		                                         &                  &                              &                  & $D^{+}\pi^{-}$                  &                       & $	2407$                       & ${} \pm 21 \pm 35	$                           &                       & $	240$                                           & ${}\pm55 \pm 59	$                            &                       & Focus                                                         &                       & \cite{Link:2003bd}                        \\ \cmidrule{6-14}
		                                         &                  &                              &                  &                                 & \cellcolor{Gray}      & \cellcolor{Gray}$2318$        & \cellcolor{Gray}$.2 \pm 16.9	$                & \cellcolor{Gray}      & $	\cellcolor{Gray} 267$                          & \cellcolor{Gray}$.4 \pm 35.6	$               & \cellcolor{Gray}      & \cellcolor{Gray} Our average                                  & \cellcolor{Gray}      &                                           \\ \midrule
		%												
		%D*0(2400)+-												
		\multirow{1}{*}{$D_{0}^{*}(2400)^{\pm}$} &                  & \multirow{1}{*}{$0^{+}$}     &                  & $D^{0}\pi^{+}$                  & \cellcolor{LightGray} & $	\cellcolor{LightGray}2403$  & \cellcolor{LightGray}${} \pm 14 \pm35	$       & \cellcolor{LightGray} & $	\cellcolor{LightGray}283$                      & \cellcolor{LightGray}${} \pm24 \pm34	$       & \cellcolor{LightGray} & \cellcolor{LightGray} Focus($m$ \& $\Gamma$) + Belle($J^{P}$) & \cellcolor{LightGray} & \cite{Link:2003bd} + \cite{Kuzmin:2006mw} \\ \midrule
		%												
		%D1(2420)0												
		\multirow{11}{*}{$D_{1}^{}(2420)^{0}$}   &                  & \multirow{11}{*}{$1^{+}$}    &                  & $D^{*+}\pi^{-}$                 &                       & $	2419$                       & $.6\pm0.1\pm0.7	$                             &                       & $	35$                                            & $.2\pm0.4\pm0.9	$                            &                       & LHCb                                                          &                       & \cite{Aaij:2013sza}                       \\
		                                         &                  &                              &                  & $D^{*+}\pi^{-}$                 &                       & $	2423$                       & $.1 \pm 1.5 ^{+0.4}_{-1.0}	$                  &                       & $	38$                                            & $.8\pm5^{+1.9}_{-5.4}	$                      &                       & Zeus                                                          &                       & \cite{Abramowicz:2012ys}                  \\
		                                         &                  &                              &                  & $D^{*+}\pi^{-}$                 &                       & $	2420$                       & $.1 \pm0.1 \pm0.8	$                           &                       & $	31$                                            & $.4\pm0.5\pm1.3	$                            &                       & \babar{}                                                      &                       & \cite{delAmoSanchez:2010vq}               \\
		                                         &                  &                              &                  & $D^{*+}\pi^{-}$                 &                       & $		$                          &                                               &                       & $	20$                                            & $.0\pm1.7\pm1.3	$                            &                       & CDF                                                           &                       & \cite{Abulencia:2005ry}                   \\
		                                         &                  &                              &                  & $D^{0}\pi^{+}\pi^{-}$           &                       & $	2426$                       & ${}\pm3 \pm1	$                                &                       & $	24$                                            & ${}\pm7\pm8	$                                &                       & Belle                                                         &                       & \cite{Abe:2004sm}                         \\
		                                         &                  &                              &                  & $D^{*+}\pi^{-}$                 &                       & $	2421$                       & $.4 \pm1.5 \pm 0.9	$                          &                       & $	23$                                            & $.7\pm2.7\pm4.0	$                            &                       & Belle                                                         &                       & \cite{Abe:2003zm}                         \\
		                                         &                  &                              &                  & $D^{*+}\pi^{-}$                 &                       & $	2421$                       & ${}^{+1}_{-2}\pm2	$                           &                       & $	20$                                            & ${}^{+6}_{-5}{}^{+3}_{-3}	$                  &                       & CLEO                                                          &                       & \cite{Avery:1994yc}                       \\
		                                         &                  &                              &                  & $D^{*+}\pi^{-}$                 &                       & $	2422$                       & ${} \pm2 \pm2	$                               &                       & $	15$                                            & ${}\pm8\pm4	$                                &                       & E687                                                          &                       & \cite{Frabetti:1993vv}                    \\
		                                         &                  &                              &                  & $D^{*+}\pi^{-}$                 &                       & $	2428$                       & ${}\pm3\pm2	$                                 &                       & $	23$                                            & ${}^{+8}_{-6}{}^{+10}_{-4}	$                 &                       & CLEO                                                          &                       & \cite{Avery:1989ui}                       \\
		                                         &                  &                              &                  & $D^{*+}\pi^{-}$                 &                       & $	2414$                       & ${}\pm2\pm5	$                                 &                       & $	13$                                            & ${}\pm6^{+10}_{-5}	$                         &                       & ARGUS                                                         &                       & \cite{Albrecht:1989pa}                    \\
		                                         &                  &                              &                  & $D^{*+}\pi^{-}$                 &                       & $	2428$                       & ${} \pm 8 \pm5	$                              &                       & $	58$                                            & ${}\pm14 \pm10	$                             &                       & TPS                                                           &                       & \cite{Anjos:1988uf}                       \\ \cmidrule{6-14}
		                                         &                  &                              &                  &                                 & \cellcolor{Gray}      & $	\cellcolor{Gray} 2420$      & \cellcolor{Gray}$.5 \pm 0.5	$                 & \cellcolor{Gray}      & $	\cellcolor{Gray} 31$                           & \cellcolor{Gray}$.7 \pm 0.7	$                & \cellcolor{Gray}      & \cellcolor{Gray} Our average                                  & \cellcolor{Gray}      &                                           \\ \midrule
		%D1(2420)+-												
		\multirow{5}{*}{$D_{1}^{}(2420)^{\pm}$}  &                  & \multirow{5}{*}{$1^{+}$}     &                  & $D^{*0}\pi^{+}$                 &                       & $	2421$                       & $.9\pm4.7^{+3.4}_{-1.2}	$                     &                       &                                                  &                                              &                       & Zeus                                                          &                       & \cite{Abramowicz:2012ys}                  \\
		                                         &                  &                              &                  & $D^{+}\pi^{-}\pi^{+}$           &                       & $	2421$                       & ${}\pm2\pm1	$                                 &                       & $	21$                                            & ${}\pm5\pm8	$                                &                       & Belle                                                         &                       & \cite{Abe:2004sm}                         \\
		                                         &                  &                              &                  & $D^{*0}\pi^{+}$                 &                       & $	2425$                       & ${}\pm2\pm2	$                                 &                       & $	26$                                            & ${}^{+8}_{-7}\pm4	$                          &                       & CLEO                                                          &                       & \cite{Bergfeld:1994af}                    \\
		                                         &                  &                              &                  & $D^{*0}\pi^{+}$                 &                       & $	2443$                       & ${} \pm7\pm5	$                                &                       & $	41$                                            & ${}\pm19\pm8	$                               &                       & TPS                                                           &                       & \cite{Anjos:1988uf}                       \\ \cmidrule{6-14}
		                                         &                  &                              &                  &                                 & \cellcolor{Gray}      & $	\cellcolor{Gray} 2423$      & \cellcolor{Gray}$.2 \pm 1.6	$\cellcolor{Gray} & \cellcolor{Gray}      & $	\cellcolor{Gray} 25$                           & \cellcolor{Gray}$.2 \pm 6.0	$                & \cellcolor{Gray}      & \cellcolor{Gray} Our average                                  & \cellcolor{Gray}      &                                           \\ \midrule
		%D*1(2430)0												
		\multirow{1}{*}{$D_{1}(2430)^{0}$}       &                  & \multirow{1}{*}{$1^{+}$}     &                  & $D^{*+}\pi^{-}$                 & \cellcolor{LightGray} & $	\cellcolor{LightGray}2427$  & \cellcolor{LightGray}${}\pm26\pm25	$          & \cellcolor{LightGray} & $	\cellcolor{LightGray} 384$                     & \cellcolor{LightGray}${}^{+107}_{-75}\pm74	$ & \cellcolor{LightGray} & \cellcolor{LightGray} Belle                                   & \cellcolor{LightGray} & \cite{Abe:2003zm}                         \\ \midrule
		%D*2(2460)0												
		\multirow{14}{*}{$D_{2}^{*}(2460)^{0}$}  &                  & \multirow{14}{*}{$2^{+}$}    &                  & $D^{*+}\pi^{-}$                 &                       & $	2460$                       & $.4\pm0.4\pm1.2	$                             &                       & $	43$                                            & $.2\pm1.2\pm3.0	$                            &                       & LHCb                                                          &                       & \cite{Aaij:2013sza}                       \\
		                                         &                  &                              &                  & $D^{+}\pi^{-}$                  &                       & $	2460$                       & $.4\pm0.1\pm0.1	$                             &                       & $	45$                                            & $.6\pm0.4\pm1.1	$                            &                       & LHCb                                                          &                       & \cite{Aaij:2013sza}                       \\
		                                         &                  &                              &                  & $D^{*+}\pi^{-}$, $D^{+}\pi^{-}$ &                       & $	2462$                       & $.5\pm2.4^{+1.3}_{-1.1}	$                     &                       & $	46$                                            & $.6\pm8.1^{+5.9}_{-3.8}	$                    &                       & Zeus                                                          &                       & \cite{Abramowicz:2012ys}                  \\
		                                         &                  &                              &                  & $D^{+}\pi^{-}$                  &                       & $	2462$                       & $.2\pm0.1\pm0.8	$                             &                       & $	50$                                            & $.5\pm0.6\pm0.7	$                            &                       & \babar{}                                                      &                       & \cite{delAmoSanchez:2010vq}               \\
		                                         &                  &                              &                  & $D^{+}\pi^{-}$                  &                       & $	2460$                       & $.4\pm1.2\pm2.2	$                             &                       & $	41$                                            & $.8\pm2.5\pm2.9	$                            &                       & \babar{}                                                      &                       & \cite{Aubert:2009wg}                      \\
		                                         &                  &                              &                  & $D^{+}\pi^{-}$                  &                       & $		$                          &                                               &                       & $	49$                                            & $.2\pm2.3\pm1.3	$                            &                       & CDF                                                           &                       & \cite{Abulencia:2005ry}                   \\
		                                         &                  &                              &                  & $D^{+}\pi^{-}$                  &                       & $	2461$                       & $.6\pm2.1\pm3.3	$                             &                       & $	45$                                            & $.6\pm4.4\pm6.7	$                            &                       & Belle                                                         &                       & \cite{Abe:2003zm}                         \\
		                                         &                  &                              &                  & $D^{+}\pi^{-}$                  &                       & $	2464$                       & $.5\pm1.1\pm1.9	$                             &                       & $	38$                                            & $.7\pm5.3\pm2.9	$                            &                       & Focus                                                         &                       & \cite{Link:2003bd}                        \\
		                                         &                  &                              &                  & $D^{+}\pi^{-}$                  &                       & $	2465$                       & ${}\pm3\pm3	$                                 &                       & $	28$                                            & ${}^{+8}_{-7}\pm6	$                          &                       & CLEO                                                          &                       & \cite{Avery:1994yc}                       \\
		                                         &                  &                              &                  & $D^{+}\pi^{-}$                  &                       & $	2453$                       & ${}\pm3\pm2	$                                 &                       & $	25$                                            & ${}\pm10\pm5	$                               &                       & E687                                                          &                       & \cite{Frabetti:1993vv}                    \\
		                                         &                  &                              &                  & $D^{*+}\pi^{-}$                 &                       & $	2461$                       & ${}\pm3\pm1	$                                 &                       & $	20$                                            & ${}^{+9}_{-12}{}^{+9}_{-10}	$                &                       & CLEO                                                          &                       & \cite{Avery:1989ui}                       \\
		                                         &                  &                              &                  & $D^{+}\pi^{-}$                  &                       & $	2455$                       & ${}\pm3\pm5	$                                 &                       & $	15$                                            & ${}^{+13}_{-10}{}^{+5}_{-10}	$               &                       & ARGUS                                                         &                       & \cite{Albrecht:1988dj}                    \\
		                                         &                  &                              &                  & $D^{+}\pi^{-}$                  &                       & $ 2459$                       & ${}\pm3\pm2$                                  &                       & $ 20$                                            & ${}\pm10\pm5$                                &                       & TPS                                                           &                       & \cite{Anjos:1988uf}                       \\	 \cmidrule{6-14}															
		\cmidrule{6-14}												
		                                         &                  &                              &                  &                                 & \cellcolor{Gray}      & $	\cellcolor{Gray} 2460$      & \cellcolor{Gray}$.47 \pm 0.14	$               & \cellcolor{Gray}      & $	\cellcolor{Gray} 47$                           & \cellcolor{Gray}$.7 \pm 0.7	$                & \cellcolor{Gray}      & \cellcolor{Gray} Our average                                  & \cellcolor{Gray}      &                                           \\ \midrule
		%D*2(2460)+-												
		\multirow{9}{*}{$D_{2}^{*}(2460)^{\pm}$} &                  & \multirow{9}{*}{$2^{+}$}     &                  & $D^{0}\pi^{+}$                  &                       & $	2463$                       & $.1\pm0.2\pm0.6	$                             &                       & $	48$                                            & $.6\pm1.3\pm1.9	$                            &                       & LHCb                                                          &                       & \cite{Aaij:2013sza}                       \\
		                                         &                  &                              &                  & $D^{*0}\pi^{+}$, $D^{0}\pi^{+}$ &                       & $	2460$                       & $.6\pm4.4^{+3.6}_{-0.8}	$                     &                       &                                                  &                                              &                       & Zeus                                                          &                       & \cite{Abramowicz:2012ys}                  \\
		                                         &                  &                              &                  & $D^{0}\pi^{+}$                  &                       & $	2465$                       & $.4\pm0.2\pm1.1	$                             &                       &                                                  &                                              &                       & \babar{}                                                      &                       & \cite{delAmoSanchez:2010vq}               \\
		                                         &                  &                              &                  & $D^{0}\pi^{+}$                  &                       & $	2465$                       & $.7\pm1.8^{+1.4}_{-4.8}	$                     &                       & $	49$                                            & $.7\pm3.8\pm6.4	$                            &                       & Belle                                                         &                       & \cite{Kuzmin:2006mw}                      \\
		                                         &                  &                              &                  & $D^{0}\pi^{+}$                  &                       & $	2467$                       & $.6\pm1.5\pm0.8	$                             &                       & $	34$                                            & $.1\pm6.5\pm4.2	$                            &                       & Focus                                                         &                       & \cite{Link:2003bd}                        \\
		                                         &                  &                              &                  & $D^{0}\pi^{+}$                  &                       & $	2463$                       & ${}\pm3\pm3	$                                 &                       & $	27$                                            & ${}^{+11}_{-8}\pm5	$                         &                       & CLEO                                                          &                       & \cite{Bergfeld:1994af}                    \\
		                                         &                  &                              &                  & $D^{0}\pi^{+}$                  &                       & $	2453$                       & ${}\pm3\pm2	$                                 &                       & $	23$                                            & ${}\pm9\pm5	$                                &                       & E687                                                          &                       & \cite{Frabetti:1993vv}                    \\
		                                         &                  &                              &                  & $D^{0}\pi^{+}$                  &                       & $	2469$                       & ${}\pm4\pm6	$                                 &                       &                                                  &                                              &                       & ARGUS                                                         &                       & \cite{Albrecht:1989gb}                    \\ \cmidrule{6-14}
		                                         &                  &                              &                  &                                 & \cellcolor{Gray}      & $	\cellcolor{Gray} 2463$      & \cellcolor{Gray}$.8 \pm 0.5	$                 & \cellcolor{Gray}      & $	\cellcolor{Gray}45$                            & \cellcolor{Gray}$.9 \pm 2.0	$                & \cellcolor{Gray}      & \cellcolor{Gray} Our average                                  & \cellcolor{Gray}      &                                           \\ \midrule
		%D(2550)^{0}												
		\multirow{1}{*}{$D(2550)^{0}$}           &                  & \multirow{1}{*}{$0^{-}$}     &                  & $D^{*+}\pi^{-}$                 & \cellcolor{LightGray} & $	\cellcolor{LightGray}2539$  & \cellcolor{LightGray}$.4\pm4.5\pm6.8	$        & \cellcolor{LightGray} & $	\cellcolor{LightGray}130$                      & \cellcolor{LightGray}${}\pm12\pm13	$         & \cellcolor{LightGray} & \cellcolor{LightGray} \babar{}                                & \cellcolor{LightGray} & \cite{delAmoSanchez:2010vq}               \\ \midrule
		%D(2550)^{0}												
		\multirow{1}{*}{$D(2580)^{0}$}           &                  & \multirow{1}{*}{Unnatural}   &                  & $D^{*+}\pi^{-}$                 & \cellcolor{LightGray} & $	\cellcolor{LightGray}2579$  & \cellcolor{LightGray}$.5\pm3.4\pm5.5	$        & \cellcolor{LightGray} & $	\cellcolor{LightGray} 117$                     & \cellcolor{LightGray}$.5\pm17.8\pm46.0	$     & \cellcolor{LightGray} & \cellcolor{LightGray} LHCb                                    & \cellcolor{LightGray} & \cite{Aaij:2013sza}                       \\ \midrule
		%D(2550)^{0}												
		\multirow{1}{*}{$D(2600)^{0}$}           &                  & \multirow{1}{*}{Natural}     &                  & $D^{+}\pi^{-}$                  & \cellcolor{LightGray} & $	\cellcolor{LightGray}2608$  & \cellcolor{LightGray}$.7\pm2.4\pm2.5	$        & \cellcolor{LightGray} & $	\cellcolor{LightGray} 93$                      & \cellcolor{LightGray}${}\pm6\pm13	$          & \cellcolor{LightGray} & \cellcolor{LightGray} \babar{}                                & \cellcolor{LightGray} & \cite{delAmoSanchez:2010vq}               \\ \midrule
		%D(2550)^{0}												
		\multirow{1}{*}{$D(2600)^{\pm}$}         &                  & \multirow{1}{*}{Natural}     &                  & $D^{0}\pi^{+}$                  & \cellcolor{LightGray} & $	\cellcolor{LightGray}2621$  & \cellcolor{LightGray}$.3\pm3.7\pm4.2	$        & \cellcolor{LightGray} & \cellcolor{LightGray}                            & \cellcolor{LightGray}                        & \cellcolor{LightGray} & \cellcolor{LightGray} \babar{}                                & \cellcolor{LightGray} & \cite{delAmoSanchez:2010vq}               \\ \midrule
		%D(2550)^{0}												
		\multirow{1}{*}{$D^{*}(2640)^{\pm}$}     &                  & \multirow{1}{*}{$1^{-}$}     &                  & $D^{*+}\pi^{+}\pi^{-}$          & \cellcolor{LightGray} & $	\cellcolor{LightGray}2637$  & \cellcolor{LightGray}${}\pm2\pm6	$             & \cellcolor{LightGray} & \cellcolor{LightGray}                            & \cellcolor{LightGray}                        & \cellcolor{LightGray} & \cellcolor{LightGray} Delphi                                  & \cellcolor{LightGray} & \cite{Abreu:1998vk}                       \\ \midrule
		%D(2550)^{0}												
		\multirow{1}{*}{$D^{*}(2650)^{0}$}       &                  & \multirow{1}{*}{Natural}     &                  & $D^{*+}\pi^{-}$                 & \cellcolor{LightGray} & $	\cellcolor{LightGray}2649$  & \cellcolor{LightGray}$.2\pm3.5\pm3.5	$        & \cellcolor{LightGray} & \cellcolor{LightGray}$	\cellcolor{LightGray}140$ & \cellcolor{LightGray}$.2\pm17.1\pm18.6	$     & \cellcolor{LightGray} & \cellcolor{LightGray} LHCb                                    & \cellcolor{LightGray} & \cite{Aaij:2013sza}                       \\ \midrule
		%												
		\multirow{1}{*}{$D(2740)^{0}$}           &                  & \multirow{1}{*}{Unnatural}   &                  & $D^{*+}\pi^{-}$                 & \cellcolor{LightGray} & $	\cellcolor{LightGray} 2737$ & \cellcolor{LightGray}$.0\pm3.5\pm11.2	$       & \cellcolor{LightGray} & \cellcolor{LightGray}$	\cellcolor{LightGray} 73$ & \cellcolor{LightGray}$.2\pm13.4\pm25.0	$     & \cellcolor{LightGray} & \cellcolor{LightGray} LHCb                                    & \cellcolor{LightGray} & \cite{Aaij:2013sza}                       \\ \midrule
		%D(2550)^{0}												
		\multirow{1}{*}{$D(2750)^{0}$}           &                  & \multirow{1}{*}{}            &                  & $D^{*+}\pi^{-}$                 & \cellcolor{LightGray} & $	\cellcolor{LightGray} 2752$ & \cellcolor{LightGray}$.4\pm1.7\pm2.7	$        & \cellcolor{LightGray} & \cellcolor{LightGray}$	\cellcolor{LightGray} 71$ & \cellcolor{LightGray}${}\pm6\pm11	$          & \cellcolor{LightGray} & \cellcolor{LightGray} \babar{}                                & \cellcolor{LightGray} & \cite{delAmoSanchez:2010vq}               \\ \midrule
		%D(2550)^{0}												
		\multirow{4}{*}{$D^{*}(2760)^{0}$}       &                  & \multirow{4}{*}{Natural}     &                  & $D^{*+}\pi^{-}$                 &                       & $	2761$                       & $.1\pm5.1\pm6.5	$                             &                       & $	74$                                            & $.4\pm3.4\pm37.0	$                           &                       & LHCb                                                          &                       & \cite{Aaij:2013sza}                       \\
		                                         &                  &                              &                  & $D^{+}\pi^{-}$                  &                       & $	2760$                       & $.1\pm1.1\pm3.7	$                             &                       & $	74$                                            & $.4\pm3.4\pm19.1	$                           &                       & LHCb                                                          &                       & \cite{Aaij:2013sza}                       \\
		                                         &                  &                              &                  & $D^{+}\pi^{-}$                  &                       & $	2763$                       & $.3\pm2.3\pm2.3	$                             &                       & $	60$                                            & $.9\pm5.1\pm3.6	$                            &                       & \babar{}                                                      &                       & \cite{delAmoSanchez:2010vq}               \\ \cmidrule{6-14}
		                                         &                  &                              &                  &                                 & \cellcolor{Gray}      & $	\cellcolor{Gray} 2761$      & \cellcolor{Gray}$.9 \pm 2.4	$                 & \cellcolor{Gray}      & $	\cellcolor{Gray} 62$                           & \cellcolor{Gray}$.5 \pm 5.9	$                & \cellcolor{Gray}      & \cellcolor{Gray} Our average                                  & \cellcolor{Gray}      &                                           \\ \midrule
		%												
		\multirow{3}{*}{$D^{*}(2760)^{\pm}$}     &                  & \multirow{1}{*}{}            &                  & $D^{0}\pi^{+}$                  &                       & $	2771$                       & $.7\pm1.7\pm3.8	$                             &                       & $	66$                                            & $.7\pm6.6\pm10.5	$                           &                       & LHCb                                                          &                       & \cite{Aaij:2013sza}                       \\
		                                         &                  &                              &                  & $D^{0}\pi^{+}$                  &                       & $	2769$                       & $.7\pm3.8\pm1.5	$                             &                       & $		$                                             &                                              &                       & \babar{}                                                      &                       & \cite{delAmoSanchez:2010vq}               \\ \cmidrule{6-14}
		                                         &                  &                              &                  &                                 & \cellcolor{Gray}      & $	\cellcolor{Gray} 2770$      & \cellcolor{Gray}$.7 \pm 2.9	$                 & \cellcolor{Gray}      & $	\cellcolor{Gray} 66$                           & \cellcolor{Gray}$.7\pm12.4	$                 & \cellcolor{Gray}      & \cellcolor{Gray} Our average                                  & \cellcolor{Gray}      &                                           \\ \bottomrule
	\end{tabular}
}
\end{adjustbox}

\caption{\label{tabel:charm:spect:2} Summary of recent measurements of mass and width for different excited $D$ mesons. The column $J^{P}$ list the most significant assignment of spin and parity. If possible an average mass or width is calculated.}
\end{table} 


\begin{table}[htb!]
\begin{center}
\begin{adjustbox}{width=\textwidth,center}
{\setlength\tabcolsep{0pt}
	\begin{tabular}{cp{5pt}cp{5pt}r@{}lp{5pt}cp{5pt}c}
		\toprule
		\rowcolor{Gray} Resonance&&Decay &&\multicolumn{2}{c}{$\mathcal{B}r[10^{-4}]$} & & \multicolumn{1}{c}{Measured by} &&  \multicolumn{1}{c}{Reference}
		\\ \midrule
		\multirow{3}{*}{$D_{0}^{*}(2400)^{0}$}       &   & \multirow{3}{*}{$B^{-}\to D_{0}^{*}(2400)^{0}(\to D^{+}\pi^{+})\pi^{-}$}                 &                       & $6$                                  & $.1\pm0.6\pm1.8$                                &                       & Belle                         &                       & \cite{Abe:2003zm}    \\ 
		                                             &   &                                                                                          &                       & $6$                                  & $.8\pm0.3\pm2.0$                                &                       & \babar{}                      &                       & \cite{Aubert:2009wg} \\  \cmidrule{4-9}
		                                             &   &                                                                                          & \cellcolor{Gray}      & \cellcolor{Gray}$6$                  & \cellcolor{Gray}$.4\pm 1.4$                     & \cellcolor{Gray}      & \cellcolor{Gray}  Our average & \cellcolor{Gray}      &                      \\ \midrule
		%
		\multirow{1}{*}{$D_{0}^{*}(2400)^{+}$}     &   & \multirow{1}{*}{$\overline{B}{}^{0}\to D_{0}^{*}(2400)^{+}(\to D^{0}\pi^{+})\pi^{-}$}    & \cellcolor{LightGray} & \cellcolor{LightGray} $0$            & \cellcolor{LightGray}$.60\pm0.13\pm0.27$        & \cellcolor{LightGray} & \cellcolor{LightGray} Belle   & \cellcolor{LightGray} & \cite{Kuzmin:2006mw} \\	\midrule		
		%
		\multirow{2}{*}{$D_{1}^{}(2420)^{0}$}        &   & \multirow{1}{*}{$B^{-}\to D_{1}^{}(2420)^{0}(\to D^{*+}\pi^{-})\pi^{-}$}                 & \cellcolor{LightGray} & \cellcolor{LightGray} $6$            & \cellcolor{LightGray}$.8\pm0.7\pm1.3$           & \cellcolor{LightGray} & \cellcolor{LightGray} Belle   & \cellcolor{LightGray} & \cite{Abe:2003zm}    \\	\cmidrule{4-9}
		                                             &   & \multirow{1}{*}{$B^{-}\to D_{1}^{}(2420)^{0}(\to D^{0}\pi^{+}\pi^{-})\pi^{-}$}           & \cellcolor{LightGray} & \cellcolor{LightGray} $1$            & \cellcolor{LightGray}$.85\pm0.29\pm0.27\pm0.41$ & \cellcolor{LightGray} & \cellcolor{LightGray} Belle   & \cellcolor{LightGray} & \cite{Abe:2004sm}    \\	\midrule					
		%				
		\multirow{1}{*}{$D_{1}^{}(2420)^{+}$}      &   & \multirow{1}{*}{$\overline{B}{}^{0}\to D_{1}(2420)^{+}(\to D^{+}\pi^{-}\pi^{+})\pi^{-}$} & \cellcolor{LightGray} & \cellcolor{LightGray} $0$            & \cellcolor{LightGray}$.89\pm0.15\pm0.22$        & \cellcolor{LightGray} & \cellcolor{LightGray} Belle   & \cellcolor{LightGray} & \cite{Abe:2004sm}    \\ \midrule
		%
		\multirow{1}{*}{$D_{1}(2430)^{0}$}           &   & $B^{-}\to D_{1}(2430)^{0}(\to D^{*+}\pi^{-})\pi^{-}$                                     & \cellcolor{LightGray} & \cellcolor{LightGray}$5$             & \cellcolor{LightGray}$.0\pm0.4\pm1.08$          & \cellcolor{LightGray} & \cellcolor{LightGray} Belle   & \cellcolor{LightGray} & \cite{Abe:2003zm}    \\ \midrule
		%%D*2(2460)0 	
		\multirow{4}{*}[-5pt]{$D_{2}^{*}(2460)^{0}$} &   & \multirow{3}{*}{$B^{-}\to D_{2}^{*}(2460)^{0}(\to D^{+}\pi^{-})\pi^{-}$}                 &                       & $3$                                  & $.4\pm0.3\pm0.7$                                &                       & Belle                         &                       & \cite{Abe:2003zm}    \\ 
		                                             &   &                                                                                          &                       & $3$                                  & $.5\pm0.2\pm0.5$                                &                       & \babar{}                      &                       & \cite{Aubert:2009wg} \\ \cmidrule{4-9}
		                                             &   &                                                                                          & \cellcolor{Gray}      & \cellcolor{Gray} $3\cellcolor{Gray}$ & \cellcolor{Gray}$.5 \pm 0.3$                    & \cellcolor{Gray}      & \cellcolor{Gray}  Our average & \cellcolor{Gray}      &                      \\ \cmidrule{3-9}
		                                             &   & \multirow{1}{*}{$B^{-}\to D_{2}^{*}(2460)^{0}(\to D^{*+}\pi^{-})\pi^{-}$}                & \cellcolor{LightGray} & \cellcolor{LightGray} $1$            & \cellcolor{LightGray}$.8\pm0.3\pm0.4$           & \cellcolor{LightGray} & \cellcolor{LightGray} Belle   & \cellcolor{LightGray} & \cite{Abe:2003zm}    \\ \midrule
		%
		\multirow{1}{*}{$D_{2}^{*}(2460)^{+}$}     &   & \multirow{1}{*}{$\overline{B}{}^{0}\to D_{2}^{*}(2460)^{+}(\to D^{0}\pi^{+})\pi^{-}$}    & \cellcolor{LightGray} & \cellcolor{LightGray} $2$            & \cellcolor{LightGray}$.15\pm0.17\pm0.31$        & \cellcolor{LightGray} & \cellcolor{LightGray} Belle   & \cellcolor{LightGray} & \cite{Kuzmin:2006mw} \\	\bottomrule	
	\end{tabular}
}
\end{adjustbox}

\caption{\label{table:charm:spect:3} Summary of branching fraction of $B$ mesons decays to excited $D$ mesons.}
\end{center}
\end{table} 

\begin{table}[htb!]
\begin{center}
\begin{tabular}{ccS[parse-numbers=false,separate-uncertainty=true,table-format=5.11]cc}
\toprule
\rowcolor{Gray} Resonance&Decay &\multicolumn{1}{c}{$\mathcal{B}r[10^{-4}]$} &  \multicolumn{1}{c}{Measured by} &  \multicolumn{1}{c}{Reference}
 \\ \midrule
\multirow{4}{*}[-5pt]{$D_{s0}^{*}(2317)^{\pm}$} & \multirow{3}{*}{$B^{0}\to D_{s0}^{*}(2317)^{+}(\to D^{+}_{s}\pi^{0})D^{-}$} & 8.6^{+3.3}_{-2.6}\pm2.6 & Belle &\cite{Krokovny:2003zq}\\ 
																		& & 18.\pm 4{}^{+6.7}_{-5} & \babar{} &\cite{Aubert:2004pw}\\  \cmidrule{3-4}
																	& &\cellcolor{Gray}10.8 \pm 3.4  &\cellcolor{Gray}  Our average &\\ \cmidrule{2-4}
							&\multirow{1}{*}{$B^{0}\to D_{s0}^{*}(2317)^{+}(\to D^{+}_{s}\pi^{0})K^{-}$} &\cellcolor{LightGray}  0.53^{+0.15}_{-0.13}\pm0.16&\cellcolor{LightGray}  Belle & \cite{Abe:2004wz} \\ \midrule
																	
%
\multirow{9}{*}[-5pt]{$D_{s1}(2460)^{\pm}$} & \multirow{3}{*}{$B^{0}\to D_{s1}(2460)^{+}(\to D^{*+}_{s}\pi^{0})D^{-}$} & 22.7^{+7.3}_{-6.2}\pm6.8 & Belle &\cite{Krokovny:2003zq}\\ 
																		& & 28.\pm 8{}^{+11.2}_{-7.8} & \babar{} &\cite{Aubert:2004pw}\\  \cmidrule{3-4}
																	& &\cellcolor{Gray}24.7 \pm 7.6  &\cellcolor{Gray}  Our average &\\ \cmidrule{2-4}
%
& \multirow{3}{*}{$B^{0}\to D_{s1}(2460)^{\pm}(\to D^{*+}_{s}\gamma)D^{-}$} & 8.2^{+2.2}_{-1.9}\pm2.5 & Belle &\cite{Krokovny:2003zq}\\ 
																		& & 8.\pm 2{}^{+3.2}_{-2.3} & \babar{} &\cite{Aubert:2004pw}\\  \cmidrule{3-4}
																	& &\cellcolor{Gray}8.1 \pm 2.3 &\cellcolor{Gray}  Our average &\\ \cmidrule{2-4}
							& $D_{s1}(2460)^{+}\to D^{*-}_{s}\pi^{0}$&\cellcolor{LightGray} (56.\pm13\pm9)\%&\cellcolor{LightGray} \babar{}&\cite{Aubert:2006nm} \\ \cmidrule{2-4}								
							& $D_{s1}(2460)^{+}\to D^{*-}_{s}\gamma$&\cellcolor{LightGray} (16.\pm4\pm3)\%&\cellcolor{LightGray} \babar{}&\cite{Aubert:2006nm} \\ \midrule					
%
								 & \multirow{1}{*}{$B^{0}\to D_{s1}(2536)^{+}(\to D^{*0}K^{+})D^{-}$} &\cellcolor{LightGray} 1.71\pm0.48\pm0.32 &\cellcolor{LightGray} \babar{} &\cite{Aubert:2007rva}\\ \cmidrule{2-4}
								& \multirow{1}{*}{$B^{0}\to D_{s1}(2536)^{+}(\to D^{*+}K^{0})D^{-}$} &\cellcolor{LightGray} 2.61\pm1.03\pm0.31 &\cellcolor{LightGray} \babar{} &\cite{Aubert:2007rva}\\ \cmidrule{2-4}
							 &\multirow{1}{*}{$B^{0}\to D_{s1}(2536)^{+}(\to D^{*0}K^{+})D^{*-}$} &\cellcolor{LightGray} 3.32\pm0.88\pm0.66 &\cellcolor{LightGray} \babar{} &\cite{Aubert:2007rva}\\ \cmidrule{2-4}
\multirow{2}{*}{$D_{s1}(2536)^{\pm}$}&\multirow{1}{*}{$B^{0}\to D_{s1}(2536)^{+}(\to D^{*+}K^{0})D^{*-}$} &\cellcolor{LightGray} 5.00\pm1.51\pm0.67 &\cellcolor{LightGray} \babar{} &\cite{Aubert:2007rva}\\ \cmidrule{2-4}
							 &\multirow{1}{*}{$B^{+}\to D_{s1}(2536)^{+}(\to D^{*0}K^{+})\overline{D}^{0}$} &\cellcolor{LightGray} 2.16\pm0.52\pm0.45 &\cellcolor{LightGray} \babar{} &\cite{Aubert:2007rva}\\ \cmidrule{2-4}
							&\multirow{1}{*}{$B^{+}\to D_{s1}(2536)^{+}(\to D^{*+}K^{0})\overline{D}^{0}$} &\cellcolor{LightGray} 2.30\pm0.98\pm0.43 &\cellcolor{LightGray} \babar{} &\cite{Aubert:2007rva}\\ \cmidrule{2-4}
							 &\multirow{1}{*}{$B^{+}\to D_{s1}(2536)^{+}(\to D^{*0}K^{+})\overline{D}^{*0}$} &\cellcolor{LightGray} 5.46\pm1.17\pm1.04 &\cellcolor{LightGray} \babar{} &\cite{Aubert:2007rva}\\ \cmidrule{2-4}
							&\multirow{1}{*}{$B^{+}\to D_{s1}(2536)^{+}(\to D^{*+}K^{0})\overline{D}^{*0}$}  &\cellcolor{LightGray} 3.92\pm2.46\pm0.83 &\cellcolor{LightGray} \babar{} &\cite{Aubert:2007rva}\\ \midrule
%
\multirow{1}{*}{$D_{s1}(2700)^{\pm}$} & \multirow{1}{*}{$B^{+}\to D_{s1}(2700)^{+}(\to D^{0}K^{+})\overline{D}{}^{0}$} &\cellcolor{LightGray}  11.3\pm2.2{}^{+1.4}_{-2.8} &\cellcolor{LightGray}  Belle &\cite{Brodzicka:2007aa}\\ \bottomrule
\end{tabular}
\caption{\label{table:charm:spect:4} Summary of branching fraction of $B$ mesons decays to excited $D_{s}$ mesons.}
\end{center}
\end{table} 


\begin{table}[htb!]
\begin{center}
{\setlength\tabcolsep{0pt}
	\begin{tabular}{cp{5pt}cp{5pt}r@{}lp{5pt}cp{5pt}c}
		\toprule
		\rowcolor{Gray}  Resonance                      &   & Relative to                                  && \multicolumn{2}{c}{$\Delta m$ [MeV$/c^{2}$]} & & \multicolumn{1}{c}{Measured by} && \multicolumn{1}{c}{Reference} 
		\\ \midrule
		%D*0(2400)0 	
		\multirow{3}{*}{$D_{1}^{*}(2420)^{0}$}          &   & \multirow{3}{*}{$D^{*+}$}                    &                       & $	410$                       & $.2\pm2.1\pm0.9	$                              &                       & Zeus                          &                       & \cite{Chekanov:2008ac}  \\
		                                                &   &                                              &                       & $	411$                       & $.7\pm0.7\pm0.4	$                              &                       & CDF                           &                       & \cite{Abulencia:2005ry} \\ \cmidrule{4-9}
		                                                &   &                                              & \cellcolor{Gray}      & \cellcolor{Gray} $411$       & \cellcolor{Gray}$.5 \pm 0.8	$                  & \cellcolor{Gray}      & \cellcolor{Gray}  Our average & \cellcolor{Gray}      &                         \\ \midrule
		%								
													
		\multirow{1}{*}{$D_{1}^{}(2420)^{\pm}$}         &   & \multirow{1}{*}{$D_{1}^{*}(2420)^{0}$}       & \cellcolor{LightGray} & $	\cellcolor{LightGray}4$    & \cellcolor{LightGray}${}^{+2}_{-3}\pm3	$       & \cellcolor{LightGray} & \cellcolor{LightGray} CLEO    & \cellcolor{LightGray} & \cite{Bergfeld:1994af}  \\ \midrule
													
		\multirow{2}{*}{$D_{2}^{*}(2460)^{0}$}          &   & $D^{+}$                                      & \cellcolor{LightGray} & $	\cellcolor{LightGray}593$  & \cellcolor{LightGray}$.9\pm0.6\pm0.5	$         & \cellcolor{LightGray} & \cellcolor{LightGray}CDF      & \cellcolor{LightGray} & \cite{Abulencia:2005ry} \\  \cmidrule{3-9}
		                                                &   & $D^{*+}$                                     & \cellcolor{LightGray} & $	\cellcolor{LightGray} 458$ & \cellcolor{LightGray}$.8\pm3.7^{+1.2}_{-1.3}	$ & \cellcolor{LightGray} & \cellcolor{LightGray} Zeus    & \cellcolor{LightGray} & \cite{Chekanov:2008ac}  \\ \midrule
		%D*2(2460)+-								
		\multirow{4}{*}[-2pt]{$D_{2}^{*}(2460)^{\pm}$}  &   & \multirow{4}{*}[-2pt]{$D_{2}^{*}(2460)^{0}$} &                       & $	3$                         & $.1\pm1.9\pm0.9	$                              &                       & Focus                         &                       & \cite{Link:2003bd}      \\
		                                                &   &                                              &                       & $-2$                         & ${}\pm4\pm4	$                                  &                       & CLEO                          &                       & \cite{Bergfeld:1994af}  \\
		                                                &   &                                              &                       & $	14$                        & ${}\pm5\pm8	$                                  &                       & ARGUS                         &                       & \cite{Albrecht:1989gb}  \\ \cmidrule{4-9}
		                                                &   &                                              & \cellcolor{Gray}      & $	\cellcolor{Gray}  3$       & \cellcolor{Gray}$.0 \pm 1.9	$                  & \cellcolor{Gray}      & \cellcolor{Gray} Our average  & \cellcolor{Gray}      &                         \\ \midrule
		%D(2550)^{0}								
													
													
		\multirow{4}{*}[-2pt]{$D_{s0}^{*}(2317)^{\pm}$} &   & \multirow{4}{*}[-2pt]{$D_{s}^{\pm}$}         &                       & $	348$                       & $.7\pm0.5\pm0.7	$                              &                       & Belle                         &                       & \cite{Abe:2003jk}       \\
		                                                &   &                                              &                       & $	350$                       & $.0\pm1.2\pm1.0	$                              &                       & CLEO                          &                       & \cite{Besson:2003cp}    \\
		                                                &   &                                              &                       & $	351$                       & $.3\pm2.1\pm1.9	$                              &                       & Belle                         &                       & \cite{Krokovny:2003zq}  \\ \cmidrule{4-9}
		                                                &   &                                              & \cellcolor{Gray}      & $	\cellcolor{Gray}  349$     & \cellcolor{Gray}$.2 \pm 0.7	$                  & \cellcolor{Gray}      & \cellcolor{Gray} Our average  & \cellcolor{Gray}      &                         \\ \midrule
		\multirow{7}{*}[-5pt]{$D_{s1}(2460)^{\pm}$}     &   & \multirow{4}{*}[-2pt]{$D_{s}^{*\pm}$}        &                       & $	344$                       & $.1\pm1.3\pm1.1	$                              &                       & Belle                         &                       & \cite{Abe:2003jk}       \\
		                                                &   &                                              &                       & $	351$                       & $.2\pm1.7\pm1.0	$                              &                       & CLEO                          &                       & \cite{Besson:2003cp}    \\
		                                                &   &                                              &                       & $	346$                       & $.8\pm1.6\pm1.9	$                              &                       & Belle                         &                       & \cite{Krokovny:2003zq}  \\ \cmidrule{4-9}
		                                                &   &                                              & \cellcolor{Gray}      & $	\cellcolor{Gray}  347$     & \cellcolor{Gray}$.1 \pm 1.1	$                  & \cellcolor{Gray}      & \cellcolor{Gray} Our average  & \cellcolor{Gray}      &                         \\ \cmidrule{3-9}
		                                                &   & \multirow{3}{*}[-2pt]{ $D_{s}^{\pm}$ }       &                       & $	491$                       & $.0\pm1.3\pm1.9	$                              &                       & Belle                         &                       & \cite{Abe:2003jk}       \\
		                                                &   &                                              &                       & $	491$                       & $.4\pm0.9\pm1.5	$                              &                       & Belle                         &                       & \cite{Abe:2003jk}       \\ \cmidrule{4-9}
		                                                &   &                                              & \cellcolor{Gray}      & $	\cellcolor{Gray}  491$     & \cellcolor{Gray}$.3 \pm 1.4	$                  & \cellcolor{Gray}      & \cellcolor{Gray} Our average  & \cellcolor{Gray}      &                         \\ \midrule
													
		\multirow{5}{*}[-4pt]{$D_{s1}(2536)^{\pm}$}     &   & \multirow{4}{*}[-2pt]{  $D^{*}(2010)^{\pm}$} &                       & $	524$                       & $.83\pm0.01\pm0.04	$                           &                       & \babar{}                      &                       & \cite{Lees:2011um}      \\
		                                                &   &                                              &                       & $	525$                       & $.30_{-0.41}^{+0.44}\pm0.10	$                  &                       & Zeus                          &                       & \cite{Chekanov:2008ac}  \\
		                                                &   &                                              &                       & $	525$                       & $.3\pm0.6\pm0.1	$                              &                       & ALEPH                         &                       & \cite{Heister:2001nj}   \\ \cmidrule{4-9}
		                                                &   &                                              & \cellcolor{Gray}      & $	\cellcolor{Gray} 524$      & \cellcolor{Gray}$.84 \pm 0.04	$                & \cellcolor{Gray}      & \cellcolor{Gray} Our average  & \cellcolor{Gray}      &                         \\ \cmidrule{3-9}
		                                                &   & \multirow{1}{*}{ $D^{*}(2007)^{0}$}          & \cellcolor{LightGray} & $	\cellcolor{LightGray} 528$ & \cellcolor{LightGray}$.7\pm1.9\pm0.5	$         & \cellcolor{LightGray} & \cellcolor{LightGray}ALEPH    & \cellcolor{LightGray} & \cite{Heister:2001nj}   \\ \midrule
													
		\multirow{1}{*}{$D_{s2}^{*}(2573)^{\pm}$}       &   & $D^{0}$                                      & \cellcolor{LightGray} & $	\cellcolor{LightGray} 704$ & \cellcolor{LightGray}${}\pm3\pm1	$             & \cellcolor{LightGray} & \cellcolor{LightGray}ALEPH    & \cellcolor{LightGray} & \cite{Heister:2001nj}   \\ \bottomrule
	\end{tabular}}

\caption{\label{table:charm:spect:5} Summary of mass difference measurements for excited $D$ mesons.}
\end{center}
\end{table} 

\begin{table}[htb!]
\begin{center}
\begin{tabular}{cS[parse-numbers=false,separate-uncertainty=true,table-format=3.11]cc}
\toprule
\rowcolor{Gray} Resonance &\multicolumn{1}{c}{$A_{D}$} &  \multicolumn{1}{c}{Measured by} &  \multicolumn{1}{c}{Reference}
 \\ \midrule
\multirow{5}{*}{$D_{1}^{}(2420)^{0}$} &  7.8_{-2.7}^{+6.7}{}_{-1.8}^{+4.6}& ZEUS &\cite{Abramowicz:2012ys}\\
				      & 5.72\pm0.25& \babar{} &\cite{delAmoSanchez:2010vq}\\  
				      & 5.9_{-1.7}^{+3.}{}_{-1.}^{+2.4}& ZEUS &\cite{Chekanov:2008ac}\\  
				      & 3.8\pm0.6\pm0.8& \babar{} &\cite{Aubert:2008zc}\\  \cmidrule{2-3}
				      &\cellcolor{Gray}  5.61 \pm 0.24&\cellcolor{Gray}  Our average &\\ \midrule
%
\multirow{1}{*}{$D_{1}^{}(2420)^{\pm}$} &\cellcolor{LightGray}  3.8\pm0.6\pm0.8& \cellcolor{LightGray} \babar{}&\cite{Aubert:2008zc}\\	\midrule		
%
\multirow{1}{*}{$D_{2}^{*}(2460)^{0}$} & \cellcolor{LightGray} -1.16\pm0.35 &\cellcolor{LightGray} ZEUS &\cite{Abramowicz:2012ys}\\	\midrule
\multirow{1}{*}{$D_{}^{}(2750)^{0}$} &\cellcolor{LightGray} -0.33\pm0.28 &\cellcolor{LightGray} \babar{} &\cite{delAmoSanchez:2010vq}\\						
\bottomrule
\end{tabular}
\caption{\label{table:charm:spect:6}Summary of measurements of polarization amplitudes for excited $D$ mesons.}
\end{center}
\end{table} 

