\subsection{\emph{$D^0$-$\dbar$} mixing and \emph{\cp}\ violation}
\label{sec:charm:mixcpv}

\subsubsection{Introduction}

In 2007 Belle~\cite{Staric:2007dt} and \babar~\cite{Aubert:2007wf} 
obtained the first evidence for $D^0$-$\dbar$ mixing, which 
had been searched for for more than two decades. 
These results were later confirmed by CDF~\cite{Aaltonen:2007uc}.
There are now numerous measurements of $D^0$-$\dbar$ mixing 
with various levels of sensitivity. All the results are
input into a global fit to determine
%combined to yield 
world averages of mixing parameters, \cp-violation (\cpv) 
parameters, and strong phases.

Our notation is as follows.
The mass eigenstates are denoted 
$D^{}_1 = p|D^0\rangle-q|\dbar\rangle$ and
$D^{}_2 = p|D^0\rangle+q|\dbar\rangle$, 
where we use the convention 
$CP|D^0\rangle=-|\dbar\rangle$ and 
$CP|\dbar\rangle=-|D^0\rangle$. Thus in the absence of 
\cp\ violation, $D^{}_1$ is \cp-even and $D^{}_2$ is \cp-odd.
The weak phase $\phi\equiv {\rm Arg}(q/p)$.
The mixing parameters are defined as 
$x\equiv(m^{}_1-m^{}_2)/\Gamma$ and 
$y\equiv (\Gamma^{}_1-\Gamma^{}_2)/(2\Gamma)$, where 
$m^{}_1,\,m^{}_2$ and $\Gamma^{}_1,\,\Gamma^{}_2$ are
the masses and decay widths for the mass eigenstates,
and $\Gamma\equiv (\Gamma^{}_1+\Gamma^{}_2)/2$. 


The global fit determines central values and errors for
ten underlying parameters. These consist of mixing parameters
$x$ and $y$; a parameter describing the ratio of decay rates
$R^{}_D\equiv\left|{\cal A}(D^0\ra K^+\pi^-)/
              {\cal A}(\dbar\ra K^+\pi^-)\right|^2$;
\cpv\ parameters $|q/p|$, $\phi$, and
$A^{}_D\equiv (R^+_D-R^-_D)/(R^+_D+R^-_D)$, where the $+\,(-)$
superscript corresponds to $D^0\,(\dbar)$ decays; 
{\it direct\/} \cpv\ parameters $A^{}_{KK}$ and 
$A^{}_{\pi\pi}$ (discussed below); 
the strong phase difference
$\delta$ between $\dbar\ra K^-\pi^+$ and 
$D^0\ra K^-\pi^+$ amplitudes; and 
the strong phase difference $\delta^{}_{K\pi\pi}$ between 
$\dbar\ra K^-\rho^+$ and $D^0\ra K^-\rho^+$ amplitudes. 

The fit uses 38 observables taken from 
measurements of \dklnu, \dkk\ and \dpipi, \dkpi, 
$D^0\ra K^+\pi^-\pi^0$, %$D^0\ra K^+\pi^-\pi^+\pi^-$, 
\dkspp, and \dkskk\ decays,\footnote{Charge-conjugate modes
are implicitly included.} and from double-tagged branching 
fractions measured at the $\psi(3770)$ resonance. Correlations 
among observables are accounted for by using covariance matrices 
provided by the experimental collaborations. Errors are assumed
to be Gaussian, and systematic errors among different experiments 
are assumed uncorrelated unless specific correlations have been 
identified.
We have checked this method with a second method that adds
together three-dimensional log-likelihood functions 
for $x$, $y$, and $\delta$ obtained from several analyses;
this combination accounts for non-Gaussian errors.
When both methods are applied to the same set of 
measurements, equivalent results are obtained. 


Mixing in heavy flavor systems such as those of $B^0$ and $B^0_s$ 
is governed by a short-distance box diagram. In the $D^0$ system,
however, this diagram is doubly-Cabibbo-suppressed relative to 
amplitudes dominating the decay width, and it is also GIM-suppressed.
Thus the short-distance mixing rate is tiny, and $D^0$-$\dbar$ 
mixing is expected to be dominated by long-distance processes. 
These are difficult to calculate reliably, and theoretical estimates 
for $x$ and $y$ range over two-three orders of 
magnitude~\cite{Bigi:2000wn,Petrov:2003un,Petrov:2004rf,Falk:2004wg}.


With the exception of $\psi(3770)\ra DD$ measurements, all methods 
identify the flavor of the $D^0$ or $\dbar$ when produced by 
reconstructing the decay $D^{*+}\ra D^0\pi^+$ or $D^{*-}\ra\dbar\pi^-$. 
The charge of the pion, which has low momentum and is usually 
referred to as the ``soft'' pion~$\pi^{}_s$,
identifies the $D$ flavor. For signal 
decays, $M^{}_{D^*}-M^{}_{D^0}-M^{}_{\pi^+}\equiv Q\approx 6$\meve, 
which is close to the threshold; thus analyses typically
require that the reconstructed $Q$ be small to suppress backgrounds. 
For time-dependent measurements, the $D^0$ decay time is 
calculated as $(d/p)\times M^{}_{D^0}$, where $d$ is
the distance between the $D^*$ and $D^0$ decay vertices and 
$p$ is the $D^0$ momentum. The $D^*$ vertex position is 
taken to be at the primary vertex for $\bar{p}p$ collider 
experiments~\cite{Aaltonen:2007uc}, and at the intersection 
of the $D^0$ momentum vector with the beamspot profile for 
$e^+e^-$ experiments.


\subsubsection{Input observables}

The global fit determines central values and errors for
the underlying parameters using a $\chi^2$ statistic.
The fitted parameters are 
$x$, $y$, $R^{}_D$, $A^{}_D$, $|q/p|$, $\phi$, $\delta$, $\delta^{}_{K\pi\pi}$,
$A^{}_{KK}$ and $A^{}_{\pi\pi}$.
The parameter $\delta^{}_{K\pi\pi}$ is the strong phase 
difference between the amplitudes ${\cal A}(\dbar\ra K^+\rho^-)$ 
and ${\cal A}(D^0\ra K^+\rho^-)$. In the $D\ra K^+\pi^-\pi^0$ 
Dalitz plot analysis that provides sensitivity to $x$ and $y$, 
the $\dbar\ra K^+\pi^-\pi^0$ isobar phases are determined 
relative to that for ${\cal A}(\dbar\ra K^+\rho^-)$, and 
the $D^0\ra K^+\pi^-\pi^0$ isobar phases are determined 
relative to that for ${\cal A}(D^0\ra K^+\rho^-)$. 
As the $\dbar$ and $D^0$ Dalitz plots are fit independently, 
the phase difference $\delta^{}_{K\pi\pi}$ between the
two ``normalizing'' amplitudes cannot be determined
from these fits.

All input measurements are listed in 
Tables~\ref{tab:observables1}-\ref{tab:observables3}. 
The observable $R^{}_M=(x^2+y^2)/2$ is calculated from \dklnu\ 
decays~\cite{Aitala:1996vz,Cawlfield:2005ze,Aubert:2007aa,Bitenc:2008bk}
and is the world average (WA) value calculated by HFAG~\cite{HFAG_charm:webpage}.
The inputs used for these averages are plotted in Fig.~\ref{fig:rm_semi}. 
The observables $y^{}_{CP}$ and $A^{}_\Gamma$ are also HFAG WA 
values~\cite{HFAG_charm:webpage}; the inputs used for these 
averages are plotted in Figs.~\ref{fig:ycp} and \ref{fig:Agamma}.
%\cite{Staric:2007dt,ycp_fnal,ycp_cleo,ycp_babar} 
The \dkpi\ observables used are from Belle~\cite{Zhang:2006dp}, 
\babar~\cite{Aubert:2007wf}, and CDF~\cite{Aaltonen:2007uc};
earlier measurements have much less precision and are not used.
The observables from \dkspp\ decays for no-\cpv\ are from 
Belle~\cite{Abe:2007rd} and \babar~\cite{delAmoSanchez:2010xz}, 
but for the \cpv-allowed case only Belle measurements~\cite{Abe:2007rd} 
are available. The $D^0\ra K^+\pi^-\pi^0$ results are from 
\babar~\cite{Aubert:2008zh}, and the $\psi(3770)\ra\overline{D}D$ 
results are from CLEOc~\cite{Sun:2010zz}.


\begin{figure}
\begin{center}
%\includegraphics[width=4.2in]{figures/charm/rm_semi_9mar08.eps}
\end{center}
\vskip-0.20in
\caption{\label{fig:rm_semi}
World average value of $R^{}_M$ from Ref.~\cite{HFAG_charm:webpage},
as calculated from $D^0\ra K^+\ell^-\nu$ 
measurements~\cite{Aitala:1996vz,Cawlfield:2005ze,Aubert:2007aa,Bitenc:2008bk}. }
\end{figure}

\begin{figure}
\begin{center}
%\includegraphics[width=4.2in]{figures/charm/ycp_5apr12.eps}
\end{center}
\vskip-0.20in
\caption{\label{fig:ycp}
World average value of $y^{}_{CP}$ from Ref.~\cite{HFAG_charm:webpage}, 
as calculated from \dkkpp\ 
measurements~\cite{Staric:2007dt,Aitala:1999dt,Link:2000cu,
Csorna:2001ww,Aubert:2007en,Zupanc:2009sy,Aaij:2011ad}.  }
\end{figure}


\begin{figure}
\begin{center}
%\includegraphics[width=4.2in]{figures/charm/a_gamma_5apr12.eps}
\end{center}
\vskip-0.20in
\caption{\label{fig:Agamma}
World average value of $A^{}_\Gamma$ from Ref.~\cite{HFAG_charm:webpage}, 
as calculated from \dkkpp\ 
measurements~\cite{Staric:2007dt,Aubert:2007en,Aaij:2011ad}.  }
\end{figure}


The relationships between the observables and the fitted
parameters are listed in Table~\ref{tab:relationships}. 
For each set of correlated observables we construct a
difference vector $\vec{V}$; e.g., for 
$D^0\ra K^0_S\,\pi^+\pi^-$ decays,
$\vec{V}=(\Delta x,\Delta y,\Delta |q/p|,\Delta \phi)$
where $\Delta$ represents the difference between the 
measured value and the fitted value. The 
contribution of a set of observables to the $\chi^2$ 
is calculated as $\vec{V}\cdot (M^{-1})\cdot\vec{V}^T$, 
where $M^{-1}$ is the inverse of the covariance matrix 
for the measurement. Covariance matrices are constructed 
from the correlation coefficients among the measured observables.
These coefficients (where applicable) are also listed in 
Tables~\ref{tab:observables1}-\ref{tab:observables3}. 

\begin{table}
\renewcommand{\arraystretch}{1.3}
\renewcommand{\arraycolsep}{0.02in}
\renewcommand{\tabcolsep}{0.05in}
\caption{\label{tab:observables1}
All observables used in the global fit except those from \dkpi\ 
and those used for measuring direct \cpv, from
%Observables from \dkk, \dpipi, \dkspp, \dkskk, \dklnu, $D^0\ra K^+\pi^-\pi^0$, 
%and $\psi(3770)\ra\overline{D}D$ used for the global fit, from 
Refs.~\cite{Staric:2007dt,
Aitala:1996vz,
Cawlfield:2005ze,
Aubert:2007aa,
Bitenc:2008bk,
Abe:2007rd,
delAmoSanchez:2010xz,
Aubert:2008zh,
Sun:2010zz,
Aitala:1999dt,
Link:2000cu,
Csorna:2001ww,
Aubert:2007en}.}
\vspace*{6pt}
\footnotesize
%\hskip-0.10in
\resizebox{0.99\textwidth}{!}{
\begin{tabular}{l|ccc}
%\begin{tabular*}{1.0\textwidth}{lccc}
%{@{\extracolsep{\fill}} l | c c c }
\hline
{\bf Mode} & \textbf{Observable} & {\bf Values} & {\bf Correlation coefficients} \\
\hline
\begin{tabular}{l}  
$D^0\ra K^+K^-/\pi^+\pi^-$, \\
\hskip0.30in $\phi\,K^0_S$~\cite{HFAG_charm:webpage} 
\end{tabular}
&
\begin{tabular}{c}
 $y^{}_{CP}$  \\
 $A^{}_{\Gamma}$
\end{tabular} & 
$\begin{array}{c}
(1.064\pm 0.209)\% \\
(0.026\pm 0.231)\% 
\end{array}$   & \\ 
\hline
\begin{tabular}{l}  
$D^0\ra K^0_S\,\pi^+\pi^-$~\cite{HFAG_charm:webpage} \\
\ (Belle+CLEO WA: \\
\ \ no \cpv\ or \\
\ \ no direct \cpv)
\end{tabular}
&
\begin{tabular}{c}
$x$ \\
$y$ \\
$|q/p|$ \\
$\phi$ 
\end{tabular} & 
\begin{tabular}{c}
 $(0.811\pm 0.334)\%$ \\
 $(0.309\pm 0.281)\%$ \\
 $0.95\pm 0.22^{+0.10}_{-0.09}$ \\
 $(-0.035\pm 0.19\pm 0.09)$ rad
\end{tabular} &  \\ 
 & & \\
\begin{tabular}{l}  
$D^0\ra K^0_S\,\pi^+\pi^-$~\cite{Abe:2007rd} \\
\ (Belle: \\
\ \ \cpv-allowed)
\end{tabular}
&
\begin{tabular}{c}
$x$ \\
$y$ \\
$|q/p|$ \\
$\phi$  
\end{tabular} & 
\begin{tabular}{c}
 $(0.81\pm 0.30^{+0.13}_{-0.17})\%$ \\
 $(0.37\pm 0.25^{+0.10}_{-0.15})\%$ \\
 $0.86\pm 0.30^{+0.10}_{-0.09}$ \\
 $(-0.244\pm 0.31\pm 0.09)$ rad
\end{tabular} &
\begin{tabular}{l} \hskip-0.15in
$\left\{ \begin{array}{cccc}
 1 &  -0.007 & -0.255\alpha & 0.216  \\
 -0.007 &  1 & -0.019\alpha & -0.280 \\
 -0.255\alpha &  -0.019\alpha & 1 & -0.128\alpha  \\
  0.216 &  -0.280 & -0.128\alpha & 1 
\end{array} \right\}$ \\
\hskip0.10in ($\alpha=(|q/p|+1)^2/2$ is a \\ 
\hskip0.30in transformation factor)
\end{tabular} \\
 & & \\
\begin{tabular}{l}  
$D^0\ra K^0_S\,\pi^+\pi^-$~\cite{delAmoSanchez:2010xz} \\
\hskip0.30in $K^0_S\,K^+ K^-$ \\
\ (\babar: no \cpv) 
\end{tabular}
&
\begin{tabular}{c}
$x$ \\
$y$ 
\end{tabular} & 
\begin{tabular}{c}
 $(0.16\pm 0.23\pm 0.12\pm 0.08)\%$ \\
 $(0.57\pm 0.20\pm 0.13\pm 0.07)\%$ 
\end{tabular} &  $0.0615$ \\ 
\hline
\begin{tabular}{l}  
$D^0\ra K^+\ell^-\nu$~\cite{HFAG_charm:webpage}
\end{tabular} 
  & $R^{}_M$  & $(0.0130\pm 0.0269)\%$  &  \\ 
\hline
\begin{tabular}{l}  
$D^0\ra K^+\pi^-\pi^0$ 
\end{tabular} 
&
\begin{tabular}{c}
$x''$ \\ 
$y''$ 
\end{tabular} &
\begin{tabular}{c}
$(2.61\,^{+0.57}_{-0.68}\,\pm 0.39)\%$ \\ 
$(-0.06\,^{+0.55}_{-0.64}\,\pm 0.34)\%$ 
\end{tabular} & $-0.75$ \\
\hline
\begin{tabular}{c}  
$\psi(3770)\ra\overline{D}D$ \\
(CLEOc)
\end{tabular}
&
\begin{tabular}{c}
$x^2$ \\
$y$ \\
$R^{}_D$ \\
$2\sqrt{R^{}_D}\cos\delta$ \\
$2\sqrt{R^{}_D}\sin\delta$ 
\end{tabular} & 
\begin{tabular}{c}
$(0.1549 \pm 0.2223)\%$ \\
$(2.997 \pm 2.293)\%$ \\
$(0.4118 \pm 0.0948)\%$ \\
$(12.64 \pm 2.86)\%$ \\
$(-0.5242 \pm 6.426)\%$ 
\end{tabular} &
$\left\{ \begin{array}{ccccc}
1 & -0.6217 & -0.00224 &  0.3698 &  0.01567 \\
  &  1 	    &  0.00414 & -0.5756 & -0.0243 \\
  &         &  1       &  0.0035 &  0.00978 \\
  &         &          &  1 	 &  0.0471 \\
  &         &          &         &  1    
\end{array} \right\}$ \\
\hline
\end{tabular}
}
\end{table}

\begin{table}
\renewcommand{\arraystretch}{1.3}
\renewcommand{\arraycolsep}{0.02in}
\caption{\label{tab:observables2}
\dkpi\ observables used for the global fit, from
Refs.~\cite{Aubert:2007wf,Aaltonen:2007uc,Zhang:2006dp}.}
\vspace*{6pt}
\footnotesize
\begin{center}
\begin{tabular}{l|ccc}
\hline
{\bf Mode} & \textbf{Observable} & {\bf Values} & {\bf Correlation coefficients} \\
\hline
\begin{tabular}{c}  
$D^0\ra K^+\pi^-$ \\
(\babar)
\end{tabular}
&
\begin{tabular}{c}
$R^{}_D$ \\
$x'^{2+}$ \\
$y'^+$ 
\end{tabular} & 
\begin{tabular}{c}
 $(0.303\pm 0.0189)\%$ \\
 $(-0.024\pm 0.052)\%$ \\
 $(0.98\pm 0.78)\%$ 
\end{tabular} &
$\left\{ \begin{array}{ccc}
 1 &  0.77 &  -0.87 \\
0.77 & 1 & -0.94 \\
-0.87 & -0.94 & 1 
\end{array} \right\}$ \\ \\
\begin{tabular}{c}  
$\dbar\ra K^-\pi^+$ \\
(\babar)
\end{tabular}
&
\begin{tabular}{c}
$A^{}_D$ \\
$x'^{2-}$ \\
$y'^-$ 
\end{tabular} & 
\begin{tabular}{c}
 $(-2.1\pm 5.4)\%$ \\
 $(-0.020\pm 0.050)\%$ \\
 $(0.96\pm 0.75)\%$ 
\end{tabular} & same as above \\
\hline
\begin{tabular}{c}  
$D^0\ra K^+\pi^-$ \\
(Belle)
\end{tabular}
&
\begin{tabular}{c}
$R^{}_D$ \\
$x'^{2+}$ \\
$y'^+$ 
\end{tabular} & 
\begin{tabular}{c}
 $(0.364\pm 0.018)\%$ \\
 $(0.032\pm 0.037)\%$ \\
 $(-0.12\pm 0.58)\%$ 
\end{tabular} &
$\left\{ \begin{array}{ccc}
 1 &  0.655 &  -0.834 \\
0.655 & 1 & -0.909 \\
-0.834 & -0.909 & 1 
\end{array} \right\}$ \\ \\
\begin{tabular}{c}  
$\dbar\ra K^-\pi^+$ \\
(Belle)
\end{tabular}
&
\begin{tabular}{c}
$A^{}_D$ \\
$x'^{2-}$ \\
$y'^-$ 
\end{tabular} & 
\begin{tabular}{c}
 $(2.3\pm 4.7)\%$ \\
 $(0.006\pm 0.034)\%$ \\
 $(0.20\pm 0.54)\%$ 
\end{tabular} & same as above \\
\hline
\begin{tabular}{c}  
$D^0\ra K^+\pi^-$ \\
\ \ \ \ \ + c.c. \\
(CDF)
\end{tabular}
&
\begin{tabular}{c}
$R^{}_D$ \\
$x'^{2}$ \\
$y'$ 
\end{tabular} & 
\begin{tabular}{c}
 $(0.304\pm 0.055)\%$ \\
 $(-0.012\pm 0.035)\%$ \\
 $(0.85\pm 0.76)\%$ 
\end{tabular} & 
$\left\{ \begin{array}{ccc}
 1 &  0.923 &  -0.971 \\
0.923 & 1 & -0.984 \\
-0.971 & -0.984 & 1 
\end{array} \right\}$ \\ 
\hline
\end{tabular}
\end{center}
\end{table}


\begin{table}
\renewcommand{\arraystretch}{1.3}
\renewcommand{\arraycolsep}{0.02in}
\caption{\label{tab:observables3}
Measurements of direct \cpv, from
Refs.~\cite{Aubert:2007if,Staric:2008rx,Aaij:2011in,Aaltonen:2011se,
cdf_public_note_10784}.
The parameter $A^{}_{CP}(f)$ is defined as
$[\Gamma(D^0\ra f)-\Gamma(\dbar\ra f)]/
[\Gamma(D^0\ra f)+\Gamma(\dbar\ra f)]$.}
\vspace*{6pt}
\footnotesize
\begin{center}
\begin{tabular}{l|ccc}
\hline
{\bf Mode} & \textbf{Observable} & {\bf Values} & 
                  {\boldmath $\Delta\langle t\rangle/\tau$} \\
\hline
\begin{tabular}{c}
$D^0\ra K^+K^-/\pi^+\pi^-$ \\
(\babar)
\end{tabular} & 
\begin{tabular}{c}
$A^{}_{CP}(K^+K^-)$ \\
$A^{}_{CP}(\pi^+\pi^-)$ 
\end{tabular} & 
\begin{tabular}{c}
$(0.00 \pm 0.34 \pm 0.13)\%$ \\
$(-0.24 \pm 0.52 \pm 0.22)\%$ 
\end{tabular} &
0 \\
\hline
\begin{tabular}{c}
$D^0\ra K^+K^-/\pi^+\pi^-$ \\
(Belle)
\end{tabular} & 
\begin{tabular}{c}
$A^{}_{CP}(K^+K^-)$ \\
$A^{}_{CP}(\pi^+\pi^-)$ 
\end{tabular} & 
\begin{tabular}{c}
$(-0.43 \pm 0.30 \pm 0.11)\%$ \\
$(0.43 \pm 0.52 \pm 0.12)\%$ 
\end{tabular} &
0 \\
\hline
\begin{tabular}{c}
$D^0\ra K^+K^-/\pi^+\pi^-$ \\
(LHCb 37~pb$^{-1}$)
\end{tabular} & 
$A^{}_{CP}(K^+K^-)-A^{}_{CP}(\pi^+\pi^-)$  &
$(-0.82 \pm 0.21 \pm 0.11)\%$ &  
$0.0983 \pm 0.00291$ \\
\hline
\begin{tabular}{c}
$D^0\ra K^+K^-/\pi^+\pi^-$ \\
(CDF 9.7~fb$^{-1}$ prelim.)
\end{tabular} & 
$A^{}_{CP}(K^+K^-)-A^{}_{CP}(\pi^+\pi^-)$  &
$(-0.62 \pm 0.21 \pm 0.10)\%$ & 
$0.26 \pm 0.01$ \\ 
\hline
(CDF 5.9~fb$^{-1}$ not used)  & 
\begin{tabular}{c}
$A^{}_{CP}(K^+K^-)$ \\
$A^{}_{CP}(\pi^+\pi^-)$ 
\end{tabular} & 
\begin{tabular}{c}
$(-0.24 \pm 0.22 \pm 0.09)\%$ \\
$(0.22 \pm 0.24 \pm 0.11)\%$ 
\end{tabular} &
\begin{tabular}{c}
$2.65 \pm 0.03$ \\
$2.40 \pm 0.03$ 
\end{tabular}
\\
\hline
\end{tabular}
\end{center}
\end{table}


\begin{table}
\renewcommand{\arraycolsep}{0.02in}
\renewcommand{\arraystretch}{1.3}
\begin{center}
\caption{\label{tab:relationships}
Left: decay modes used to determine fitted parameters 
$x,\,y,\,\delta,\,\delta^{}_{K\pi\pi},\,R^{}_D,\,A^{}_D,\,|q/p|$, and $\phi$.
Middle: the observables measured for each decay mode. Right: the 
relationships between the observables measured and the fitted parameters.}
\vspace*{6pt}
\footnotesize
\resizebox{0.99\textwidth}{!}{
\begin{tabular}{l|c|l}
\hline
\textbf{Decay Mode} & \textbf{Observables} & \textbf{Relationship} \\
\hline
$D^0\ra K^+K^-/\pi^+\pi^-$  & 
\begin{tabular}{c}
 $y^{}_{CP}$  \\
 $A^{}_{\Gamma}$
\end{tabular} & 
$\begin{array}{c}
2y^{}_{CP} = 
\left(\left|q/p\right|+\left|p/q\right|\right)y\cos\phi - \\
\hskip0.50in \left(\left|q/p\right|-\left|p/q\right|\right)x\sin\phi \\
2A^{}_\Gamma = 
\left(\left|q/p\right|-\left|p/q\right|\right)y\cos\phi - \\
\hskip0.50in \left(\left|q/p\right|+\left|p/q\right|\right)x\sin\phi
\end{array}$   \\
\hline
$D^0\ra K^0_S\,\pi^+\pi^-$ & 
$\begin{array}{c}
x \\ 
y \\ 
|q/p| \\ 
\phi
\end{array}$ &   \\ 
\hline
$D^0\ra K^+\ell^-\nu$ & $R^{}_M$  & $R^{}_M = (x^2 + y^2)/2$ \\
\hline
\begin{tabular}{l}
$D^0\ra K^+\pi^-\pi^0$ \\
(Dalitz plot analysis)
\end{tabular} & 
$\begin{array}{c}
x'' \\ 
y''
\end{array}$ &
$\begin{array}{l}
x'' = x\cos\delta^{}_{K\pi\pi} + y\sin\delta^{}_{K\pi\pi} \\ 
y'' = y\cos\delta^{}_{K\pi\pi} - x\sin\delta^{}_{K\pi\pi}
\end{array}$ \\
\hline
\begin{tabular}{l}
``Double-tagged'' \\
branching fractions \\
measured in \\
$\psi(3770)\ra DD$ decays
\end{tabular} & 
$\begin{array}{c}
R^{}_M \\
y \\
R^{}_D \\
\sqrt{R^{}_D}\cos\delta
\end{array}$ &   $R^{}_M = (x^2 + y^2)/2$ \\
\hline
$D^0\ra K^+\pi^-$ &
$\begin{array}{c}
%R^+_D,\ R^-_D \\
x'^2,\ y' \\
x'^{2+},\ x'^{2-} \\
y'^+,\ y'^-
\end{array}$ & 
$\begin{array}{l}
%R^{}_D = (R^+_D + R^-_D)/2 \\
%A^{}_D = (R^+_D - R^-_D)/(R^+_D + R^-_D)  \\ \\
%R^\pm_M=(x'^{\pm 2}+y'^{\pm 2})/2 \\
%(|q/p|^4-1)/(|q/p|^4+1)=(R^+_M-R^-_M)/(R^+_M+R^-_M)\equiv A^{}_M \\ \\
x' = x\cos\delta + y\sin\delta \\ 
y' = y\cos\delta - x\sin\delta \\
A^{}_M\equiv (|q/p|^4-1)/(|q/p|^4+1) \\
x'^\pm = [(1\pm A^{}_M)/(1\mp A^{}_M)]^{1/4} \times \\
\hskip0.50in (x'\cos\phi\pm y'\sin\phi) \\
y'^\pm = [(1\pm A^{}_M)/(1\mp A^{}_M)]^{1/4} \times \\
\hskip0.50in (y'\cos\phi\mp x'\sin\phi) \\
%x'^\pm = |q/p|^{\pm 1}(x'\cos\phi\pm y'\sin\phi) \\
%y'^\pm = |q/p|^{\pm 1}(y'\cos\phi\mp x'\sin\phi) \\
\end{array}$ \\
\hline
\begin{tabular}{c}
$D^0\ra K^+\pi^-/K^-\pi^+$ \\
(time-integrated)
\end{tabular} & 
\begin{tabular}{c}
$\frac{\displaystyle \Gamma(D^0\ra K^+\pi^-)+\Gamma(\dbar\ra K^-\pi^+)}
{\displaystyle \Gamma(D^0\ra K^-\pi^+)+\Gamma(\dbar\ra K^+\pi^-)}$  \\ \\
$\frac{\displaystyle \Gamma(D^0\ra K^+\pi^-)-\Gamma(\dbar\ra K^-\pi^+)}
{\displaystyle \Gamma(D^0\ra K^+\pi^-)+\Gamma(\dbar\ra K^-\pi^+)}$ 
\end{tabular} & 
\begin{tabular}{c}
$R^{}_D$ \\ \\ \\
$A^{}_D$ 
\end{tabular} \\
\hline
\begin{tabular}{c}
$D^0\ra K^+K^-/\pi^+\pi^-$ \\
(time-integrated)
\end{tabular} & 
\begin{tabular}{c}
$\frac{\displaystyle \Gamma(D^0\ra K^+K^-)-\Gamma(\dbar\ra K^+K^-)}
{\displaystyle \Gamma(D^0\ra K^+K^-)+\Gamma(\dbar\ra K^+K^-)}$    \\ \\
$\frac{\displaystyle \Gamma(D^0\ra\pi^+\pi^-)-\Gamma(\dbar\ra\pi^+\pi^-)}
{\displaystyle \Gamma(D^0\ra\pi^+\pi^-)+\Gamma(\dbar\ra\pi^+\pi^-)}$ 
\end{tabular} & 
\begin{tabular}{c}
$A^{}_K  + \frac{\displaystyle \langle t\rangle}
{\displaystyle \tau^{}_D}\,{\cal A}_{CP}^{\rm indirect}$ 
\ \ (${\cal A}_{CP}^{\rm indirect}\approx -A^{}_\Gamma$)
\\ \\ \\
$A^{}_\pi + \frac{\displaystyle \langle t\rangle}
{\displaystyle \tau^{}_D}\,{\cal A}_{CP}^{\rm indirect}$ 
\ \ (${\cal A}_{CP}^{\rm indirect}\approx -A^{}_\Gamma$)
\end{tabular} \\
\hline
%2{\cal A}_{CP}^{\rm indirect} & = & 
%\Big(\left|q/p\right| + \left|p/q\right|\Big) x \sin\phi\ -\ 
%\Big(\left|q/p\right| - \left|p/q\right|\Big) y \cos\phi \\
\end{tabular}
}
\end{center}
\end{table}


\subsubsection{Fit results}

The global fit uses MINUIT with the MIGRAD minimizer, 
and all errors are obtained from MINOS~\cite{MINUIT:webpage}. 
Four separate fits are performed: 
{\it (a)}\ assuming \cp\ conservation, i.e., fixing
$A^{}_D\!=\!0$, $A_K\!=\!0$, $A^{}_\pi\!=\!0$, $\phi\!=\!0$, 
and $|q/p|\!=\!1$;
{\it (b)}\ assuming no direct \cpv\ and fitting for
parameters $x$, $y$, and $\phi$; 
{\it (c)}\ assuming no direct \cpv\ and fitting for
parameters $x^{}_{12}= 2|M^{}_{12}|/\Gamma$, 
$y^{}_{12}= \Gamma^{}_{12}/\Gamma$, and 
$\phi^{}_{12}= {\rm Arg}(M^{}_{12}/\Gamma^{}_{12})$,
where $M^{}_{12}$ and $\Gamma^{}_{12}$ are the off-diagonal
elements of the $D^0$-$\dbar$ mass and decay matrices, respectively; and
{\it (d)}\ allowing full \cpv, i.e., floating all parameters. 

For the no-direct-\cpv\ fits, we set direct-\cpv\ parameters 
$A^{}_D\!=\!0$, $A_K\!=\!0$, and $A^{}_\pi\!=\!0$. In addition, for the 
first fit {\it (b)\/} we impose the relation~\cite{Ciuchini:2007cw,Kagan:2009gb}
$\tan\phi = (1-|q/p|^2)/(1+|q/p|^2)\times (x/y)$; this reduces 
four independent parameters to 
three.\footnote{One can also use Eq.~(15) of Ref.~\cite{Grossman:2009mn}
to reduce four parameters to three.} 
We impose this relationship in two ways:
first we float parameters $x$, $y$, and $\phi$ and from them derive $|q/p|$;
then we repeat the fit floating $x$, $y$, and $|q/p|$ and from them derive 
$\phi$. The central values returned by the two fits are identical, but the 
first fit yields MINOS errors for $\phi$, while the second fit 
yields MINOS errors for $|q/p|$. For no-direct-\cpv\ fit 
{\it (c)}, we fit for the underlying parameters $x^{}_{12}$, $y^{}_{12}$, 
and $\phi^{}_{12}$, from which parameters $x$, $y$, $|q/p|$, and $\phi$ 
are derived. 

All fit results are listed in 
Table~\ref{tab:results}. For the \cpv-allowed fit,
individual contributions to the $\chi^2$ are listed 
in Table~\ref{tab:results_chi2}. The total $\chi^2$ 
is 35.6 for $37-10=27$ degrees of freedom; this 
corresponds to a confidence level of~0.124, which 
is satisfactory.


Confidence contours in the two dimensions $(x,y)$ or 
in $(|q/p|,\phi)$ are obtained by letting, for any point in the
two-dimensional plane, all other fitted parameters take their 
preferred values. The resulting $1\sigma$-$5\sigma$ contours 
are shown 
in Fig.~\ref{fig:contours_ncpv} for the \cp-conserving case, 
in Fig.~\ref{fig:contours_ndcpv} for the no-direct-\cpv\ case, 
and in Fig.~\ref{fig:contours_cpv} for the \cpv-allowed 
case. The contours are determined from the increase of the
$\chi^2$ above the minimum value.
One observes that the $(x,y)$ contours for the no-\cpv\ fit 
are very similar to those for the \cpv-allowed fit. 
In the latter fit, the
$\chi^2$ at the no-mixing point $(x,y)\!=\!(0,0)$ is 110 units above 
the minimum value; for two degrees of freedom this has a confidence 
level corresponding to $10.2\sigma$. Thus, no mixing is excluded 
at this high level. In the $(|q/p|,\phi)$ plot, the point $(1,0)$ 
is within the $1\sigma$ contour; thus the data is consistent 
with \cp\ conservation.

One-dimensional confidence curves for individual parameters 
are obtained by letting, for any value of the parameter, all other 
fitted parameters take their preferred values. The resulting
functions $\Delta\chi^2=\chi^2-\chi^2_{\rm min}$ ($\chi^2_{\rm min}$
is the minimum value) are shown in Fig.~\ref{fig:1dlikelihood}.
The points where $\Delta\chi^2=3.84$ determine 95\% C.L. intervals 
for the parameters; these intervals are listed in Table~\ref{tab:results}.


\begin{figure}
\begin{center}
%\includegraphics[width=4.2in]{figures/charm/fig_plot_xyn2d.eps}
\end{center}
\vskip-0.20in
\caption{\label{fig:contours_ncpv}
Two-dimensional contours for mixing parameters $(x,y)$, for no \cpv. }
\end{figure}


\begin{figure}
\begin{center}
\vbox{
%\includegraphics[width=84mm]{figures/charm/fig_plot_xy122d.eps}
%\vskip0.10in
%\includegraphics[width=84mm]{figures/charm/fig_plot_xp122d.eps}
\vskip0.30in
%\includegraphics[width=84mm]{figures/charm/fig_plot_yp122d.eps}
}
\end{center}
\vskip-0.10in
\caption{\label{fig:contours_ndcpv}
Two-dimensional contours for theoretical parameters 
$(x^{}_{12},y^{}_{12})$ (top left), 
$(x^{}_{12},\phi^{}_{12})$ (top right), and 
$(y^{}_{12},\phi^{}_{12})$ (bottom), 
for no direct \cpv.}
\end{figure}


\begin{figure}
\begin{center}
\vbox{
%\includegraphics[width=4.2in]{figures/charm/fig_plot_xy2d.eps}
\vskip0.10in
%\includegraphics[width=4.2in]{figures/charm/fig_plot_qp2d.eps}
}
\end{center}
\vskip-0.10in
\caption{\label{fig:contours_cpv}
Two-dimensional contours for parameters $(x,y)$ (top) 
and $(|q/p|,\phi)$ (bottom), allowing for \cpv.}
\end{figure}


\begin{figure}
\begin{center}
\hbox{\hskip0.50in
%\includegraphics[width=72mm]{figures/charm/fig_plot_x1d.eps}
\hskip0.20in
%\includegraphics[width=72mm]{figures/charm/fig_plot_y1d.eps}}
\hbox{\hskip0.50in
%\includegraphics[width=72mm]{figures/charm/fig_plot_d1d.eps}
\hskip0.20in
%\includegraphics[width=72mm]{figures/charm/fig_plot_d21d.eps}}
\hbox{\hskip0.50in
%\includegraphics[width=72mm]{figures/charm/fig_plot_q1d.eps}
\hskip0.20in
%\includegraphics[width=72mm]{figures/charm/fig_plot_p1d.eps}}
\end{center}
\vskip-0.30in
\caption{\label{fig:1dlikelihood}
The function $\Delta\chi^2=\chi^2-\chi^2_{\rm min}$ 
for fitted parameters
$x,\,y,\,\delta,\,\delta^{}_{K\pi\pi},\,|q/p|$, and $\phi$.
The points where $\Delta\chi^2=3.84$ (denoted by dashed 
horizontal lines) determine 95\% C.L. intervals. }
\end{figure}


\begin{table}
\renewcommand{\arraystretch}{1.4}
\begin{center}
\caption{\label{tab:results}
Results of the global fit for different assumptions concerning~\cpv.}
\vspace*{6pt}
\footnotesize
\begin{tabular}{c|cccc}
\hline
\textbf{Parameter} & \textbf{\boldmath No \cpv} & \textbf{\boldmath No direct \cpv} 
& \textbf{\boldmath \cpv-allowed} & \textbf{\boldmath \cpv-allowed 95\% C.L.}  \\
\hline
$\begin{array}{c}
x\ (\%) \\ 
y\ (\%) \\ 
\delta\ (^\circ) \\ 
R^{}_D\ (\%) \\ 
A^{}_D\ (\%) \\ 
|q/p| \\ 
\phi\ (^\circ) \\
\delta^{}_{K\pi\pi}\ (^\circ)  \\
A^{}_{\pi} \\
A^{}_K \\
x^{}_{12}\ (\%) \\ 
y^{}_{12}\ (\%) \\ 
\phi^{}_{12} (^\circ)
\end{array}$ & 
$\begin{array}{c}
0.65\,^{+0.18}_{-0.19} \\
0.73\,\pm 0.12 \\
21.0\,^{+9.8}_{-11.0} \\
0.3307\,\pm 0.0080 \\
- \\
- \\
- \\
17.8\,^{+21.7}_{-22.8} \\
- \\
- \\
- \\
- \\
- 
\end{array}$ &
$\begin{array}{c}
0.62\,\pm 0.19 \\
0.75\,\pm 0.12 \\
22.2\,^{+9.9}_{-11.2} \\
0.3305\,\pm 0.0080 \\
- \\
1.04\,^{+0.07}_{-0.06} \\ 
-2.02\,^{+2.67}_{-2.74} \\ 
19.4\,^{+21.8}_{-22.9} \\
- \\
- \\
0.62\,\pm 0.19 \\
0.75\,\pm 0.12 \\
4.9\,^{+7.7}_{-6.5} 
\end{array}$ &
$\begin{array}{c}
0.63\,^{+0.19}_{-0.20}  \\
0.75\,\pm 0.12 \\
22.1\,^{+9.7}_{-11.1} \\
0.3311\,\pm 0.0081 \\
-1.7\,\pm 2.4 \\
0.88\,^{+0.18}_{-0.16} \\
-10.1\,^{+9.5}_{-8.9} \\ 
19.3\,^{+21.8}_{-22.9} \\
0.36\,\pm 0.25 \\
-0.31\,\pm 0.24 \\
- \\
- \\
- 
\end{array}$ &
$\begin{array}{c}
\left[0.24 ,\, 0.99\right] \\
\left[0.51 ,\, 0.98\right] \\
\left[-2.6 ,\, 40.6\right] \\
\left[0.315 ,\, 0.347\right] \\
\left[-6.4 ,\, 3.0\right] \\
\left[0.59 ,\, 1.26\right] \\
\left[-27.4 ,\, 8.7\right] \\
\left[-26.3 ,\, 61.8\right] \\ 
\left[-0.13 ,\, 0.86\right] \\
\left[-0.78 ,\, 0.15\right] \\
\left[0.25 ,\, 0.99\right] \\
\left[0.51 ,\, 0.98\right] \\
\left[-8.4 ,\, 24.6\right] \\
\end{array}$ \\
\hline
\end{tabular}
\end{center}
\end{table}


\begin{table}
\renewcommand{\arraystretch}{1.4}
\begin{center}
\caption{\label{tab:results_chi2}
Individual contributions to the $\chi^2$ for the \cpv-allowed fit.}
\vspace*{6pt}
\footnotesize
\begin{tabular}{l|rr}
\hline
\textbf{Observable} & \textbf{\boldmath $\chi^2$} & \textbf{\boldmath $\sum\chi^2$} \\
\hline
$y^{}_{CP}$                      & 2.61 & 2.61 \\
$A^{}_\Gamma$                    & 0.00 & 2.61 \\
\hline
$x^{}_{K^0\pi^+\pi^-}$ Belle       & 0.28 & 2.88 \\
$y^{}_{K^0\pi^+\pi^-}$ Belle       & 1.65 & 4.54 \\
$|q/p|^{}_{K^0\pi^+\pi^-}$ Belle   & 0.01 & 4.54 \\
$\phi^{}_{K^0\pi^+\pi^-}$  Belle   & 0.51 & 5.05 \\
\hline
$x^{}_{K^0 h^+ h^-}$ \babar         & 2.97 & 8.02 \\
$y^{}_{K^0 h^+ h^-}$ \babar         & 0.37 & 8.38 \\
\hline
$R^{}_M(K^+\ell^-\nu)$           & 0.09 & 8.48 \\
\hline
$x^{}_{K^+\pi^-\pi^0}$ \babar       & 5.71 & 14.19 \\
$y^{}_{K^+\pi^-\pi^0}$ \babar       & 2.22 & 16.40 \\
\hline
CLEOc                           &      &       \\
($x/y/R^{}_D/\sqrt{R^{}_D}\cos\delta/\sqrt{R^{}_D}\sin\delta$) 
                                & 7.28 & 23.68 \\
\hline
$R^+_D/x'{}^{2+}/y'{}^+$ \babar   & 2.34 & 26.02    \\
$R^-_D/x'{}^{2-}/y'{}^-$ \babar   & 1.30 & 27.31    \\
$R^+_D/x'{}^{2+}/y'{}^+$ Belle   & 4.12 & 31.44    \\
$R^-_D/x'{}^{2-}/y'{}^-$ Belle   & 1.35 & 32.79    \\
$R^{}_D/x'{}^{2}/y'$ CDF         & 0.39 & 33.17    \\
\hline
$A^{}_{KK}/A^{}_{\pi\pi}$  \babar  & 1.89 & 35.06  \\
$A^{}_{KK}/A^{}_{\pi\pi}$  Belle  & 0.12 & 35.18  \\
$A^{}_{KK}/A^{}_{\pi\pi}$  CDF    & 0.06 & 35.25  \\
$A^{}_{KK}-A^{}_{\pi\pi}$  LHCb   & 0.37 & 35.62  \\
\hline
\end{tabular}
\end{center}
\end{table}

%\newpage

\subsubsection{Conclusions}

From the fit results listed in Table~\ref{tab:results}
and shown in Figs.~\ref{fig:contours_cpv} and \ref{fig:1dlikelihood},
we conclude the following:
\begin{itemize}
\item the experimental data consistently indicate that 
$D^0$ mesons undergo mixing. The no-mixing point $x=y=0$
is excluded at $10.2\sigma$. The parameter $x$ differs from
zero by $2.7\sigma$, and $y$ differs from zero by
$6.0\sigma$. This mixing is presumably dominated 
by long-distance processes, which are difficult to calculate.
Unless it turns out that $|x|\gg |y|$~\cite{Bigi:2000wn},
which is not indicated, it will probably be difficult to 
identify new physics from $(x,y)$ alone.
\item Since \ycp\ is positive, the \cp-even state is shorter-lived
as in the $K^0$-$\kbar$ system. However, since $x$ also appears
to be positive, the \cp-even state is heavier, 
unlike in the $K^0$-$\kbar$ system.
\item The LHCb and CDF experiments have obtained first evidence
for {\it direct\/} \cpv\ in $D^0$ decays. Higher statistics 
measurements should be able to clarify this effect. There is 
no evidence for \cpv\ arising from $D^0$-$\dbar$ mixing 
($|q/p|\neq 1$) or from a phase difference between the 
mixing amplitude and a direct decay amplitude ($\phi\neq 0$). 
\end{itemize}


