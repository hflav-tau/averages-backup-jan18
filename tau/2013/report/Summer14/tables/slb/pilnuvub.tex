
\begin{table}[hbtf]
\caption{\label{tab:pilnuvub}
Determinations of \vub\ based on the average partial
$\Bb\to\pi\ell\nub$ decay branching fractions stated in
Table~\ref{tab:pilnubf}. 
The $q^2$ ranges for the partial branching fractions corresponding to the 
validity ranges of the form factor calculations are indicated. 
The first uncertainty is experimental and the second is from theory.  
}
%\scriptsize
%\vspace{5mm} 
\begin{center}
\renewcommand{\arraystretch}{1.2}
\begin{tabular}{|lcc|}
\hline
Method                                         & $q^2$ range [$\gev^2/c^2$] & $\Vub [10^{-3}]$ \\\hline\hline
Khodjamirian et al. (LCSR) ~\cite{Khodjamirian:2011ub} & 0 -- 12                    & $3.41\pm 0.06 {}^{+0.37}_{-0.32}$ \\ \hline
Ball \& Zwicky (LCSR)~\cite{Ball:2004ye}              & 0 -- 16                    & $3.58\pm 0.06 {}^{+0.59}_{-0.40}$ \\ \hline
HPQCD (LQCD)~\cite{Dalgic:2006dt}                     & 16 -- 26.4                 & $3.52\pm 0.08 {}^{+0.61}_{-0.40}$ \\  \hline
FNAL/MILC (LQCD)~\cite{Bailey:2008wp}                 & 16 -- 26.4                 & $3.36\pm 0.08 {}^{+0.37}_{-0.31}$ \\ 
%APE, $q^2>16\,\gev^2/c^2$~\cite{Abada:2000ty}         & $3.72\pm 0.21 {}^{+1.43}_{-0.66}$ \\ 
\hline
\end{tabular}
\end{center}
\end{table}
