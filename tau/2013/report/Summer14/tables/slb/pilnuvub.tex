
\begin{table}[hbtf]
\caption{\label{tab:pilnuvub}
Determinations of \vub\ based on the average partial
$\Bb\to\pi\ell\nub$ decay branching fractions stated in
Table~\ref{tab:pilnubf}. The first
uncertainty is experimental and the second is from theory.  The
full or partial branching fractions are used as indicated. 
Acronyms for the calculations refer to either the method (LCSR) or
the collaboration working on it (HPQCD, FNAL/MILC).
}
%\scriptsize
%\vspace{5mm} 
\begin{center}
\begin{tabular}{|lc|}
\hline
Method                                                     & $\Vub [10^{-3}]$ \\\hline\hline
LCSR~1,    $q^2<12\,\gev^2/c^2$~\cite{Khodjamirian:2011ub} & $3.40\pm 0.07 {}^{+0.37}_{-0.32}$ \\ \hline
LCSR~2,    $q^2<16\,\gev^2/c^2$~\cite{Ball:2004ye}         & $3.57\pm 0.06 {}^{+0.59}_{-0.39}$ \\ \hline
HPQCD,     $q^2>16\,\gev^2/c^2$~\cite{Dalgic:2006dt}       & $3.45\pm 0.09 {}^{+0.60}_{-0.39}$ \\  \hline
FNAL/MILC, $q^2>16\,\gev^2/c^2$~\cite{Bailey:2008wp}       & $3.30\pm 0.09 {}^{+0.37}_{-0.30}$ \\  \hline
%APE, $q^2>16\,\gev^2/c^2$~\cite{Abada:2000ty}         & $3.72\pm 0.21 {}^{+1.43}_{-0.66}$ \\ 
\hline
\end{tabular}
\end{center}
\end{table}
