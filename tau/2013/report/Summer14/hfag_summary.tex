\clearpage

\section{Summary}
\label{sec:summary}

This article provides updated world averages for 
$b$-hadron properties using results available through summer 2014. 
A small selection of highlights of the results described in Sections
\ref{sec:life_mix}-\ref{sec:tau} is given in 
Tables~\ref{tab_summary1},~\ref{tab_summary2} and~\ref{tab_summary3}.

\begin{table}
\caption{Selected world averages 
from Chapters~\ref{sec:life_mix} and~\ref{sec:cp_uta}.}
\label{tab_summary1}
\renewcommand{\arraystretch}{1.15}
\begin{center}
\begin{tabular}{|l|c|}
\hline
 {\bf\boldmath \b-hadron lifetimes} &   \\
 ~~$\tau(\Bd)$  & \hfagTAUBD \\
 ~~$\tau(\Bu)$  & \hfagTAUBU \\
% ~~$\tau(\Bs\to~\mbox{flavour specific})$  & \hfagTAUBSSL \\
 ~~$\bar{\tau}(\Bs) = 1/\Gs$  & \hfagTAUBSMEANC \\
 ~~$\tau(\Bc)$  & \hfagTAUBC \\
 ~~$\tau(\Lb)$  & \hfagTAULB \\
 % ~~$\tau(\Xib)$ (mean) & \hfagTAUXB \\
 % ~~$\tau(\Omegab)$ & \hfagTAUOB \\
\hline
 {\bf\boldmath \b-hadron fractions} &   \\
 ~~$f^{+-}/f^{00}$ in \Ups decays  & \hfagFF \\ 
 ~~\fBs in \Upsfive decays & \hfagFSFIVE \\
 ~~\fBs, \fbb in $Z$ decays & \hfagZFBS, \hfagZFBB \\
 ~~\fBs, \fbb at Tevatron & \hfagTFBS, \hfagTFBB \\
\hline
 {\bf\boldmath \Bd\ and \Bs\ mixing / \CP violation parameters} &   \\
 ~~\dmd &  \hfagDMDWU \\
 ~~$|q/p|_{\particle{d}}$ & \hfagQPDB  \\
 ~~\dms  &  \hfagDMS \\
 %~~$\DGGs = (\Gamma_{\rm L} - \Gamma_{\rm H})/\Gs$ & \hfagDGSGSCON \\
 ~~$\DGs = \Gamma_{\rm L} - \Gamma_{\rm H}$ & \hfagDGSCON \\
 ~~$|q/p|_{\particle{s}}$ & \hfagQPS   \\
 ~~\phiccbars  & \hfagPHISCOMB \\
\hline
{\bf Measurements related to Unitarity Triangle angles} & \\
 ~~ $\stwob \equiv \sin\! 2\phi_1$ & $0.682 \pm 0.019$ \\
 ~~ $\beta \equiv \phi_1$          & $\left( 21.5 \,^{+0.8}_{-0.7} \right)^\circ$ \\
 ~~ $-\etacp S_{\phi \KS}$       & $0.74\,^{+0.11}_{-0.13}$ \\
 ~~ $-\etacp S_{\etapr \Kz}$       & $0.63 \pm 0.06$ \\
 ~~ $-\etacp S_{\KS \KS \KS}$       & $0.72 \pm 0.19$ \\
 ~~ $-\etacp S_{\Kp \Km \KS}$       & $0.68\,^{+0.09}_{-0.10}$ \\
 ~~ $-\etacp S_{\jpsi \piz}$       & $0.93 \pm 0.15$ \\
 ~~ $S_{K^* \gamma}$       & $-0.16 \pm 0.22$ \\
 ~~ $S_{\pi^+\pi^-}$               & $-0.66 \pm 0.06$ \\  
 ~~ $C_{\pi^+\pi^-}$               & $-0.31 \pm 0.05$ \\  
 ~~ $S_{\rho^+\rho^-}$       & $-0.05 \pm 0.17$ \\
 ~~ $a(D^{*\pm}\pi^{\mp})$       & $-0.039 \pm 0.010$ \\
 ~~ $A^{}_{CP}(B\ra D^{}_{CP+}K)$       & $0.19 \pm 0.03$ \\
 ~~ $A_{\rm ADS}(B\ra D^{}_{K\pi}K)$       & $-0.54 \pm 0.12$ \\
 ~~ $R_{\rm ADS}(B\ra D^{}_{K\pi}K)$       & $0.0153 \pm 0.0017$ \\
\hline
\end{tabular}
\end{center}
\end{table}
\begin{table}
\caption{Selected world averages at the end of 2011
from Chapters~\ref{sec:slbdecays}--\ref{sec:rare}.}
\label{tab_summary2}
\renewcommand{\arraystretch}{1.15}
\begin{center}
\begin{tabular}{|l|c|}
\hline
{\bf\boldmath Semileptonic \B decay parameters} & \\
 ~~${\cal B}(\Bzb\to D^{*+}\ell^-\nub)$ & $(4.95\pm 0.11)\%$\\
 ~~${\cal B}(\B^-\to D^{*0}\ell^-\nub)$ & $(5.70\pm 0.19)\%$\\
 ~~${\cal F}(1)\vcb$ & $(35.90\pm 0.45)\times 10^{-3}$\\
 ~~$\vcb$ from $\bar B\to D^*\ell^-\bar\nu_\ell$ & $(39.54\pm
 0.50_{\rm exp}\pm 0.74_{\rm th})\times 10^{-3}$\\
\hline
 ~~${\cal B}(\Bzb\to D^+\ell^-\nub)$ & $(2.18\pm 0.12)\%$\\
 ~~${\cal B}(\B^-\to D^0\ell^-\nub)$ & $(2.26\pm 0.11)\%$\\
 ~~${\cal G}(1)\vcb$ & $(42.64 \pm 1.53)\times 10^{-3}$\\
 ~~$\vcb$ from $\bar B\to D\ell^-\bar\nu_\ell$ & $(39.70\pm 1.42_{\rm
 exp}\pm 0.89_{\rm th})\times 10^{-3}$\\
\hline
 ~~${\cal B}(\bar B\to X_c\ell^-\bar\nu_\ell)$ & $(10.51\pm 0.13)\%$\\
 ~~${\cal B}(\bar B\to X\ell^-\bar\nu_\ell)$ & $(10.72\pm 0.13)\%$\\
 ~~$\vcb$ from $\bar B\to X\ell^-\bar\nu_\ell$ & $(41.88\pm
 0.73)\times 10^{-3}$\\
\hline
 ~~${\cal B}(\Bb\to\pi\ell^-\nub)$ & $(1.42\pm 0.05)\times 10^{-4}$\\
 ~~$\vub$ from $\Bb\to\pi\ell^-\nub$ & $(3.23\pm 0.30)\times
 10^{-3}$\\
 ~~$\vub$ from $\Bb\to X_u\ell^-\nub$ & $(4.40\pm 0.15_{\rm exp}\pm
 0.20_{\rm th})\times 10^{-3}$\\
\hline
%% {\bf\boldmath Rare \B decays} &   \\
%% ~~ ${\cal B}(B \to X_s \gamma)$ & $(3.55 \pm 0.24 \pm 0.09) \times 10^{-4}$ \\
%% ~~ ${\cal B}(\Bp \to \tau^+ \nu)$ & $(1.67 \pm 0.30) \times 10^{-4}$ \\
%% ~~ $A_{\rm FB}(\Bz \to K^{*0}\mu^+\mu^-)$ in bins of $q^2 = m^2(\mu^+\mu^-)$ & see Table~\ref{tab:Kstarmumu-Afb} \\
%% ~~ ${\cal B}(\Bs \to \mu^+\mu^-)$ & $<1.2 \times 10^{-8}$ (90\,\% C.L.) \\
%% ~~$A_{\CP}(\particle{\Bd\to K^+\pi^-})$ & $(-0.087 \pm 0.008)$\\
%% ~~$A_{\CP}(\particle{B^+\to K^+\pi^0})$ & $(0.037 \pm 0.021)$ \\
%% ~~$A_{\CP}(\particle{\Bs\to K^-\pi^+})$ & $(0.29 \pm 0.07)$ \\
\hline
\end{tabular}
\end{center}
\end{table}
\begin{table}
\caption{Selected world averages at the end of 2011
from Chapters~\ref{sec:charm_physics} and~\ref{sec:tau}.}
\label{tab_summary3}
\renewcommand{\arraystretch}{1.15}
\begin{center}
\begin{tabular}{|l|c|}
\hline
 {\bf\boldmath $D^0$ mixing and \CP violation parameters} &   \\
 ~~$x$ &  $(0.63\,^{+0.19}_{-0.20})\%$  \\
 ~~$y$ &  $(0.75\,\pm 0.12)\%$  \\
% ~~$R^{}_D$ &  $(0.3319\,\pm 0.0081)\%$  \\
 ~~$A^{}_D$ &  $(-1.7\,\pm 2.4)\%$  \\
 ~~$|q/p|$ & $0.88\,^{+0.18}_{-0.16}$  \\
 ~~$\phi$ &  $(-10.1\,^{+9.5}_{-8.9})^\circ$  \\
\hline
 ~~$x^{}_{12}$ (no direct \CP violation) &  $(0.62\,\pm 0.19)\%$  \\
 ~~$y^{}_{12}$ (no direct \CP violation) &  $(0.75\,\pm 0.12)\%$  \\
 ~~$\phi^{}_{12}$ (no direct \CP violation) &  $(4.9\,^{+7.7}_{-6.5})^\circ$  \\
\hline
~~$a^{\rm ind}_{CP}$ & $(-0.02 \pm 0.23)\%$ \\
~~$\Delta a^{\rm dir}_{CP}$ & $(-0.66 \pm 0.15)\%$ \\
\hline
%% {\bf\boldmath $\tau$ parameters, Lepton Universality, and $|V_{us}|$} &   \\
%% %% ~~ $m^{}_\tau$ (MeV/$c^2$)                   & $1776.77\, \pm 0.15$ \\
%%  ~~ $g^{}_\mu/g^{}_e$ & \quantgmubygeUtau \\
%%  ~~ $g^{}_{\tau}/g^{}_{\mu}$ & \quantgtaubygmuUtau \\
%%  ~~ $g^{}_{\tau}/g^{}_{e}$ & \quantgtaubygeUtau \\
%%  ~~ ${\cal B}_e^{\text{uni}}$ & $(\quantBeUuniv)\%$ \\
%%  ~~ $R_{\text{had}}$ & \quantRUtau \\
%%  ~~ $|V_{us}|$ from ${\cal{B}}(\tau^-\to K^-\nu^{}_\tau)$ &  \quantVusUtauKnu \\
%%  ~~ $|V_{us}|$ from ${\cal{B}}(\tau^- \to K^-\nu^{}_\tau)/ {\cal{B}}(\tau^- \to \pi^-\nu^{}_\tau)$ & \quantVusUtauKpi \\ 
%%  ~~ $|V_{us}|$ from inclusive sum of strange branching fractions & \quantVus \\
%%  ~~ $|V_{us}|$ tau average & \quantVusUtau \\
\hline
\end{tabular}
\end{center}
\end{table}

%%% lifetimes and mixing highlights
Concerning $b$-hadron lifetime and mixing averages,
the most significant changes in the past two years
are due to new results from the CDF, \dzero and LHCb experiments, 
mainly in the \Bs sector. While the Tevatron 
experiments have updated some of their analyses with the 
full Run II data sample, LHCb has just entered the game 
and is taking the lead already 
with results based on the 2010--2011 data samples collected 
at the LHC.
While the updated \dzero like-sign dimuon asymmetry 
still deviates from the Standard Model prediction
(with a significance increased to $3.9\,\sigma$), 
there is still no evidence of \CP violation in either 
\Bd or \Bs mixing, with precisions on the semileptonic asymmetries 
reaching below the 1\% level. 
However, the most impressive progress was achieved in the 
analysis of $\Bs \to\jpsi\phi$ decays, 
where new or significantly improved results became recently 
available from CDF, \dzero and LHCb. 
The non-zero decay width difference in the $\Bs-\Bsbar$ system 
is now firmly established, with a relative difference of
$(14\pm2)\%$.
Its sign has also been determined by LHCb: 
the heavy state of the  $\Bs-\Bsbar$ system lives longer 
than the light state, as expected in the Standard Model.
In contrast, and despite the 
recent efforts from Belle, the relative decay width difference in 
the $\Bd-\Bdbar$ system,
which has momentarily reached a slightly better absolute precision,
is still consistent with zero. 
One the other hand, a quantum step
has been achieved in the measurement of 
mixing-induced \CP violation in \Bs decays proceeding through 
the $b\to c\bar{c}s$ transition: the corresponding weak phase 
has been pinned down to a precision below 0.1 radian 
and is so far compatible with the Standard Model expectation.

%% %%% cp(t) & ut highlightss
The measurement of $\sin 2\beta \equiv \sin 2\phi_1$ from $b \to
c\bar{c}s$ transitions such as $\Bz \to \jpsi\KS$ has reached $<3\,\%$
precision: $\sin 2\beta \equiv \sin 2\phi_1 = 0.682 \pm 0.019$.
Measurements of the same parameter using different quark-level processes
provide a consistency test of the Standard Model and allow insight into
possible new physics.  Recent improvements include the use of
time-dependent Dalitz plot analyses of $\Bz \to \KS\Kp\Km$ and $\Bz \to
\KS\pip\pim$ to obtain \CP\ violation parameters for $\phi\KS$,
$f_0(980)\KS$ and $\rho\KS$.  All results among hadronic $b \to s$
penguin dominated decays are currently consistent with the Standard
Model expectations.  Among measurements related to the Unitarity
Triangle angle $\alpha \equiv \phi_2$, results from the
$\rho\rho$ system allow constraints at the level of $\approx
6^\circ$.  Knowledge of the third angle $\gamma \equiv \phi_3$ also
continues to improve.  The world average values of
the parameters in $B \to DK$ decays now show significant direct 
\CP\ violation effects, and determinations of $\gamma$ from the individual experiments now approach the level of $10^\circ$ precision.

%% %%% semileptonic highlights
Regarding semileptonic $B$~meson decays, the $B$ factories Belle
and \babar\ continue to dominate the field and a number of results
have appeared since the last update. Semileptonic decays remain a
focus of interest for theorists: New lattice QCD and light-cone sum
rule results help to understand exclusive transitions. Inclusive
semileptonic decays are understood at full ${\cal
O}(\alpha^2_s)$. Still, the experimental situation is not satisfactory:
While inclusive and exclusive determinations of $\vcb$ agree at the
level of $2\sigma$, inclusive and exclusive measurements of $\vub$
differ by three standard deviations. Clearly more effort on the
experimental and theory side is required in the future.

%% %%% rare decays highlights
The most important new measurements of rare decays are coming from the
LHC.  CMS and LHCb both have restrictive limits for the decays
$B\to\mumu$ and $B_s\to\mumu$.  The sensitivity is approaching the SM
expectations with no significant signals seen yet.  LHCb has already
produced many other results on a wide variety of decays as indicated in 
the tables in Sec.~\ref{sec:rare}.  Belle and \babar\ continue to
produce new results though their rates are dwindling.  It will still
be some years before we see new results from upgraded $B$ factories.

%% %%% b to charm highlights
Many $b$ to charm results from LHCb are included in our report for the  
first time this year, combining with
results from \babar, Belle and CDF to yield a total of 632  
measurements reported in 216 papers.
The huge combined sample of $b$ hadrons allows measurements of decays  
to states with open or hidden charm
content with unprecedented precision.

%%% charm highlights
In the charm sector, $D^0$-$\dbar$ mixing is now well-established.
Measurements of 38 separate observables from five experiments are input 
into a global fit for 10 underlying parameters, and the no-mixing 
hypothesis is excluded at a confidence level corresponding to 
$10.2\sigma$. The mixing parameters
$x$ and $y$ (see Table~\ref{tab_summary3}) differ from zero by 
$2.7\sigma$ and $6.0\sigma$, respectively. The central values are 
consistent with mixing arising from long-distance processes, as
predicted by theory; thus it will probably be difficult to identify 
new physics from mixing alone. The WA value for the observable $y_{\CP}$ 
is positive, which indicates that the \CP-even state is shorter-lived 
as in the $\Kz$--$\Kzb$ system. However, $x$ also appears to be 
positive, which implies that the \CP-even state is heavier, 
unlike in the $\Kz$--$\Kzb$ system. 
%
%Concerning \CP\ violation in the $D^0$-$\dbar$ system,
In the $\Dz$--$\Dzb$ system, 
there is no evidence for \CP violation arising from mixing ($|q/p|\neq 1$) or 
from a phase difference between the mixing amplitude and 
a direct decay amplitude ($\phi\neq 0$). However, both the LHCb 
and CDF experiments have obtained evidence for direct
\CP violation in $D^0\ra K^+K^-$ and $D^0\ra \pi^+\pi^-$ decays. These
experiments measure nonzero values for the {\it difference\/} 
in direct \CP violation between $K^+K^-$ and $\pi^+\pi^-$ modes, which 
requires that direct \CP violation exists in at least
one of them. Inputting these measurements into a global 
fit and also including measurements from Belle and \babar\ 
gives $\Delta a^{\rm dir}_{CP}\neq 0$ with a significance 
greater than $4\sigma$.

%% tau highlights
Concerning tau decays, in this report we include three new 
tau branching fraction measurements from the $B$-factories, 
and we provide more information on the tau branching fraction 
fit. The \Vus calculation uses now a more complete set of tau
branching fractions to strange final states, and thanks primarily to
improvements in QCD lattice predictions, two tau determinations of \Vus
have reduced errors. For the first time, we compute an average of all \Vus
determinations with tau data.
