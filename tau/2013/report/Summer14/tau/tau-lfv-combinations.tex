%% -*- mode: LaTeX; TeX-master: "../EndOfYear11.tex" -*-
\tausection{Combination of Upper Limits on \mtau LFV Branching Fractions}
\cutname{lfv-limits-combinations.html}
\label{sec:tau:lfv_com}
\newcommand{\cls}{\ensuremath{CL_s}\xspace}

There is no standard and generally agreed method of combining experimental upper
limits on branching fractions. The published \mtau LFV searches upper
limits are extracted from the data with a variety of methods and there is
no straightforward way to combine them with a uniform simple procedure.
Therefore we re-compute the published limits with a suitable
method to provide a way to combine the experimental findings with a uniform
procedure. We proceed as follows:
\begin{itemize}

\item
  for each published limit, a new limit is
  computed with the \cls method \cite{Mistlberger:2012rs} using the published
  information on the observed candidates, the expected background, the signal
  efficiency, and the number of recorded \mtau decays.

\item
  the newly computed limits are combined to provide the HFAG combined limits.

\end{itemize}
The \cls method is a well known and widely used method for extracting upper
limits (see the Statistics review of PDG 2013~\cite{PDG_2013}). Limits
computed with the \cls method are easily combined (see below) and are
resilient and conservative in presence of a downward fluctuation of the
number of detected signal candidates. This is observed for several
published \mtau LFV searches, with the consequence that the published
limits are better than the experimental sensitivity when computed with
methods like Feldman-Cousins~\cite{Feldman:1998}.

The \cls method is based on two hypothesis: signal plus background and
background only. We calculate the observed confidence levels for the two
hypothesis:
\begin{align}
&CL_{s+b} = P_{s+b}(Q \leq Q_{obs}) = \int_{- \infty}^{Q_{obs}} \frac{dP_{s+b}}{dQ} dQ,
\label{eq:tau:clspdf1} \\
&CL_{b} = P_{b}(Q \leq Q_{obs}) = \int_{- \infty}^{Q_{obs}} \frac{dP_{b}}{dQ} dQ,
\label{eq:tau:clspdf2}
\end{align}
where $CL_{s+b}$ is the confidence level observed for the signal plus background
hypothesis, $CL_{b}$ is the confidence level observed for the background only
hypothesis, $\frac{dP_{s+b}}{dQ}$ and $\frac{dP_{b}}{dQ}$ are the probability
distribution functions (p.d.f.'s) for the two corresponding hypothesis and
$Q$ is called the test statistics. The \cls value is defined as the ratio
between the confidence level for the signal plus background hypothesis to
the confidence level for the background hypothesis:
\begin{equation}
CL_s=\dfrac{CL_{s+b}}{CL_{b}}.
\end{equation}
When multiple results are combined, the p.d.f.'s in
Equations~\ref{eq:tau:clspdf1} and \ref{eq:tau:clspdf2} are the
product of the individual p.d.f.'s,
\begin{equation}
CL_s = \dfrac{\prod_{i=1}^{N_{chan}}\sum_{n=0}^{n_i} \dfrac{e^{-(s_i+b_i)} (s_i+b_i)^{n}}{n!} }{\prod_{i=1}^{n_{chan}}  \sum_{n=0}^{n_i} \dfrac{e^{-b_i} b_i^{n}}{n!}}    \dfrac{\prod_{j=1}^{n} s_iS_i(x_{ij})+b_iB_i(x_{ij})}{\prod_{j=1}^{n_i}B_i(x_{ij})}~,
\end{equation}
where $N_{chan}$ is the number of results (or channels), and, for each channel $i$,
$n_i$ is the number of observed candidates, $x_{ij}$ are the values of the
discriminating variables (with index $j$), $s_i$ and $b_i$ are the number
of signal and background events and $S_i$, $B_i$ are the probability
distribution functions of the discriminating variables.

The extraction of the upper limits is performed using code 
For the technical implementation we used an implementation provided by Tom Junk~\cite{junk:2007:cdfnote}. For each experiment we estimated the number of expected signal events using the formula:
\begin{equation}
\begin{split}
s_i & =2\mathcal{L}_i\sigma_{\tau\tau}Br(\tau \rightarrow LFV) \\
& =\frac{Br(\tau \rightarrow LFV)}{\alpha},
\end{split}
\end{equation}
where $\mathcal{L}_i$ is the integrated luminosity of an given experiment, $\sigma_{\tau\tau}$ is the cross section of a process $e^+ e^- \rightarrow  \tau^+ \tau^-$, $Br(\tau \rightarrow LFV)$ is the branching fraction of searched process and $\alpha$ is so called normalization factor\footnote{In case of LHCb results we take the the normalization directly from the appropriate paper.}. The systematics uncertainties are evaluated in pseudoexperiments generation by varying nuisance parameters($s_i$, $b_i$). The values are varied accordingly to Gaussian distribution width equal to corresponding systematic.\\
As mentioned at the begging most of the limits calculated by the B factories used Feldman-Cousin method, for comparison purposes in table~\ref{tab:tau:lfv-upper-limits_comb} we reported individual limits calculated with \cls method as well our combined limit. The same numbers are shown in Figure~\ref{fig:tau:lfv-limits-plot_average}. CLEO results were not taken into account because of negligible impact, also if a result between Belle an \babar is an order of magnitude stronger we did not perform a combination of this limits.



\begin{center}
\begin{longtable}{lcl@{}rl}

\caption{HFAG 2014  upper limit for the lepton flavor violating $\tau$ decay modes combinations. Individual experiments limits are recalculated using \cls method and final combination is reported. For convenience, the decay modes
are grouped in categories labelled according to their particle content. The label ``(L)'' in the category column means that the decay mode implies lepton number violation as well as the lepton flavor violation. ``BNV'' indicates that the channel is Baryon Number Violating.
\label{tab:tau:lfv-upper-limits_comb}}%
\\
\hline
\multicolumn{1}{l}{\bfseries Decay mode} &
\multicolumn{1}{l}{\bfseries Category} &
\multicolumn{2}{c}{\bfseries \begin{tabular}{@{}c@{}}90\% CL\\Limit\end{tabular}} &
\multicolumn{1}{l}{\bfseries Exp.}\\% &
%\multicolumn{1}{l}{\bfseries Ref.} \\
\hline
\endfirsthead
\multicolumn{5}{c}{{\bfseries \tablename\ \thetable{} -- continued from previous page}} \\ \hline
\multicolumn{1}{l}{\bfseries Decay mode} &
\multicolumn{1}{l}{\bfseries Category} &
\multicolumn{2}{c}{\bfseries \begin{tabular}{@{}c@{}}90\% CL\\Limit\end{tabular}} &
\multicolumn{1}{l}{\bfseries Exp.}\\% &
%\multicolumn{1}{l}{\bfseries Ref.} \\
\hline
\endhead
%
%  l\gamma
%   

\begin{ensuredisplaymath}
\Gamma_{156} =  {e^- \gamma} 
\end{ensuredisplaymath}
 &\(l\gamma\) & \( <\; \) &  \(5.4 \cdot 10^{-8}\)         & HFAG \\
 &            & \( <\; \) & \(22.0 \cdot 10^{-8}\)         & Belle \\
 &            & \( <\; \) & \(6.1 \cdot 10^{-8}\)         & \babar  \\ 
%\hline
\begin{ensuredisplaymath}
\Gamma_{157} =  {\mu^- \gamma} 
\end{ensuredisplaymath}
 &            & \( <\; \) &  \(5.0 \cdot 10^{-8}\)         & HFAG \\
 &            & \( <\; \) & \(17.0 \cdot 10^{-8}\)         & Belle  \\
 &            & \( <\; \) & \(5.9 \cdot 10^{-8}\)         & \babar   \\ 
\hline
%
%  lP0
\begin{ensuredisplaymath}
\Gamma_{160} =  {e^- K^0_S} 
\end{ensuredisplaymath}
 & \(lP^0 \)  & \( <\; \) & \(1.4 \cdot 10^{-8}\)         & HFAG  \\
 &            & \( <\; \) & \(1.8 \cdot 10^{-8}\)         & Belle  \\
 &            & \( <\; \) & \(4.7 \cdot 10^{-8}\)         & \babar   \\ 
%\hline
\begin{ensuredisplaymath}
\Gamma_{161} =  {\mu^- K^0_S} 
\end{ensuredisplaymath}
 &            & \( <\; \) & \(1.5 \cdot 10^{-8}\)         & HFAG  \\
&            & \( <\; \) & \(1.7 \cdot 10^{-8}\)         & Belle  \\
 &            & \( <\; \) & \(6.9 \cdot 10^{-8}\)         & \babar   \\ 
\hline
%
% l V0
%
\begin{ensuredisplaymath}
\Gamma_{164} =  {e^- \rho^0} 
\end{ensuredisplaymath}
 &  \(l V^0\) & \( <\; \) & \(1.5 \cdot 10^{-8}\)         & HFAG  \\
 &            & \( <\; \) & \(1.9 \cdot 10^{-8}\)         & Belle\\
 &            & \( <\; \) & \(5.2 \cdot 10^{-8}\)         & \babar   \\ 
%\hline
\begin{ensuredisplaymath}
\Gamma_{165} =  {\mu^- \rho^0} 
\end{ensuredisplaymath}
 &            & \( <\; \) & \(1.5 \cdot 10^{-8}\)         & HFAG  \\
 &            & \( <\; \) & \(2.1 \cdot 10^{-8}\)         & Belle\\
 &            & \( <\; \) & \(6.2 \cdot 10^{-8}\)         & \babar \\ 
%\hline
\begin{ensuredisplaymath}
\Gamma_{168} =  {e^- K^*(892)^0} 
\end{ensuredisplaymath}
 &            & \( <\; \) & \(2.3 \cdot 10^{-8}\)         & HFAG \\
 &            & \( <\; \) & \(3.4 \cdot 10^{-8}\)         & Belle \\
 &            & \( <\; \) & \(6.1 \cdot 10^{-8}\)         & \babar   \\ 
%\hline
\begin{ensuredisplaymath}
\Gamma_{169} =  {\mu^- K^*(892)^0} 
\end{ensuredisplaymath}
 &            & \( <\; \) & \(6.0 \cdot 10^{-8}\)         & HFAG \\
 &            & \( <\; \) & \(6.6 \cdot 10^{-8}\)         & Belle \\
 &            & \( <\; \) & \(17.0 \cdot 10^{-8}\)         & \babar   \\ 
%\hline
\begin{ensuredisplaymath}
\Gamma_{170} =  {e^- \bar{K}^*(892)^0} 
\end{ensuredisplaymath}
 &            & \( <\; \) & \(2.2 \cdot 10^{-8}\)         & HFAG \\
 &            & \( <\; \) & \(3.3 \cdot 10^{-8}\)         & Belle \\
 &            & \( <\; \) & \(5.6 \cdot 10^{-8}\)         & \babar   \\ 
%\hline
\begin{ensuredisplaymath}
\Gamma_{171} =  {\mu^- \bar{K}^*(892)^0} 
\end{ensuredisplaymath}
 &            & \( <\; \) & \(4.2 \cdot 10^{-8}\)         & HFAG  \\
 &            & \( <\; \) & \(6.3 \cdot 10^{-8}\)         & Belle  \\
 &            & \( <\; \) & \(9.1 \cdot 10^{-8}\)         & \babar \\ 
%\hline

\begin{ensuredisplaymath}
\Gamma_{176} =  {e^- \phi} 
\end{ensuredisplaymath}
 &            & \( <\; \) & \(2.0 \cdot 10^{-8}\)         & HFAG \\
 &            & \( <\; \) & \(3.5 \cdot 10^{-8}\)         & Belle \\
 &            & \( <\; \) & \(4.3 \cdot 10^{-8}\)         & \babar   \\ 
%\hline
\begin{ensuredisplaymath}
\Gamma_{177} =  {\mu^- \phi} 
\end{ensuredisplaymath}
 &            & \( <\; \) &\(6.8 \cdot 10^{-8}\)         & HFAG \\
 &            & \( <\; \) &\(7.6 \cdot 10^{-8}\)         & Belle  \\
 &            & \( <\; \) & \(18.0 \cdot 10^{-8}\)         & \babar   \\ 
%\hline
\begin{ensuredisplaymath}
\Gamma_{166} =  {e^- \omega} 
\end{ensuredisplaymath}
 &            & \( <\; \) & \(3.3 \cdot 10^{-8}\)         & HFAG \\
 &            & \( <\; \) & \(5.2 \cdot 10^{-8}\)         & Belle  \\
 &            & \( <\; \) & \(9.4 \cdot 10^{-8}\)         & \babar    \\ 
%\hline
\begin{ensuredisplaymath}
\Gamma_{167} =  {\mu^- \omega} 
\end{ensuredisplaymath}
 &            & \( <\; \) & \(4.0 \cdot 10^{-8}\)         & HFAG \\
 &            & \( <\; \) & \(6.1 \cdot 10^{-8}\)         & Belle  \\
 &            & \( <\; \) &  \(10.0 \cdot 10^{-8}\)         & \babar   \\ 
\hline
%
% lll
%
\begin{ensuredisplaymath}
\Gamma_{178} =  {e^- e^+ e^-} 
\end{ensuredisplaymath}
 &  \(lll\)   & \( <\; \) & \(1.4 \cdot 10^{-8}\)         & HFAG \\
 &            & \( <\; \) & \(2.7 \cdot 10^{-8}\)         & Belle  \\
 &            & \( <\; \) & \(3.1 \cdot 10^{-8}\)         & \babar    \\ 
%\hline
\begin{ensuredisplaymath}
\Gamma_{181} =  {\mu^- e^+ e^-} 
\end{ensuredisplaymath}
 &            & \( <\; \) & \(1.1 \cdot 10^{-8}\)         & HFAG \\
 &            & \( <\; \) & \(1.7 \cdot 10^{-8}\)         & Belle \\
 &            & \( <\; \) & \(3.0 \cdot 10^{-8}\)         & \babar    \\ 
%\hline
\begin{ensuredisplaymath}
\Gamma_{179} =  {e^- \mu^+ \mu^-} 
\end{ensuredisplaymath}
 &            & \( <\; \) & \(1.6 \cdot 10^{-8}\)         & HFAG \\
 &            & \( <\; \) & \(2.6 \cdot 10^{-8}\)         & Belle  \\
 &            & \( <\; \) & \(4.1 \cdot 10^{-8}\)         & \babar     \\ 
%\hline
\begin{ensuredisplaymath}
\Gamma_{183} =  {\mu^- \mu^+ \mu^-} 
\end{ensuredisplaymath}
 &            & \( <\; \) & \(1.2 \cdot 10^{-8}\)         & HFAG \\
 &            & \( <\; \) & \(2.1 \cdot 10^{-8}\)         & Belle\\
 &            & \( <\; \) & \(4.0 \cdot 10^{-8}\)         & \babar  \\ 
 &            & \( <\; \) & \(4.6 \cdot 10^{-8}\)         & \lhcb   \\ 

%\hline
\begin{ensuredisplaymath}
\Gamma_{182} =  {e^- \mu^+ e^-} 
\end{ensuredisplaymath}
 &            & \( <\; \) & \(8.4 \cdot 10^{-9}\)         & HFAG \\
 &            & \( <\; \) & \(1.4 \cdot 10^{-8}\)         & Belle  \\
 &            & \( <\; \) & \(2.1 \cdot 10^{-8}\)         & \babar    \\ 
%\hline
\begin{ensuredisplaymath}
\Gamma_{180} =  {\mu^- e^+ \mu^-} 
\end{ensuredisplaymath}
 &            & \( <\; \) & \(9.8 \cdot 10^{-9}\)         & HFAG \\
 &            & \( <\; \) & \(1.6 \cdot 10^{-8}\)         & Belle \\
 &            & \( <\; \) & \(2.6 \cdot 10^{-8}\)         & \babar     \\ 
 %
% Lambda h
%
\hline
\begin{ensuredisplaymath}
\Gamma_{211} =  { \pi^- \Lambda } 
\end{ensuredisplaymath}
& BNV & \( <\; \) & \(1.9 \cdot 10^{-8}\)         & HFAG  \\
&                & \( <\; \) & \(3.2 \cdot 10^{-8}\)         & Belle  \\
 &               & \( <\; \) & \(6.7 \cdot 10^{-8}\)        & \babar     \\  
%\hline
\begin{ensuredisplaymath}
\Gamma_{212} =  { \pi^- \bar{\Lambda}} 
\end{ensuredisplaymath}
 &            & \( <\; \) & \(1.8 \cdot 10^{-9}\)         & HFAG \\
 &            & \( <\; \) & \(2.9 \cdot 10^{-8}\)         & Belle  \\
 &            & \( <\; \) & \(6.5 \cdot 10^{-8}\)        & \babar     \\  
%\hline
\begin{ensuredisplaymath}
\Gamma_{213} =  { K^- \Lambda } 
\end{ensuredisplaymath}
 &            & \( <\; \) & \(3.7 \cdot 10^{-9}\)         & HFAG \\
 &            & \( <\; \) & \(4.4 \cdot 10^{-8}\)         & Belle  \\
 &            & \( <\; \) & \(9.2\cdot 10^{-8}\)         & \babar     \\  
%\hline
\begin{ensuredisplaymath}
\Gamma_{214} =  { K^- \bar{\Lambda}} 
\end{ensuredisplaymath}
 &            & \( <\; \) & \(2.0 \cdot 10^{-9}\)         & HFAG \\
 &            & \( <\; \) & \(3.4 \cdot 10^{-8}\)         & Belle \\
 &            & \( <\; \) & \(5.0 \cdot 10^{-8}\)         & \babar     \\  
\hline 
\end{longtable}
\end{center}
