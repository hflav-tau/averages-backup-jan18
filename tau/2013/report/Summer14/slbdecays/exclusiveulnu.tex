% ======================================================================
\subsection{Exclusive CKM-suppressed decays}
\label{slbdecays_b2uexcl}
% ----------------------------------------------
In this section, we list results on exclusive charmless semileptonic branching fractions
and determinations of $\vub$ based on $\Bb\to\pi\ell\nub$ decays.
The measurements are based on two different event selections: tagged
events, in which case the second $B$ meson in the event is fully
reconstructed in either a hadronic decay (``$B_{reco}$'') or in a 
CKM-favored semileptonic decay (``SL''); and untagged events, in which case the selection infers the momentum
of the undetected neutrino based on measurements of the total 
momentum sum of detected particles and knowledge of the initial state.
We present averages for $\Bb\to\rho\ell\nub$ and $\Bb\to\omega\ell\nub$. Moreover, the average for the  branching fraction $\Bb\to\eta\ell\nub$ is presented for the first time. 

The results for the full and partial branching fraction for $\Bb\to\pi\ell\nub$ are given
in Table~\ref{tab:pilnubf} and shown in Figure~\ref{fig:xlnu} (a).   

When averaging these results, systematic uncertainties due to external
inputs, e.g., form factor shapes and background estimates from the
modeling of $\Bb\to X_c\ell\nub$ and $\Bb\to X_u\ell\nub$ decays, are
treated as fully correlated (in the sense of Eq.~\ref{eq:correlrho}).
Uncertainties due to experimental reconstruction effects are treated
as fully correlated among measurements from a given experiment.  Varying
the assumed dependence of the quoted errors on the measured value
for error sources where the dependence was not obvious had no significant impact.

\begin{sidewaystable}[!htb]
\begin{center}
\caption{\label{tab:pilnubf}
Summary of exclusive determinations of $\cbf(\Bb\to\pi
\ell\nub)$. The errors quoted
correspond to statistical and systematic uncertainties, respectively.
Measured branching fractions for $B\rightarrow \pi^0 l \nu$ have been
multiplied by $2\times \tau_{B^0}/\tau_{B^+}$ in accordance with
isospin symmetry. The labels ``$B_{reco}$'' and ``SL'' tags refer to
the type of $B$
decay tag used in a measurement, and ``untagged'' refers to an untagged measurement.}
\begin{small}
\begin{tabular}{|lcccc|}
\hline
& $\cbf [10^{-4}]$
& $\cbf(q^2<12\,\gev^2/c^2) [10^{-4}]$
& $\cbf(q^2<16\,\gev^2/c^2) [10^{-4}]$
& $\cbf(q^2>16\,\gev^2/c^2) [10^{-4}]$
\\
\hline\hline
CLEO $\pi^+,\pi^0$~\cite{Adam:2007pv}
& $1.38\pm 0.15\pm 0.11\ $ 
& $0.70\pm 0.12\pm 0.07$
& $0.97\pm 0.13\pm 0.09$
& $0.41\pm 0.08\pm 0.04$
\\ 
\babar $\pi^+,\pi^0$~\cite{delAmoSanchez:2010af}
& $1.41\pm 0.05\pm 0.08\ $
& $0.88\pm 0.04\pm 0.05$
& $1.10\pm 0.04\pm 0.06$
& $0.32\pm 0.02\pm 0.03$
\\  
\babar $\pi^+$~\cite{delAmoSanchez:2010zd}
& $1.42\pm 0.05\pm 0.07\ $
& $0.83\pm 0.03\pm 0.04$
& $1.09\pm 0.04\pm 0.05$
& $0.33\pm 0.03\pm 0.03$
\\  
Belle $\pi^+$~\cite{Ha:2010rf}
& $1.49\pm 0.04\pm 0.07\ $
& $0.83\pm 0.03\pm 0.04$
& $1.10\pm 0.03\pm 0.05$
& $0.40\pm 0.02\pm 0.02$
\\  
Belle SL $\pi^+$~\cite{Hokuue:2006nr}
& $1.42\pm 0.19\pm 0.15\ $
& $0.80\pm 0.14\pm 0.08$
& $1.04\pm 0.16\pm 0.11$
& $0.37\pm 0.10\pm 0.04$
\\ 
Belle SL $\pi^0$~\cite{Hokuue:2006nr}
& $1.41\pm 0.26\pm 0.15\ $
& $0.71\pm 0.17\pm 0.08$
& $1.04\pm 0.22\pm 0.12$
& $0.36\pm 0.15\pm 0.04$
\\ 
\babar SL $\pi^+$~\cite{Aubert:2008bf}
& $1.39\pm 0.21\pm 0.08\ $
& $0.77\pm 0.14\pm 0.05$
& $0.92\pm 0.16\pm 0.05$
& $0.46\pm 0.13\pm 0.03$
\\ 
\babar SL $\pi^0$~\cite{Aubert:2008bf}
& $1.78\pm 0.28\pm 0.15\ $
& $1.07\pm 0.20\pm 0.09$
& $1.34\pm 0.22\pm 0.11$
& $0.44\pm 0.17\pm 0.06$
\\ 
\babar $B_{reco}$ $\pi^+$~\cite{Aubert:2006ry}
& $1.07\pm 0.27\pm 0.19\ $
& $0.26\pm 0.15\pm 0.04$
& $0.42\pm 0.18\pm 0.06$
& $0.65\pm 0.20\pm 0.13$
\\ 
\babar $B_{reco}$ $\pi^0$~\cite{Aubert:2006ry}
& $1.52\pm 0.41 \pm0.30\ $
& $0.67\pm 0.30\pm 0.12$
& $1.04\pm 0.35\pm 0.18$
& $0.48\pm 0.22\pm 0.12$
\\ 
Belle $B_{reco}$ $\pi^+$~\cite{:2008kn}
& $1.12\pm 0.18\pm 0.05\ $
& $0.65\pm 0.14\pm 0.03$
& $0.85\pm 0.16\pm 0.04$
& $0.26\pm 0.08\pm 0.01$
\\ 
Belle $B_{reco}$ $\pi^0$~\cite{:2008kn}
& $1.22\pm 0.22\pm 0.05\ $
& $0.65\pm 0.19\pm 0.03$
& $0.80\pm 0.19\pm 0.03$
& $0.41\pm 0.11\pm 0.02$
\\  \hline
{\bf Average}
& \mathversion{bold}$1.42\pm 0.03\pm 0.04\ $
& \mathversion{bold}$0.81\pm 0.02\pm 0.03$
& \mathversion{bold}$1.05\pm 0.02\pm 0.03$
& \mathversion{bold}$0.37\pm 0.01\pm 0.02$
\\ 
\hline
\end{tabular}\\
\end{small}
\end{center}
\end{sidewaystable}


\begin{figure}[!ht]
 \begin{center}
  \unitlength1.0cm % coordinates in cm
  \begin{picture}(14.,8.0)  %ys(25.,6.)
   \put(  8.0,  0.0){\includegraphics[width=8.0cm]{figures/slb/rholnu.pdf}}
   \put( -0.5,  0.0){\includegraphics[width=8.0cm]{figures/slb/pilnu11.eps}}
   \put(  5.5,  7.3){{\large\bf a)}}  
   \put( 14.4,  7.3){{\large\bf b)}}
   \end{picture} \caption{
(a) Summary of exclusive determinations of $\cbf(\Bb\to\pi
\ell\nub)$ and their average.
Measured branching fractions for $B\rightarrow \pi^0 l \nu$ have been
multiplied by $2\times \tau_{B^0}/\tau_{B^+}$ in accordance with
isospin symmetry. The labels ``$B_{reco}$'' and ``SL''
refer to type of $B$ decay tag used in a measurement. ``untagged'' refers to an untagged measurement.
(b) Summary of exclusive determinations of $\cbf(\Bb\to\rho\ell\nub)$ and their average.
}
\label{fig:xlnu}
\end{center}
\end{figure}

The determination of \vub\ from $\Bb\to\pi\ell\nub$ decays is
shown in Table~\ref{tab:pilnuvub}, and uses our averages for the partial branching
fractions given in Table~\ref{tab:pilnubf}. Two theoretical approaches are
used: unquenched Lattice QCD and QCD light-cone sum rules.
Lattice calculations of the form factors are limited to small hadron momenta, i.e.
large $q^2$, while calculations based on light-cone sum rules are restricted
to small $q^2$. 


\begin{table}[hbtf]
\caption{\label{tab:pilnuvub}
Determinations of \vub\ based on the average partial
$\Bb\to\pi\ell\nub$ decay branching fractions stated in
Table~\ref{tab:pilnubf}. 
The $q^2$ ranges for the partial branching fractions corresponding to the 
validity ranges of the form factor calculations are indicated. 
The first uncertainty is experimental and the second is from theory.  
}
%\scriptsize
%\vspace{5mm} 
\begin{center}
\renewcommand{\arraystretch}{1.2}
\begin{tabular}{|lcc|}
\hline
Method                                         & $q^2$ range [$\gev^2/c^2$] & $\Vub [10^{-3}]$ \\\hline\hline
Khodjamirian et al. (LCSR) ~\cite{Khodjamirian:2011ub} & 0 -- 12                    & $3.41\pm 0.06 {}^{+0.37}_{-0.32}$ \\ \hline
Ball \& Zwicky (LCSR)~\cite{Ball:2004ye}              & 0 -- 16                    & $3.58\pm 0.06 {}^{+0.59}_{-0.40}$ \\ \hline
HPQCD (LQCD)~\cite{Dalgic:2006dt}                     & 16 -- 26.4                 & $3.52\pm 0.08 {}^{+0.61}_{-0.40}$ \\  \hline
FNAL/MILC (LQCD)~\cite{Bailey:2008wp}                 & 16 -- 26.4                 & $3.36\pm 0.08 {}^{+0.37}_{-0.31}$ \\ 
%APE, $q^2>16\,\gev^2/c^2$~\cite{Abada:2000ty}         & $3.72\pm 0.21 {}^{+1.43}_{-0.66}$ \\ 
\hline
\end{tabular}
\end{center}
\end{table}


An alternative method to determine \vub\ from $\Bb\to\pi\ell\nub$ decays that makes use
of the measurement over the full $q^2$ range is based on a simultaneous fit of the 
BCL (Bourrely, Caprini, Lellouch) form factor parameterization to the data and the LQCD predictions.
The result of the simultaneous fit to the three untagged measurements from \babar and Belle and the 
FNAL/MILC LQCD calculations is shown in Figure~\ref{fig:vub_pilnu_simultaneous}.
A value of $\Vub = (3.23 \pm 0.30) \times 10^{-3}$ is obtained.

\begin{figure}[!ht]
 \begin{center}
  \unitlength1.0cm % coordinates in cm
  \begin{picture}(14.,8.0)  %ys(25.,6.)
   \put( -0.0,  0.0){\includegraphics[width=8.0cm]{figures/slb/vub_pilnu_combinedFit.eps}}
   \end{picture} \caption{
    Simultaneous fit of the untagged $\Bb\to\pi\ell\nub$ measurements from \babar and Belle and the
    FNAL/MILC LQCD calculations. This fit yields $\Vub = (3.23 \pm 0.30) \times 10^{-3}$.
}
\label{fig:vub_pilnu_simultaneous}
\end{center}
\end{figure}


The branching fractions for 
$\Bb\to \rho\ell\nub$ decays is computed based on the measurements in
Table~\ref{tab:rholnu} and is shown in Figure~\ref{fig:xlnu} (b). The determination of $\Vub$
from these other channels looks less promising than for
$\Bb\to\pi\ell\nub$ and at the moment it is not extracted.

\begin{table}[!htb]
\begin{center}
\caption{Summary of exclusive determinations of $\cbf(\Bb\to\rho
\ell\nub)$. The errors quoted
correspond to statistical and systematic uncertainties, respectively.}
\label{tab:rholnu}
\begin{small}
\begin{tabular}{|lc|}
\hline
& $\cbf [10^{-4}]$
\\
\hline\hline
CLEO $\rho^+$~\cite{Behrens:1999vv}
& $2.75\pm 0.41\pm 0.52\ $ 
\\ 
CLEO $\rho^+$~\cite{Adam:2007pv}
& $2.93\pm 0.37\pm 0.37\ $ 
\\ 
%BABAR $\rho^+$~\cite{Aubert:2005cd}
%& $2.16\pm 0.21\pm 0.57\ $
% Belle Breco
Belle $\rho^+$~\cite{Sibidanov:2013rkk}
& $3.22\pm 0.27\pm 0.24\ $
\\
Belle $\rho^0$~\cite{Sibidanov:2013rkk}
& $3.39\pm 0.18\pm 0.18\ $
\\
%Belle SL
Belle $\rho^+$~\cite{Hokuue:2006nr}
& $2.17\pm 0.54\pm 0.32\ $
\\
Belle $\rho^0$~\cite{Hokuue:2006nr}
& $2.47\pm 0.43\pm 0.33\ $
\\
\babar $\rho^+$~\cite{delAmoSanchez:2010af}
& $1.98\pm 0.21\pm 0.38\ $
\\
\babar $\rho^0$~\cite{delAmoSanchez:2010af}
& $1.87\pm 0.19\pm 0.32\ $

\\  \hline
{\bf Average}
& \mathversion{bold}$2.94 \pm 0.11\pm 0.17 $
%\hline
%{\bf Average of published results}
%& \mathversion{bold}$2.34 \pm 0.15\pm 0.24 $
\\ 
\hline
\end{tabular}\\
\end{small}
\end{center}
\end{table}


We also report the branching fraction average for $\Bb\to\omega\ell\nub$, $\Bb\to\eta\ell\nub$ 
and $\Bb\to\eta'\ell\nub$. The measurements for $\Bb\to\omega\ell\nub$ are reported in Table~\ref{tab:omegalnu} 
and shown in Figure~\ref{fig:xulnu1}, while the ones for $\Bb\to\eta\ell\nub$ and  $\Bb\to\eta'\ell\nub$ are reported in Table~\ref{tab:etalnu} and~\ref{tab:etaprimelnu},  and are shown in Figure~\ref{fig:xulnu2}. 

\begin{table}[!htb]
\begin{center}
\caption{Summary of exclusive determinations of $\cbf(\Bb\to\omega
\ell\nub)$. The errors quoted
correspond to statistical and systematic uncertainties, respectively.}
\label{tab:omegalnu}
\begin{small}
\begin{tabular}{|lc|}
\hline
& $\cbf [10^{-4}]$
\\
\hline\hline
Belle $\omega$~\cite{Schwanda:2004fa}
& $1.30\pm 0.40\pm 0.36\ $
\\
\babar $\omega$~\cite{Lees:2012vv}
& $1.19\pm 0.16\pm 0.09\ $
\\  
\babar $\omega$~\cite{Lees:2012mq}
& $1.21\pm 0.14\pm 0.08\ $
\\  
Belle $\omega$~\cite{Sibidanov:2013rkk}
& $1.07\pm 0.16\pm 0.07 $
\\
\babar $\omega$~\cite{Lees:2013gja}
& $1.35\pm 0.21\pm 0.11\ $
\\  

\hline

{\bf Average}
& \mathversion{bold}$1.19 \pm 0.08 \pm 0.06\ $
\\ 
\hline
\end{tabular}\\
\end{small}
\end{center}
\end{table}

\begin{table}[!htb]
\begin{center}
\caption{Summary of exclusive determinations of $\cbf(\Bb\to\eta
\ell\nub)$. The errors quoted
correspond to statistical and systematic uncertainties, respectively.}
\label{tab:etalnu}
\begin{small}
\begin{tabular}{|lc|}
\hline
& $\cbf [10^{-4}]$
\\
\hline\hline
CLEO $\eta$~\cite{Gray:2007pw}
& $0.44\pm 0.23\pm 0.11\ $
\\
BABAR $\eta$~\cite{Aubert:2008ct}
& $0.31\pm 0.06\pm 0.08\ $
\\ 
BABAR $\eta$~\cite{Aubert:2008bf}
& $0.64\pm 0.20\pm 0.03\ $
\\
BABAR $\eta$~\cite{Lees:2012vv}
& $0.36\pm 0.05\pm 0.04\ $
\\  
 \hline
{\bf Average}
& \mathversion{bold}$0.37 \pm 0.04 \pm 0.04 $
\\ 
\hline
\end{tabular}\\
\end{small}
\end{center}
\end{table}

\begin{table}[!htb]
\begin{center}
\caption{Summary of exclusive determinations of $\cbf(\Bb\to\eta'
\ell\nub)$. The errors quoted
correspond to statistical and systematic uncertainties, respectively.}
\label{tab:etaprimelnu}
\begin{small}
\begin{tabular}{|lc|}
\hline
& $\cbf [10^{-4}]$
\\
\hline\hline
CLEO $\eta'$~\cite{Gray:2007pw}
& $2.71\pm 0.80\pm 0.56\ $
\\
BABAR $\eta'$~\cite{Aubert:2008bf}
& $0.04\pm 0.22\pm 0.04\ $
\\ 
BABAR $\eta'$~\cite{Lees:2012vv}
& $0.24\pm 0.08\pm 0.03\ $
\\  
 \hline
{\bf Average}
& \mathversion{bold}$0.23 \pm 0.08 \pm 0.03 $
\\ 
\hline
\end{tabular}\\
\end{small}
\end{center}
\end{table}


\begin{figure}[!ht]
 \begin{center}
  \unitlength1.0cm % coordinates in cm
  \begin{picture}(14.,8.0)  %ys(25.,6.)
   \put( -0.5,  0.0){\includegraphics[width=8.0cm]{figures/slb/omegalnu.pdf}}
   \put(  5.5,  7.3){{\large\bf a)}}  
   \end{picture} \caption{
(a) Summary of exclusive determinations of $\cbf(\Bb\to\omega\ell\nub)$ and their average.
}
\label{fig:xulnu1}
\end{center}
\end{figure}

\begin{figure}[!ht]
 \begin{center}
  \unitlength1.0cm % coordinates in cm
  \begin{picture}(14.,8.0)  %ys(25.,6.)
   \put( -0.5,  0.0){\includegraphics[width=8.0cm]{figures/slb/etalnu.pdf}}
   \put( 8.0,  0.0){\includegraphics[width=8.0cm]{figures/slb/etaprimelnu.pdf}} 
   \put(  5.5,  7.3){{\large\bf a)}}     
   \put( 18.4,  7.3){{\large\bf b)}}
   
   \end{picture} \caption{
(a) Summary of exclusive determinations of $\cbf(\Bb\to\eta\ell\nub)$ and their average.
(b) Summary of exclusive determinations of $\cbf(\Bb\to\eta'\ell\nub)$ and their average.
}
\label{fig:xulnu2}
\end{center}
\end{figure}

%Branching fractions for other $\Bb\to X_u\ell\nub$ decays are given in
%Table~\ref{tab:xslother}. 
%\input{tables/slb/xslother.tex}

% ----------------------------------------------------------------------
