% =====================================================================
\subsection{Summary of the $B\to D^{(*)}\tau \nu_\tau$ decays}
\label{slbdecays_b2dtaunu}
% -------------------------------------------
%This section contains a summary of the existing measurements
%of $\bar B\to D^*\tau^-\bar\nu_\tau$ and $\bar B\to D\tau^-\bar\nu_\tau$
%braching ratios. 

The leptonic and semileptonic decays with $\tau$ in the final state 
are probes of physics beyond the SM. In the SM these decays proceed via the $W$ emission
diagrams. In models with extended Higgs sectors, such as the Two Higgs Doublet Models (2HDM)
or the MSSM, charged Higgs can contribute to the decay amplitude at the tree level. These further contributions
can affect the branching fraction. Compared to $B^+\to\tau\nu_\tau$, the $B\to D^{(*)}\tau \nu_\tau$ decay
has advantages: the branching fraction is relatively high, because it is not Cabibbo-suppressed, and because
it is a three-body decay, many observables beside the branching fraction can be studied, such as the $D^*$
polarisation, or the $q^2$ distribution (see Ref.~\cite{Duraisamy:2014sna} and reference therein for recent 
calculations).

The $B^0\to D^{*+}\tau\nu_\tau$ decay was first observed by Belle~\cite{Matyja:2007kt} performing 
an inclusive reconstruction of the $B_{tag}$ candidates using all the particles that remain after the selection
of the $B_{sig}$ decay products. Since than, both \babar and Belle have published improved measurements 
and have found evidence for the $B\to D\tau\nu_\tau$ decays~\cite{Aubert:2007dsa,Adachi:2009qg,Bozek:2010xy}. 
The most powerful way to study these decays is the full hadronic $B_{tag}$ technique widely used by both \babar and Belle.

Using the full dataset and an improved $B_{tag}$ selection, \babar measured~\cite{Lees:2012xj} the ratios:  
%
\begin{eqnarray}
{\cal R}(D)&=&\dfrac{ {\cal B}(B\to D\tau\nu_\tau) }{ {\cal B}(B\to D\ell\nu_\ell) }=0.440\pm 0.058\pm 0.042 \\
{\cal R}(D^*)&=&\dfrac{ {\cal B}(B\to D^*\tau\nu_\tau) }{ {\cal B}(B\to D^*\ell\nu_\ell) }=0.332\pm 0.024\pm 0.018 %
\end{eqnarray}
%
where $\ell=e,\mu$ and the $B^0$ and $B^+$ are combined in a isospin-constrained fit. The ratios ${\cal R}(D)$
and ${\cal R}(D^*)$ are independent of $|V_{cb}|$ and to a large extent of the parametrizations of the 
form factors, so the SM predictions for these ratios are quite precise, ${\cal R}(D)=0.297\pm 0.017$ and
 ${\cal R}(D)=0.252\pm 0.003$ (results obtained in Ref.\cite{Lees:2012xj,Lees:2013uzd} 
 updating the calculations in Ref.~\cite{Kamenik:2008tj,Fajfer:2012vx}
 with the recent $B\to D^{(*)}$ measurements from the B-Factories). The \babar result exceed SM predictions, in 
both $D$ and $D^*$ channels, by $2.0\sigma$ and $2.7\sigma$ respectively. 
The combined result disagree with the SM by $3.4\sigma$.  \babar also interpreted these measurements
in terms of the 2HDM type-II and found that their results are not compatible with this model for any value
of $\tan\beta$ and $m_H$.  
The result obtained by \babar and Belle on ${\cal R}(D)$ and ${\cal R}(D^*)$ are reported in Tab.\ref{tab:dtaunu}.
Before a published result by Belle using the new $B_{tag}$ reconstruction, we do not attempt to average the existing results.

\begin{table}[!htb]
\begin{center}
\caption{Summary of the results on ${\cal R}(D)$ and ${\cal R}(D^*)$. The errors quoted
correspond to statistical and systematic uncertainties, respectively.}
\label{tab:dtaunu}
\begin{small}
\begin{tabular}{|lcc|}
\hline
  & Belle~\cite{Adachi:2009qg}   & \babar~\cite{Lees:2012xj} \\
\hline\hline
${\cal R}(D^0) $    &  $0.70^{+0.19}_{-0.18}~^{+0.11}_{-0.09}$ & $0.99\pm 0.19\pm 0.13$\\
${\cal R}(D^{*0})$  &  $0.47^{+0.11}_{-0.10}~^{+0.06}_{-0.07}$ & $1.71\pm 0.17\pm 0.13$\\
${\cal R}(D^+) $    &  $0.48^{+0.22}_{-0.19}~^{+0.06}_{-0.05}$ & $1.01\pm 0.18\pm 0.12$\\
${\cal R}(D^{*+})$  &  $0.48^{+0.14}_{-0.12}~^{+0.06}_{-0.04}$ & $1.74\pm 0.19\pm 0.12$\\
\hline
\end{tabular}\\
\end{small}
\end{center}
\end{table}

 
 
 
 
