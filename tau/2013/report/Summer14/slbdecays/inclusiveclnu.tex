% -- \include{b2cincl.tex}
% ======================================================================
\subsection{Inclusive CKM-favored decays}
\label{slbdecays_b2cincl}
% -------------------------------------------

\subsubsection{Global analysis of $\bar B\to X_c\ell^-\bar\nu_\ell$}

The semileptonic width $\Gamma(\bar B\to X_c\ell^-\bar\nu_\ell)$ has
been calculated in the framework of the Operator Product
Expansion. The result is a double-expansion in $\Lambda_{\rm QCD}/m_b$
and $\alpha_s$, which depends on a number of non-perturbative
parameters. These parameters can be measured using other observables
in $\bar B\to X_c\ell^-\bar\nu_\ell$ decays, such as the moments of
the lepton energy and the hadronic mass spectrum.

Two independent sets of theoretical expressions, referred to as
kinetic~\cite{Benson:2003kp,Gambino:2004qm,Gambino:2011cq} and 1S
schemes~\cite{Bauer:2004ve} are available for this kind of
analysis. The non-perturbative parameters in the kinetic scheme
are: the quark masses $m_b$ and $m_c$, $\mu^2_\pi$ and
$\mu^2_G$ at $O(1/m^2_b)$, and $\rho^3_D$ and $\rho^3_{LS}$ at
$O(1/m^3_b)$. In the 1S scheme, the parameters are: $m_b$, $\lambda_1$
at $O(1/m^2_b)$, and $\rho_1$, $\tau_1$, $\tau_2$ and $\tau_3$ at
$O(1/m^3_b)$. Note that due to the different definitions, the results
for the quark masses cannot be compared directly between the two
schemes.

Our analysis uses all available measurements of moments in $\bar B\to
X_c\ell^-\bar\nu_\ell$, excluding only points with too high
correlation to avoid numerical issues. The list of included
measurements is given in
Table~\ref{tab:gf_input}. The only external input is the average
lifetime~$\tau_B$ of neutral and charged $B$~mesons, taken to be
$(1.582\pm 0.007)$~ps (Sect.~\ref{sec:life_mix}).
\begin{table}[!htb]
\caption{Experimental inputs used in the global analysis of $\bar B\to
  X_c\ell^-\bar\nu_\ell$. $n$ is the order of the moment, $c$ is the
  threshold value in GeV. In total, there are 29 measurements from
  \babar, 25 measurements from Belle and 12 from other
  experiments.} \label{tab:gf_input}
\begin{center}
\resizebox{0.99\textwidth}{!}{
\begin{tabular}{|l|l|l|l|}
  \hline
  Experiment
  & Hadron moments $\langle M^n_X\rangle$
  & Lepton moments $\langle E^n_\ell\rangle$
  & Photons moment $\langle E^n_\gamma\rangle$\\
  \hline \hline
  \babar & $n=2$, $c=0.9,1.1,1.3,1.5$ & $n=0$, $c=0.6,1.2,1.5$ & $n=1$,
  $c=1.9,2.0$\\
  & $n=4$, $c=0.8,1.0,1.2,1.4$ & $n=1$, $c=0.6,0.8,1.0,1.2,1.5$ & $n=2$,
  $c=1.9$~\cite{Aubert:2005cua,Aubert:2006gg}\\
  & $n=6$, $c=0.9,1.3$~\cite{Aubert:2009qda} & $n=2$, $c=0.6,1.0,1.5$
  & \\
  & & $n=3$, $c=0.8,1.2$~\cite{Aubert:2009qda,Aubert:2004td} & \\
  \hline
  Belle & $n=2$, $c=0.7,1.1,1.3,1.5$ & $n=0$, $c=0.6,1.0,1.4$ & $n=1$,
  $c=1.8,1.9$\\
  & $n=4$, $c=0.7,0.9,1.3$~\cite{Schwanda:2006nf} & $n=1$,
  $c=0.6,0.8,1.0,1.2,1.4$ & $n=2$, $c=1.8,2.0$~\cite{Limosani:2009qg}\\
  & & $n=2$, $c=0.6,1.0,1.4$ & \\
  & & $n=3$, $c=0.8,1.0, 1.2$~\cite{Urquijo:2006wd} & \\
  \hline
  CDF & $n=2$, $c=0.7$ & & \\
  & $n=4$, $c=0.7$~\cite{Acosta:2005qh} & & \\
  \hline
  CLEO & $n=2$, $c=1.0,1.5$ & & $n=1$, $c=2.0$~\cite{Chen:2001fja}\\
  & $n=4$, $c=1.0,1.5$~\cite{Csorna:2004kp} & & \\
  \hline
  DELPHI & $n=2$, $c=0.0$ & $n=1$, $c=0.0$ & \\
  & $n=4$, $c=0.0$~\cite{Abdallah:2005cx} & $n=2$, $c=0.0$ & \\
  & & $n=3$, $c=0.0$~\cite{Abdallah:2005cx} & \\
  \hline
\end{tabular}
}
\end{center}
\end{table}

Both in the kinetic and 1S scheme, the moments in $\bar B\to
X_c\ell^-\bar\nu_\ell$ are not sufficient to constrain the $b$-quark
mass precisely, which limits the precision of the determination of
$\vcb$. This limitation can be overcome:
\begin{itemize}
  \item by including the photon energy moments in $B\to X_s\gamma$
    into the fit, or
  \item by applying a precise constraint on the $c$-quark mass.
\end{itemize}
For the former, calculations of the $B\to X_s\gamma$~moments are
available both in the kinetic~\cite{Benson:2004sg} and the
1S~scheme~\cite{Bauer:2004ve}. For the latter, we use the $c$-quark
mass calculated in Ref.~\cite{Dehnadi:2011gc} in the kinetic scheme
analysis,
\begin{equation}
  m_c^{\overline{\rm MS}}(3~{\rm GeV})=(0.998\pm 0.029)~{\rm GeV}~.
\end{equation}

\subsubsection{Analysis in the kinetic scheme}
\label{globalfitsKinetic}

The fit relies on the calculations of the spectral moments in $\bar
B\to X_c\ell^-\bar\nu_\ell$~decays described in
Ref.~\cite{Gambino:2011cq}. The
photon energy moments are calculated in
Ref.~\cite{Benson:2004sg}. The theoretical uncertainties and
correlations are estimated as explained in
Ref.~\cite{Schwanda:2008kw}. Namely, we assume 100\% correlation
between calculations of the same moment at different threshold values
and no theory correlation between different moments. The fit
determines $\vcb$ and the 6 non-perturbative parameters mentioned
above.

The result of the fit using the $c$-quark mass constraint is
\begin{eqnarray}
  \vcb & = & (41.88\pm 0.73)\times 10^{-3}~, \\
  m_b^{\rm kin} & = & 4.560\pm 0.023~{\rm GeV}~, \\
  \mu^2_\pi & = & 0.453\pm 0.036~{\rm GeV^2}~,
\end{eqnarray}
with a $\chi^2$ of 33.4 for $55-7$ degrees of freedom. The detailed
result of the fit is given in Table~\ref{tab:gf_res_mc_kin}. This
result is also consistent with the fit using the $B\to
X_s\gamma$~constraint (Table~\ref{tab:gf_res_kin}). An illustration of
the fit is given in Fig.~\ref{fig:gf_res_kin}.
\begin{table}[!htb]
\caption{Fit result in the kinetic scheme, using a precise $c$-quark
  mass constraint. The error matrix of the fit ($\sigma_{\rm fit}$) contains
  experimental and theoretical contributions. The expression for
  calculating $\vcb$ has an additional uncertainty of 1.4\%
  ($\sigma_{\rm th}$). In the lower part of the table, the
  correlation matrix of the parameters is given.} \label{tab:gf_res_mc_kin}
\begin{center}
\resizebox{0.99\textwidth}{!}{
\begin{tabular}{|l|ccccccc|}
  \hline
  & \vcb\ [10$^{-3}$] & $m_b^{\rm kin}$ [GeV] &
  $m_c^{\overline{\rm MS}}$ [GeV] & $\mu^2_\pi$ [GeV$^2$]
  & $\rho^3_D$ [GeV$^3$] & $\mu^2_G$ [GeV$^2$] & $\rho^3_{LS}$ [GeV$^3$]\\
  \hline \hline
  value & 41.88 & \phantom{$-$}4.560 & \phantom{$-$}1.010 &
  \phantom{$-$}0.453 & \phantom{$-$}0.164 & \phantom{$-$}0.229 &
  $-$0.140\\
  $\sigma_{\rm fit}$ & 0.44 & \phantom{$-$}0.023 &
  \phantom{$-$}0.027 & \phantom{$-$}0.036 & \phantom{$-$}0.020 &
  \phantom{$-$}0.043 & \phantom{$-$}0.086\\
  $\sigma_{\rm th}$ & 0.59 & & & & & & \\
  \hline
  $|V_{cb}|$ & 1.000 & $-$0.164 & \phantom{$-$}0.137 &
  \phantom{$-$}0.089 & \phantom{$-$}0.328 & $-$0.324 &
  \phantom{$-$}0.146\\
  $m_b^{\rm kin}$ & & \phantom{$-$}1.000 & \phantom{$-$}0.745 &
  $-$0.117 & $-$0.177 & \phantom{$-$}0.128 & $-$0.179\\
  $m_c^{\overline{\rm MS}}$ & & & \phantom{$-$}1.000
  & $-$0.199 & $-$0.006 & $-$0.433 & \phantom{$-$}0.258\\
  $\mu^2_\pi$ & & & & \phantom{$-$}1.000 & \phantom{$-$}0.335 &
  $-$0.109 & $-$0.078\\
  $\rho^3_D$ & & & & & \phantom{$-$}1.000 & $-$0.308 & $-$0.238\\
  $\mu^2_G$ & & & & & & \phantom{$-$}1.000 & $-$0.323\\
  $\rho^3_{LS}$ & & & & & & & \phantom{$-$}1.000\\
  \hline
\end{tabular}
}
\end{center}
\end{table}
\begin{table}[!htb]
\caption{Fit result in the kinetic scheme for different
  constraints. Refer to the text for more
  details.} \label{tab:gf_res_kin}
\begin{center}
\begin{tabular}{|c|c|c|c|c|}
  \hline
  Constraint & $\vcb$ [10$^{-3}$] & $m_b^{\rm kin}$ [GeV] &
  $\mu^2_\pi$ [GeV$^2$] & $\chi^2/$d.o.f.\\
  \hline \hline
  $B\to X_s\gamma$ & $41.94\pm 0.43_{\rm fit}\pm 0.59_{\rm th}$ &
  $4.574\pm 0.032$ & $0.459\pm 0.037$ & $27.0/(66-7)$\\
  $m_c^{\overline{\rm MS}}(3~{\rm GeV})$ & $41.88\pm 0.44_{\rm fit}\pm
  0.59_{\rm th}$ & $4.560\pm 0.023$ & $0.453\pm 0.036$ &
  $33.4/(55-7)$\\
  \hline
\end{tabular}
\end{center}
\end{table}
\begin{figure}
\begin{center}
  \includegraphics[width=0.45\columnwidth]{figures/slb/mbmu2pi_kin.eps}
  \includegraphics[width=0.45\columnwidth]{figures/slb/mbvcb_kin.eps}
\end{center}
\caption{$\Delta\chi^2=1$~contours of the fit result in the kinetic mass
  scheme.} \label{fig:gf_res_kin}
\end{figure}

The fit using the $c$-quark mass constraint yields a $\bar B\to
X_c\ell^-\bar\nu_\ell$ branching fraction of
\begin{equation}
  \cbf(\bar B\to X_c\ell^-\bar\nu_\ell)=(10.51\pm 0.13)\%~.
\end{equation}
Correcting for charmless semileptonic decays
(Sect.~\ref{slbdecays_b2uincl}), $\cbf(\bar B\to
X_u\ell^-\bar\nu_\ell)=(2.08\pm 0.30)\times 10^{-3}$, we obtain the
semileptonic branching fraction,
\begin{equation}
  \cbf(\bar B\to X\ell^-\bar\nu_\ell)=(10.72\pm 0.13)\%~.
\end{equation}

\subsubsection{Analysis in the 1S scheme}
\label{globalfits1S}

The fit relies on the calculations of the spectral moments described in
Ref.~\cite{Bauer:2004ve}. The theoretical uncertainties are estimated
as explained in Ref.~\cite{Schwanda:2008kw}. Only trivial theory
correlations, {\it i.e.}, between the same moment at the same
threshold are included in the analysis. The fit determines $\vcb$ and
the 6 non-perturbative parameters mentioned above.

The result of the fit using the $B\to X_s\gamma$ constraint is
\begin{eqnarray}
  \vcb & = & (41.96\pm 0.45)\times 10^{-3}~, \\
  m_b^{1S} & = & 4.691\pm 0.037~{\rm GeV}~, \\
  \lambda_1 & = & -0.362\pm 0.067~{\rm GeV^2}~,
\end{eqnarray}
with a $\chi^2$ of 23.0 for $66-7$ degrees of freedom. The detailed
result of the fit is given in Table~\ref{tab:gf_res_xsgamma_1s}. This
result is consistent with the fit using the $\bar B\to
X_c\ell^-\bar\nu_\ell$~data only (Table~\ref{tab:gf_res_1s}).
\begin{table}[!htb]
\caption{Fit result in the 1S scheme, using $B\to X_s\gamma$~moments
  as a constraint. In the lower part of the table, the correlation
  matrix of the parameters is given.} \label{tab:gf_res_xsgamma_1s}
\begin{center}
\begin{tabular}{|l|ccccccc|}
  \hline
  & $m_b^{1S}$ [GeV] & $\lambda_1$ [GeV$^2$] & $\rho_1$ [GeV$^3$] &
  $\tau_1$ [GeV$^3$] & $\tau_2$ [GeV$^3$] & $\tau_3$ [GeV$^3$] &
  $\vcb$ [10$^{-3}$]\\
  \hline \hline
  value & 4.691 & $-0.362$ & \phantom{$-$}0.043 &
  \phantom{$-$}0.161 & $-0.017$ & \phantom{$-$}0.213 &
  \phantom{$-$}41.96\\
  error & 0.037 & \phantom{$-$}0.067 & \phantom{$-$}0.048 &
  \phantom{$-$}0.122 & \phantom{$-$}0.062 & \phantom{$-$}0.102 &
  \phantom{$-$}0.45\\
  \hline
  $m_b^{1S}$ & 1.000 & \phantom{$-$}0.434 & \phantom{$-$}0.213 &
  $-0.058$ & $-0.629$ & $-0.019$ & $-0.215$\\
  $\lambda_1$ & & \phantom{$-$}1.000 & $-0.467$ & $-0.602$ & $-0.239$
  & $-0.547$ & $-0.403$\\
  $\rho_1$ & & & \phantom{$-$}1.000 & \phantom{$-$}0.129 & $-0.624$ &
  \phantom{$-$}0.494 & \phantom{$-$}0.286\\
  $\tau_1$ & & & & \phantom{$-$}1.000 & \phantom{$-$}0.062 & $-0.148$ &
  \phantom{$-$}0.194\\
  $\tau_2$ & & & & & \phantom{$-$}1.000 & $-0.009$ & $-0.145$\\
  $\tau_3$ & & & & & & \phantom{$-$}1.000 & \phantom{$-$}0.376\\
  $\vcb$ & & & & & & & \phantom{$-$}1.000\\
  \hline
\end{tabular}
\end{center}
\end{table}
\begin{table}[!htb]
\caption{Fit result in the 1S scheme for different data
  sets.} \label{tab:gf_res_1s}
\begin{center}
\begin{tabular}{|c|c|c|c|c|}
  \hline
  Data & $\vcb$ [10$^{-3}$] & $m_b^{1S}$ [GeV] &
  $\lambda_1$ [GeV$^2$] & $\chi^2/$d.o.f.\\
  \hline \hline
  $X_c\ell\nu$ and $X_s\gamma$ & $41.96\pm 0.45$ & $4.691\pm 0.037$ &
  $-0.362\pm 0.067$ & $23.0/(66-7)$\\
  $X_c\ell\nu$ only & $42.37\pm 0.65$ & $4.622\pm 0.085$ & $-0.412\pm
  0.084$ & $13.7/(55-7)$\\
  \hline
\end{tabular}
\end{center}
\end{table}
