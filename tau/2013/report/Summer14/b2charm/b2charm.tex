\section{$b$-Hadron Decays to Charmed Hadrons}
\label{sec:b2c}
Ground state $B$ mesons and $b$-baryons dominantly decay to particles containing a charm quark via the $b \rightarrow c$ quark transition.
Therefore these decays are sensitive to the $|V_{cb}|$ CKM matrix element.
Usually semileptonic modes are used for $|V_{cb}|$ measurements which are discussed in Section~\ref{sec:slbdecays}.
The hadronic decay modes presented in this section are affected by larger theoretical uncertainties which makes them less useful for the extraction of fundamental parameters.
Of large direct interest are only a few hadronic $B$ to charm decay measurements, like the branching ratio of $B_s^0 \rightarrow D_s^{(*)+} D_s^{(*)-}$ which can be used to determine the decay width difference in the $B_s^0$ system.

The fact that decays to charmed hadrons are the dominant $b$-hadron decays makes them a very important part of experimental measurements.
They are often used as normalization mode for measurements of rarer decays.
In addition, they are the dominant background in many analyses.
To accurately model the background with simulated data it is essential to have a precise knowledge of the contributing decay modes.
In particular, with the expected increase in the data samples at LHCb and Belle II, the enhanced statistical sensitivity has to be matched by a low systematic uncertainty of the dominant $b$-hadron decay modes.

Compared to the previous version of $B$ to charm results the averaging procedure was updated to follow the methodology described in Section~\ref{sec:method}.
% The COMBOS~\cite{Combos:1999} program is used to calculate the averages.
Where available, correlations between measurements and dependencies on input parameters are taken into account.

The results are presented in subsections organized by the mother bottom hadron of the decay: $\bar{B}^0$, $B^-$, $\bar{B}^0/B^-$ combined, $\bar{B}_s^0$, $B_c^-$, $b$ baryons.
For each subsection the measurements are arranged, considering the decay mode, into the following groups: a single charmed meson, two charmed mesons, a charmonium state, a charm baryon, or other states, like for example the $X(3872)$.
Charge conjugate modes are always included.

\clearpage

\newcommand{\btocharmplot}[1]{\begin{center}\includegraphics[width=0.99\textwidth]{#1}\end{center}}
\newenvironment{btocharmtab}[1]{\begin{table}[H]\begin{center}\caption{#1}\begin{tabular}{| l l l |}}{\end{tabular}\end{center}\end{table}}
\newcommand{\btocharm}[1]{\btocharmplot{b2charm/#1}\input{b2charm/#1}}

% unccmment the following line to disable plots
%\renewcommand{\btocharmplot}[1]{}

% uncomment the following line to use longtables
%\renewenvironment{btocharmtab}[1]{\begin{center}\begin{longtable}{| l l l |}\caption{#1}\endfirsthead\multicolumn{3}{c}{continued from previous page}\endhead\endfoot\endlastfoot}{\end{longtable}\end{center}}


\subsection{$\bar{B}^0$}

\subsubsection{Decays to a single open charm meson}
\btocharm{Bd_D_1}
\btocharm{Bd_D_2}
\btocharm{Bd_D_3}
\btocharm{Bd_D_4}
\btocharm{Bd_D_5}
\btocharm{Bd_D_6}
\btocharm{Bd_D_7}
\btocharm{Bd_D_8}
\btocharm{Bd_D_9}
\btocharm{Bd_D_10}

\subsubsection{Decays to two open charm mesons}
\btocharm{Bd_DD_1}
\btocharm{Bd_DD_2}
\btocharm{Bd_DD_3}
\btocharm{Bd_DD_4}
\btocharm{Bd_DD_5}
\btocharm{Bd_DD_6}
\btocharm{Bd_DD_7}

\subsubsection{Decays to charmonium states}
\btocharm{Bd_cc_1}
\btocharm{Bd_cc_2}
\btocharm{Bd_cc_3}
\btocharm{Bd_cc_4}
\btocharm{Bd_cc_5}
\btocharm{Bd_cc_6}

\subsubsection{Decays to charm baryons}
\btocharm{Bd_baryon_1}
\btocharm{Bd_baryon_2}

\subsubsection{Decays to other (XYZ) states}
\btocharm{Bd_other_1}
\btocharm{Bd_other_2}
\btocharm{Bd_other_3}
\btocharm{Bd_other_4}
\btocharm{Bd_other_5}


\subsection{$B^-$}

\subsubsection{Decays to a single open charm meson}
\btocharm{Bu_D_1}
\btocharm{Bu_D_2}
\btocharm{Bu_D_3}
\btocharm{Bu_D_4}
\btocharm{Bu_D_5}
\btocharm{Bu_D_6}
\btocharm{Bu_D_7}
\btocharm{Bu_D_8}
\btocharm{Bu_D_9}
\btocharm{Bu_D_10}
\btocharm{Bu_D_11}

\subsubsection{Decays to two open charm mesons}
\btocharm{Bu_DD_1}
\btocharm{Bu_DD_2}
\btocharm{Bu_DD_3}
\btocharm{Bu_DD_4}
\btocharm{Bu_DD_5}
\btocharm{Bu_DD_6}

\subsubsection{Decays to charmonium states}
\btocharm{Bu_cc_1}
\btocharm{Bu_cc_2}
\btocharm{Bu_cc_3}
\btocharm{Bu_cc_4}
\btocharm{Bu_cc_5}
\btocharm{Bu_cc_6}
\btocharm{Bu_cc_7}

\subsubsection{Decays to charm baryons}
\btocharm{Bu_baryon_1}
\btocharm{Bu_baryon_2}

\subsubsection{Decays to other (XYZ) states}
\btocharm{Bu_other_1}
\btocharm{Bu_other_2}
\btocharm{Bu_other_3}
\btocharm{Bu_other_4}
\btocharm{Bu_other_5}


\subsection{$\bar{B}^0$ / $B^-$}

\subsubsection{Decays to two open charm mesons}
\btocharm{B_DD_1}

\subsubsection{Decays to charmonium states}
\btocharm{B_cc_1}
\btocharm{B_cc_2}
\btocharm{B_cc_3}
\btocharm{B_cc_4}
\btocharm{B_cc_5}

\subsubsection{Decays to other (XYZ) states}
\btocharm{B_other_1}


\subsection{$\bar{B}_s^0$}

\subsubsection{Decays to a single open charm meson}
\btocharm{Bs_D_1}
\btocharm{Bs_D_2}
\btocharm{Bs_D_3}
\btocharm{Bs_D_4}

\subsubsection{Decays to two open charm mesons}
\btocharm{Bs_DD_1}
\btocharm{Bs_DD_2}

\subsubsection{Decays to charmonium states}
\btocharm{Bs_cc_1}
\btocharm{Bs_cc_2}
\btocharm{Bs_cc_3}
\btocharm{Bs_cc_4}

\subsubsection{Decays to charm baryons}
\btocharm{Bs_baryon_1}
\btocharm{Bs_baryon_2}


\subsection{$B_c^-$}

\subsubsection{Decays to charmonium states}
\btocharm{Bc_cc_1}
\btocharm{Bc_cc_2}

\subsubsection{Decays to a $B$ meson}
\btocharm{Bc_B_1}


\subsection{b baryons}

\subsubsection{Decays to a single open charm meson}
\btocharm{Bbaryon_D_1}

\subsubsection{Decays to charmonium states}
\btocharm{Bbaryon_cc_1}
\btocharm{Bbaryon_cc_2}
\btocharm{Bbaryon_cc_3}

\subsubsection{Decays to charm baryons}
\btocharm{Bbaryon_baryon_1}
\btocharm{Bbaryon_baryon_2}
\btocharm{Bbaryon_baryon_3}
