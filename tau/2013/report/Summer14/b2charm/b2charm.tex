\section{$b$-hadron decays to charmed hadrons}
\label{sec:b2c}
Ground state $B$ mesons and $b$-baryons dominantly decay to particles containing a charm quark via the $b \rightarrow c$ quark transition.
Therefore these decays are sensitive to the $|V_{cb}|$ CKM matrix element.
Usually semileptonic modes are used for $|V_{cb}|$ measurements which are discussed in Section~\ref{sec:slbdecays}.
Some $B$ meson decays to open charmed hadrons are fundamental decays for the measurements of $CP$-violation phases like $\phi_s^{c\bar{c}s}$ (Section~\ref{sec:life_mix}), $\beta=\phi_1$ and $\gamma=\phi_3$ (Section~\ref{sec:cp_uta}). 
They are also used to search for exotic particles, like the X(3872)~\cite{Choi:2003ue}.

The fact that decays to charmed hadrons are the dominant $b$-hadron decays makes them a very important part of experimental measurements.
They are often used as normalization mode for measurements of rarer decays.
In addition, they are the dominant background in many analyses.
To accurately model the background with simulated data it is essential to have a precise knowledge of the contributing decay modes.
In particular, with the expected increase in the data samples at LHCb and Belle II, the enhanced statistical sensitivity has to be matched by a low systematic uncertainty of the dominant $b$-hadron decay modes.

In this section, we give the exhaustive list of branching ratios of decay modes to charmed hadrons.
Compared to the previous version of $B$ to charm results the averaging procedure was updated to follow the methodology described in Section~\ref{sec:method}.
% The COMBOS~\cite{Combos:1999} program is used to calculate the averages.
Where available, correlations between measurements 
%and dependencies on input parameters 
are taken into account.

The results are presented in subsections organized by the mother bottom hadron of the decay: $\bar{B}^0$, $B^-$, $\bar{B}^0/B^-$ combined, $\bar{B}_s^0$, $B_c^-$, $b$ baryons.
For each subsection the measurements are arranged, considering the decay mode, into the following groups: a single charmed meson, two charmed mesons, a charmonium state, a charm baryon, or other states, like for example the $X(3872)$.
The individual measurements and averages are shown as numerical values in tables followed by a graphical representation of the averages.
The symbol $\mathcal{B}$ is used for branching ratios, $f$ for production fractions (see Section~\ref{sec:life_mix}), and $\sigma$ for cross sections.
The decay amplitudes for longitudinal, parallel, and perpendicular transverse polarization in pseudoscalar to vector-vector decays are denoted ${\cal{A}}_0$, ${\cal{A}}_\parallel$, and ${\cal{A}}_\perp$, respectively, and the definitions $\delta_\parallel = \arg({\cal{A}}_\parallel/{\cal{A}}_0)$ and $\delta_\perp = \arg({\cal{A}}_\perp/{\cal{A}}_0)$ are used for their relative phases.
Charge conjugate modes are always included.

\clearpage

% \input{b2charm/b2charm_captions}

\newcommand{\btocharmplot}[1]{\begin{center}\includegraphics[width=0.99\textwidth]{#1}\end{center}}
\newenvironment{btocharmtab}[1]{\begin{table}[H]\begin{center}\caption{#1}\begin{tabular}{| l l l |}}{\end{tabular}\end{center}\end{table}}
\newcommand{\btocharm}[1]{\input{b2charm/#1}\btocharmplot{b2charm/#1}}
\renewcommand{\tablename}{Table+Figure}

% unccmment the following line to disable plots
%\renewcommand{\btocharmplot}[1]{}

% uncomment the following line to use longtables
%\renewenvironment{btocharmtab}[1]{\begin{center}\begin{longtable}{| l l l |}\caption{#1}\endfirsthead\multicolumn{3}{c}{continued from previous page}\endhead\endfoot\endlastfoot}{\end{longtable}\end{center}}


\subsection{Decays of $\bar{B}^0$ mesons}

\subsubsection{Decays to a single open charm meson}
\btocharm{Bd_D_1}
\btocharm{Bd_D_2}
\btocharm{Bd_D_3}
\btocharm{Bd_D_4}
\btocharm{Bd_D_5}
\btocharm{Bd_D_6}
\btocharm{Bd_D_7}
\btocharm{Bd_D_8}
\btocharm{Bd_D_9}
\btocharm{Bd_D_10}

\subsubsection{Decays to two open charm mesons}
\btocharm{Bd_DD_1}
\btocharm{Bd_DD_2}
\btocharm{Bd_DD_3}
\btocharm{Bd_DD_4}
\btocharm{Bd_DD_5}
\btocharm{Bd_DD_6}
\btocharm{Bd_DD_7}

\subsubsection{Decays to charmonium states}
\btocharm{Bd_cc_1}
\btocharm{Bd_cc_2}
\btocharm{Bd_cc_3}
\btocharm{Bd_cc_4}
\btocharm{Bd_cc_5}
\btocharm{Bd_cc_6}

\subsubsection{Decays to charm baryons}
\btocharm{Bd_baryon_1}
\btocharm{Bd_baryon_2}

\subsubsection{Decays to other ($XYZ$) states}
\btocharm{Bd_other_1}
\btocharm{Bd_other_2}
\btocharm{Bd_other_3}
\btocharm{Bd_other_4}
\btocharm{Bd_other_5}


\subsection{Decays of $B^-$ mesons}

\subsubsection{Decays to a single open charm meson}
\btocharm{Bu_D_1}
\btocharm{Bu_D_2}
\btocharm{Bu_D_3}
\btocharm{Bu_D_4}
\btocharm{Bu_D_5}
\btocharm{Bu_D_6}
\btocharm{Bu_D_7}
\btocharm{Bu_D_8}
\btocharm{Bu_D_9}
\btocharm{Bu_D_10}
\btocharm{Bu_D_11}

\subsubsection{Decays to two open charm mesons}
\btocharm{Bu_DD_1}
\btocharm{Bu_DD_2}
\btocharm{Bu_DD_3}
\btocharm{Bu_DD_4}
\btocharm{Bu_DD_5}
\btocharm{Bu_DD_6}

\subsubsection{Decays to charmonium states}
\btocharm{Bu_cc_1}
\btocharm{Bu_cc_2}
\btocharm{Bu_cc_3}
\btocharm{Bu_cc_4}
\btocharm{Bu_cc_5}
\btocharm{Bu_cc_6}
%\btocharm{Bu_cc_7}

\subsubsection{Decays to charm baryons}
\btocharm{Bu_baryon_1}
\btocharm{Bu_baryon_2}

\subsubsection{Decays to other ($XYZ$) states}
\btocharm{Bu_other_1}
\btocharm{Bu_other_2}
\btocharm{Bu_other_3}
\btocharm{Bu_other_4}
\btocharm{Bu_other_5}


\subsection{Decays of admixtures of $\bar{B}^0$ / $B^-$ mesons}

\subsubsection{Decays to two open charm mesons}
\btocharm{B_DD_1}

\subsubsection{Decays to charmonium states}
\btocharm{B_cc_1}
\btocharm{B_cc_2}
\btocharm{B_cc_3}
\btocharm{B_cc_4}
\btocharm{B_cc_5}

\subsubsection{Decays to other ($XYZ$) states}
\btocharm{B_other_1}


\subsection{Decays of $\bar{B}_s^0$ mesons}

\subsubsection{Decays to a single open charm meson}
\btocharm{Bs_D_1}
\btocharm{Bs_D_2}
\btocharm{Bs_D_3}
\btocharm{Bs_D_4}

\subsubsection{Decays to two open charm mesons}
\btocharm{Bs_DD_1}
\btocharm{Bs_DD_2}

\subsubsection{Decays to charmonium states}
\btocharm{Bs_cc_1}
\btocharm{Bs_cc_2}
\btocharm{Bs_cc_3}
\btocharm{Bs_cc_4}

\subsubsection{Decays to charm baryons}
\btocharm{Bs_baryon_1}
\btocharm{Bs_baryon_2}


\subsection{Decays of $B_c^-$ mesons}

\subsubsection{Decays to charmonium states}
\btocharm{Bc_cc_1}
\btocharm{Bc_cc_2}

\subsubsection{Decays to a $B$ meson}
\btocharm{Bc_B_1}


\subsection{Decays of $b$ baryons}

\subsubsection{Decays to a single open charm meson}
\btocharm{Bbaryon_D_1}

\subsubsection{Decays to charmonium states}
\btocharm{Bbaryon_cc_1}
\btocharm{Bbaryon_cc_2}
\btocharm{Bbaryon_cc_3}

\subsubsection{Decays to charm baryons}
\btocharm{Bbaryon_baryon_1}
\btocharm{Bbaryon_baryon_2}
\btocharm{Bbaryon_baryon_3}


\renewcommand{\tablename}{Table}
