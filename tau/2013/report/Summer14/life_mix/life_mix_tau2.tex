%%%%%%%%%%%%%%%%%%%%%%%%%%%%%%%%%%%%%%%%%%%%%%%%
%
% This is file life_mix_tau2.tex containing
% the second part of the chapter on the b-hadron lifetimes: 
% Bs, Bc, lambda_b and b-baryon lifetimes
% as well as theroretical predictions for all b-hadron lifetimes.
%
%%%%%%%%%%%%%%%%%%%%%%%%%%%%%%%%%%%%%%%%%%%%%%%
%

\mysubsubsection{\Bs lifetimes}
\labs{taubs}

Like neutral kaons, neutral \B mesons contain
short- and long-lived components, since the
light (L) and heavy (H)
eigenstates, $\B_{\rm L}$ and $\B_{\rm H}$, differ not only
in their masses, but also in their total decay widths,  
with a decay width difference defined as 
$\DG = \Gamma_{\rm L} - \Gamma_{\rm H}$. 
Neglecting \CP violation in $\B-\Bbar$ mixing, 
which is expected to be very
small~\cite{Lenz:2011ti,*Lenz:2006hd,Beneke:1998sy} (see also Sec.~\ref{qpd}),
the mass eigenstates are also \CP eigenstates,
with the light $\B_{\rm L}$ state being \CP-even 
and the heavy $\B_{\rm H}$ state being \CP-odd. 
% Final states can be decomposed into
% \CP-even and \CP-odd components, each with a different
% lifetime.
While the decay width difference \DGd can be neglected in the \Bd system, 
the \Bs system exhibits a significant
value of \DGs: the sign of \DGs is known 
to be positive~\cite{Aaij:2012eq}, {\em i.e.}
the heavy eigenstate lives longer than the light eigenstate. 
Specific measurements of \DGs and 
$\Gs = (\Gamma_{\rm L} + \Gamma_{\rm H})/2$ are explained
and averaged in \Sec{DGs}, but the results for
$1/\Gamma_{\rm L}$, $1/\Gamma_{\rm H}$ and
the mean \Bs lifetime, defined as $\tau(\Bs) = 1/\Gs$,
are also quoted at the end of this section. 

Many \Bs lifetime analyses, in particular the early 
ones performed before the non-zero value of \DGs was 
firmly established, ignore \DGs and fit the proper time 
distribution of a sample of \Bs candidates 
reconstructed in a certain final state $f$
with a model assuming a single exponential function 
for the signal. We denote such {\rm effective lifetime}
measurements~\cite{Fleischer:2011cw} as $\tau_{\rm single}(\Bs\to f)$; 
their true values may lie {\em a priori} anywhere
between $1/\Gamma_{\rm L} = 1/(\Gs+\DGs/2)$ and
$1/\Gamma_{\rm H}= 1/(\Gs-\DGs/2)$, 
depending on the proportion of $\B_{\rm L}$ and $\B_{\rm H}$
in the final state $f$. 
% TJG
More recent determinations of effective lifetimes may be interpreted as
measurements of the relative composition of $\B_{\rm L}$ and $\B_{\rm H}$
decaying to the final state $f$. 
% TJG
\Table{lifebs} summarizes the effective 
lifetime measurements.

Averaging measurements of $\tau_{\rm single}(\Bs\to f)$
over several final states $f$ will yield a result 
corresponding to an ill-defined observable
when the proportions of $\B_{\rm L}$ and $\B_{\rm H}$ differ. 
Therefore,
the effective \Bs lifetime measurements are broken down into
several categories and averaged separately.

\begin{table}[t]
\caption{Measurements of the effective \Bs lifetimes obtained from single exponential fits.}
% without attempting to separate the short and long components.} % \CP-even and \CP-odd 
\labt{lifebs}
\begin{center}
\resizebox{\textwidth}{!}{
\begin{tabular}{lc@{}cc@{}rcl} \hline
Experiment & \multicolumn{2}{c}{Final state $f$}           & \multicolumn{2}{c}{Data set} & $\tau_{\rm single}(\Bs\to f)$ (ps) & Ref. \\
\hline \hline
ALEPH  & \particle{D_s^- \ell^+}  & flavour-specific & 91--95 & & $1.54^{+0.14}_{-0.13}\pm 0.04$   & \cite{Buskulic:1996ei}          \\
CDF1   & \particle{D_s^- \ell^+}  & flavour-specific & 92--96 & & $1.36\pm 0.09 ^{+0.06}_{-0.05}$  & \cite{Abe:1998cj}           \\
DELPHI & \particle{D_s^- \ell^+}  & flavour-specific & 91--95 & & $1.42^{+0.14}_{-0.13}\pm 0.03$   & \cite{Abreu:2000sh}          \\
OPAL   & \particle{D_s^- \ell^+}  & flavour-specific & 90--95 & & $1.50^{+0.16}_{-0.15}\pm 0.04$   & \cite{Ackerstaff:1997qi}  \\
%% superseded by line below: \dzero & \particle{D_s^- \mu^+}   & flavour-specific & 02--04 & 0.4 fb$^{-1}$ & $1.398 \pm 0.044 ^{+0.028}_{-0.025}   $   & \cite{Abazov:2006cb}       \\ 
\dzero & \particle{D_s^-\mu^+X}   & flavour-specific & Run II & 10.4 fb$^{-1}$ & $1.479 \pm 0.010 \pm 0.021$   & \cite{Abazov:2014rua,*Abazov:2006cb_cont} \\
%%% CDF2   & \particle{D_s^- \ell^+}  & flavour-specific & 02--04 & & $1.381 \pm 0.055 ^{+0.052}_{-0.046} $ & \cite{CDFnote7757:2005}$^p$ \\
%%%%% CDF2   & \particle{D_s^- \pi^+, D_s^- \pi^+ \pi^- \pi^+} 
CDF2   & \particle{D_s^- \pi^+ (X)} 
                              & flavour-specific & 02--06 & 1.3 fb$^{-1}$ & $1.518 \pm 0.041 \pm 0.027     $   & \cite{Aaltonen:2011qsa,*Aaltonen:2011qsa_cont} \\ %was \cite{CDFnote9203:2008,*CDFnote9203:2008_cont}$^p$      \\
LHCb   &  \particle{D_s^- D^+} & flavour-specific & 11--12 & 3 fb$^{-1}$ & $1.52 \pm 0.15 \pm 0.01$ & \cite{Aaij:2013bvd} \\
LHCb   &  \particle{D_s^- \pi^+} & flavour-specific & 11 & 1 fb$^{-1}$ & $1.535 \pm 0.015 \pm 0.014$ & \cite{Aaij:2014sua} \\
\multicolumn{5}{l}{Average of above 8 flavour-specific measurements} &  \hfagTAUBSSLnounit & \\  
\hline\hline
LHCb    & \particle{\pi^+K^-}   &  $\sim$ flavour-specific ? & 11 & 1.0 fb$^{-1}$ & $1.60 \pm 0.06 \pm 0.01$ & \cite{Aaij:2014fia,*Aaij:2012ns_cont} \\
\hline
ALEPH  & \particle{D_s h}     & ill-defined & 91--95 & & $1.47\pm 0.14\pm 0.08$           & \cite{Barate:1997ua}          \\
DELPHI & \particle{D_s h}     & ill-defined & 91--95 & & $1.53^{+0.16}_{-0.15}\pm 0.07$   & \cite{Abreu:2000ev} \\
%%OS 23apr2005: this is superseded by \cite{Abreu:2000ev} %% DELPHI & \particle{D_s} incl. & mixture & 91--94 & $1.60\pm 0.26^{+0.13}_{-0.15}$   & \cite{DELBS2}          \\
OPAL   & \particle{D_s} incl. & ill-defined & 90--95 & & $1.72^{+0.20+0.18}_{-0.19-0.17}$ & \cite{Ackerstaff:1997ne}          \\ 
%% ALEPH    & \particle{D_s^{(*)+}D_s^{(*)-}} & \CP-even ? & 91--95 & 4M \particle{Z\to q\bar{q}} & $1.27 \pm 0.33 \pm 0.08$ & \cite{Barate:2000kd} \\
\hline
CDF1     & \particle{\jpsi\phi} & \CP even+odd & 92--95 &  & $1.34^{+0.23}_{-0.19}    \pm 0.05$ & \cite{Abe:1997bd} \\
%%% CDF2     & \particle{\jpsi\phi} & \CP even+odd & 02--06 &  & $1.494 \pm 0.054 \pm 0.009$ & \cite{CDFnote8524:2007,*CDFnote8524:2007_cont}$^p$ \\
\dzero   & \particle{\jpsi\phi} & \CP even+odd & 02--04 &  & $1.444^{+0.098}_{-0.090} \pm 0.02$ & \cite{Abazov:2004ce}  \\
ATLAS & \particle{\jpsi\phi} & \CP even+odd & 10 & 40 pb$^{-1}$ & $1.41 \pm0.08 \pm0.05$ & \cite{ATLAS-CONF-2011-092}$^p$ \\
%%% LHCb  & \particle{\jpsi\phi} & \CP even+odd & 10 & 36 pb$^{-1}$ & $1.447 \pm0.064 \pm 0.056$ & \cite{LHCb-CONF-2011-001}$^p$ \\
LHCb  & \particle{\jpsi\phi} & \CP even+odd & 11 & 1 fb$^{-1}$ & $1.480 \pm0.011 \pm 0.005$ & \cite{Aaij:2014owa} \\
\multicolumn{5}{l}{Average of above 4 \particle{\jpsi \phi} measurements} &  \hfagTAUBSJFnounit & \\ 
\hline\hline
ALEPH    & \particle{D_s^{(*)+}D_s^{(*)-}} & mostly \CP even & 91--95 & & $1.27 \pm 0.33 \pm 0.08$ & \cite{Barate:2000kd} \\
%%% CDF2 measurement below removed from the average because it remained unpublished for two long
%%% CDF2     & \particle{K^+K^-}   & \CP-even & 02--04 & 0.36 fb$^{-1}$ & $1.53 \pm 0.18 \pm 0.02$ & \cite{Tonelli:2006np}$^p$ \\
LHCb    & \particle{K^+K^-}   &  $\sim$ \CP-even & 10 & 0.037 fb$^{-1}$ & $1.440 \pm 0.096 \pm 0.009$ & \cite{Aaij:2011kn} \\
%%% superseded by next line LHCb    & \particle{K^+K^-}   &  \CP-even & 11 & 1.0 fb$^{-1}$ & $1.455 \pm 0.046 \pm 0.006$ & \cite{Aaij:2012ns} \\
LHCb    & \particle{K^+K^-}   &  $\sim$ \CP-even & 11 & 1.0 fb$^{-1}$ & $1.407 \pm 0.016 \pm 0.007$ & \cite{Aaij:2014fia,*Aaij:2012ns_cont} \\
\multicolumn{5}{l}{Average of above 2 \particle{K^+K^-} measurements} &  \hfagTAUBSKKnounit & \\ 
LHCb   &  \particle{D_s^+ D_s^-} & \CP-even & 11--12 & 3 fb$^{-1}$ & $1.379 \pm 0.026 \pm 0.017$ & \cite{Aaij:2013bvd} \\
\multicolumn{5}{l}{Average of above 1 \CP-even measurement} &  \hfagTAUBSSHORTnounit & \\ \hline \hline
LHCb     & \particle{\jpsi K^0_{\rm S}} & $\sim$ \CP-odd & 11   & 1.0 fb$^{-1}$ & $1.75 \pm 0.12 \pm 0.07$ & \cite{Aaij:2013eia} \\
CDF2     & \particle{\jpsi f_0(980)} & \CP-odd & 02--08 & 3.8 fb$^{-1}$ & $1.70^{+0.12}_{-0.11} \pm 0.03$ & \cite{Aaltonen:2011nk} \\
%%LHCb     & \particle{\jpsi f_0(980)} & \CP-odd & 11   & 1.0 fb$^{-1}$ & $1.700 \pm 0.040 \pm 0.026$ & \cite{Aaij:2012nta} \\
LHCb     & \particle{\jpsi \pi^+\pi^-} & \CP-odd & 11   & 1.0 fb$^{-1}$ & $1.652 \pm 0.024 \pm 0.024$ & \cite{Aaij:2013oba,*LHCb:2011aa_mod,*LHCb:2012ad_mod,*LHCb:2011ab_mod,*Aaij:2012nta_mod} \\
% \multicolumn{5}{l}{Average of above 2 \particle{\jpsi f_0(980)}, \particle{\jpsi \pi^+\pi^-} measurements} &  \hfagTAUBSJPSIPIPInounit & \\ \hline 
\multicolumn{5}{l}{Average of above 2 \CP-odd measurements} &  \hfagTAUBSLONGnounit & \\ \hline \hline
\multicolumn{5}{l}{$^p$ \footnotesize Preliminary.}
\end{tabular}
}
\end{center}
\end{table}

\afterpage{\clearpage}

\begin{itemize}
\item 
{\bf\em Flavour-specific decays},
such as $\Bs \to \particle{D_s^- \ell^+ \nu}$
or $\Bs\to \particle {D_s^- \pi^+}$, have equal 
fractions of $\B_{\rm L}$ and $\B_{\rm H}$ at time zero. 
% , where $\tau_{\rm L} = 1/\Gamma_{\rm L}$ 
% is expected to be the shorter-lived component and
% $\tau_{\rm H} = 1/\Gamma_{\rm H}$ 
% expected to be the longer-lived component. 
If the resulting superposition of two exponential distributions
is fitted with a single exponential function, 
one obtains a measure of the so-called {\em flavour-specific lifetime}~\cite{Hartkorn:1999ga}:
% A superposition of two exponentials thus results with decay
% widths $\Gs \pm \DGs /2$.  Fitting to a single exponential one obtains a
% measure of the flavour-specific lifetime~\cite{Hartkorn:1999ga}:
\begin{equation}
\tau_{\rm single}(\Bs\to \mbox{flavour specific}) = \frac{1}{\Gs}
\frac{{1+\left(\frac{\DGs}{2\Gs}\right)^2}}{{1-\left(\frac{\DGs}{2\Gs}\right)^2}
}.
\labe{fslife}
\end{equation}
The average of all flavour-specific 
\Bs lifetime measurements\footnote{
An old unpublished measurement~\cite{CDFnote7757:2005} is not included.}
is
\begin{equation}
\tau_{\rm single}(\Bs\to \mbox{flavour specific}) = \hfagTAUBSSL \,.
\labe{tau_fs}
\end{equation}
% is used in \Sec{DGs} as one of the ingredients 
% to determine $\tau(\Bs) = 1/\Gs$ and \DGs.

\item
{\bf\em \boldmath $\Bs\to D_s^{\mp} X$ decays}
include flavour-specific decays but also decays 
with an unknown mixture of light and heavy components. 
Measurements performed with such inclusive states are
no longer used in averages. 
%OLD% The corresponding effective lifetime average,
%OLD% \begin{equation}
%OLD% \tau_{\rm single}(\Bs\to D_s^{\mp} X) = \hfagTAUBSwaschanged \,,
%OLD% \end{equation}
%OLD% can still be a useful input
%OLD% for analyses examining an inclusive $D_s$ sample.
%OLD% The following correlated systematic errors were considered:
%OLD% average \B lifetime used in backgrounds,
%OLD% \Bs decay multiplicity, and branching ratios used to determine 
%OLD% backgrounds (\eg\ \BR{B\to D_s D}).
%OLD% A knowledge of the multiplicity of \Bs decays is important for
%OLD% measurements that partially reconstruct the final state such as 
%OLD% \particle{\B\to D_s \mbox{$X$}} (where $X$ is not a lepton). 
%OLD% The boost deduced from Monte Carlo simulation depends on the multiplicity used.
%OLD% Since this is not well known, the multiplicity in the simulation is
%OLD% varied and this range of values observed is taken to be a systematic.
%OLD% Similarly not all the branching ratios for the potential background
%OLD% processes are measured. Where they are available, the PDG values are
%OLD% used for the error estimate. Where no measurements are available
%OLD% estimates can usually be made by using measured branching ratios of
%OLD% related processes and using some reasonable extrapolation.

\item
{\bf\em 
{\boldmath $\Bs \to \jpsi\phi$ \unboldmath}decays}
contain a well-measured mixture of \CP-even and \CP-odd states.
% are expected to be
% dominated by the \CP-even state and its lifetime.
There are no known correlations
between the existing 
\particle{\Bs\to \jpsi\phi}
effective lifetime measurements; these are combined  
into the average\footnote{
An old unpublished measurement~\cite{CDFnote8524:2007,*CDFnote8524:2007_cont} is not included.}
% \begin{equation}
$\tau_{\rm single}(\Bs\to \jpsi \phi) = \hfagTAUBSJF$. % \,.
% \end{equation}
A caveat is that different experimental acceptances
may lead to different admixtures of the 
\CP-even and \CP-odd states, and simple fits to a single
exponential may result in inherently different 
values of $\tau_{\rm single}(\Bs\to \jpsi \phi)$.
Analyses that separate the \CP-even and \CP-odd components in
this decay through a full angular study, outlined in \Sec{DGs},
provide directly precise measurements of $1/\Gs$ and $\DGs$ (see \Table{phisDGsGs}).

\item
{\bf\em Decays to \boldmath\CP eigenstates} have also 
been measured, in the \CP-even modes 
$\Bs \to D_s^{(*)+}D_s^{(*)-}$ by ALEPH~\cite{Barate:2000kd},
$\Bs \to K^+ K^-$ by LHCb~\cite{Aaij:2011kn,Aaij:2014fia,*Aaij:2012ns_cont}%
\footnote{An old unpublished measurement of the $\Bs \to K^+ K^-$
effective lifetime by CDF~\cite{Tonelli:2006np} is no longer considered.}
and $\Bs \to D_s^+D_s^-$ by LHCb~\cite{Aaij:2013bvd}, as well as in the \CP-odd modes 
$\Bs \to \jpsi f_0(980)$ by CDF~\cite{Aaltonen:2011nk}, 
$\Bs \to \jpsi \pi^+\pi^-$ by LHCb~\cite{Aaij:2013oba,*LHCb:2011aa_mod,*LHCb:2012ad_mod,*LHCb:2011ab_mod,*Aaij:2012nta_mod}
and $\Bs \to \jpsi K^0_{\rm S}$ by LHCb~\cite{Aaij:2013eia}.
If these 
decays are dominated by a single weak phase and if \CP violation 
can be neglected, then $\tau_{\rm single}(\Bs \to \mbox{\CP-even}) \sim 1/\Gamma_{\rm L}$ 
and  $\tau_{\rm single}(\Bs \to \mbox{\CP-odd}) \sim 1/\Gamma_{\rm H}$ 
(see \Eqss{tau_KK_approx}{tau_Jpsif0_approx} for approximate relations in presence of
\CP violation in the mixing). 
However, not all these modes can be considered as pure \CP eigenstates; 
a small \CP-odd component is most probably present
in $\Bs \to D_s^{(*)+}D_s^{(*)-}$ decays, and the $\Bs \to K^+ K^-$
and $\Bs \to \jpsi K^0_{\rm S}$ may perhaps also be contaminated. 
The averages for the effective lifetimes obtained with pure \CP-even 
($D_s^+D_s^-$) and \CP-odd ($\jpsi f_0(980)$, $\jpsi \pi^+\pi^-$)
final states are
\begin{eqnarray}
\tau_{\rm single}(\Bs \to \mbox{\CP-even}) & = & \hfagTAUBSSHORT \,,
\labe{tau_KK}
\\
\tau_{\rm single}(\Bs \to \mbox{\CP-odd}) & = & \hfagTAUBSLONG \,.
\labe{tau_Jpsif0}
\end{eqnarray}
% A measurement of the effective lifetime of $\Bs \to D_s^{(*)+}D_s^{(*)-}$ decays by ALEPH~\cite{Barate:2000kd}
% is not included in the above \CP-even average, since a small \CP-odd component is most probably present. 

\end{itemize}

As described in \Sec{DGs}, 
the effective lifetime averages of \Eqsss{tau_fs}{tau_KK}{tau_Jpsif0}
are used as ingredients to improve the 
determination of $1/\Gs$ and \DGs obtained from the full angular analyses
of $\Bs\to \jpsi\phi$ and $\Bs\to \jpsi K^+K^-$ decays. 
The resulting world averages for the \Bs lifetimes are
\begin{eqnarray}
\tau(\B_{s\rm L}) =
\frac{1}{\Gamma_{\rm L}} = \frac{1}{\Gs+\DGs/2} & = & \hfagTAUBSLCON \,, \\
\tau(\B_{s\rm H}) =
\frac{1}{\Gamma_{\rm H}} = \frac{1}{\Gs-\DGs/2} & = & \hfagTAUBSHCON \,, \\
\tau(\Bs) = \frac{1}{\Gs} = \frac{2}{\Gamma_{\rm L}+\Gamma_{\rm H}} & = & \hfagTAUBSMEANCON \,.
\labe{oneoverGs}
\end{eqnarray}

\mysubsubsection{\Bc lifetime}
\labs{taubc}

Early measurements of the \Bc meson lifetime,
from CDF~\cite{Abe:1998wi,CDFnote9294:2008,Abulencia:2006zu} and \dzero~\cite{Abazov:2008rba},
use the semileptonic decay mode \particle{\Bc \to \jpsi \ell} and are based on a 
simultaneous fit to the mass and lifetime using the vertex formed
with the leptons from the decay of the \particle{\jpsi} and
the third lepton. Correction factors
to estimate the boost due to the missing neutrino are used.
In the analysis of the CDF Run~I data~\cite{Abe:1998wi},
a mass value of 
$6.40 \pm 0.39 \pm 0.13$~GeV/$c^2$ 
is found by fitting
to the tri-lepton invariant mass spectrum. 
%%% START WARNING
%%% Text below is valid when CDFnote9294:2008,*Abulencia:2006zu_mod_cont is the published result from 2006
In the CDF Run~II result~\cite{Abulencia:2006zu}, the mass is fixed
to 6.271~GeV/$c^2$, but then varied between 
6.2 and 6.4~GeV/$c^2$ to assess the systematic error on the
lifetime due to the \Bc mass value.
Finally, in the \dzero Run~II result~\cite{Abazov:2008rba}, 
%%% Text below valid when CDFnote9294:2008,*Abulencia:2006zu_mod_cont is the new CDF prel. result from CD note 9294
% In the CDF and \dzero Run~II results~\cite{CDFnote9294:2008,*Abulencia:2006zu_mod_cont,Abazov:2008rba}, 
%%% END WARNING
the \Bc mass is assumed to be 
$6285.7 \pm 5.3 \pm 1.2$~MeV/$c^2$, taken from a 
CDF result~\cite{Abulencia:2005usa}. 
These mass measurements
are consistent within uncertainties, and also consistent with the
most recent precision determination from CDF of 
$6275.6 \pm 2.9 \pm 2.5$~MeV/$c^2$~\cite{Aaltonen:2007gv}.
Correlated systematic errors include the impact
of the uncertainty of the \Bc $p_T$ spectrum on the correction
factors, the level of feed-down from $\psi(2S)$ decays, 
Monte-Carlo modeling of the decay model varying from phase space
to the ISGW model, and mass variations.

The latest determination of the \Bc lifetime from CDF2~\cite{Aaltonen:2012yb} is based on fully reconstructed 
$\Bc \to J/\psi \pi^+$ decays and does not suffer from a missing neutrino. 
All the measurements\footnote{We do not list (nor include in the average) an unpublished result from CDF2~\cite{CDFnote9294:2008}.}
are summarized in 
\Table{lifebc} and the world average, dominated by a recent LHCb measurement~\cite{Aaij:2014bva}, is 
determined to be
\begin{equation}
\tau(\Bc) = \hfagTAUBC \,.
\end{equation}

\begin{table}[tb]
\caption{Measurements of the \Bc lifetime.}
\labt{lifebc}
\begin{center}
\begin{tabular}{lccrcl} \hline
Experiment & Method                    & \multicolumn{2}{c}{Data set}  & $\tau(\Bc)$ (ps)
      & Ref.\\   \hline
CDF1       & \particle{\jpsi \ell} & 92--95 & 0.11 fb$^{-1}$ & $0.46^{+0.18}_{-0.16} \pm
 0.03$   & \cite{Abe:1998wi}  \\ 
CDF2       & \particle{\jpsi e} & 02--04 & 0.36 fb$^{-1}$ & $0.463^{+0.073}_{-0.065} \pm 0.036$   & \cite{Abulencia:2006zu} \\
%%unpublished%% CDF2       & \particle{\jpsi \ell} & 02--06 & 1.0 fb$^{-1}$ & $0.475^{+0.053}_{-0.049} \pm 0.018$   & \cite{CDFnote9294:2008,*Abulencia:2006zu_mod_cont}$^p$ \\
 \dzero & \particle{\jpsi \mu} & 02--06 & 1.3 fb$^{-1}$  & $0.448^{+0.038}_{-0.036} \pm 0.032$
   & \cite{Abazov:2008rba}  \\
CDF2       & \particle{\jpsi \pi} & & 6.7 fb$^{-1}$ & $0.452 \pm 0.048 \pm 0.027$  & \cite{Aaltonen:2012yb} \\
LHCb & \particle{\jpsi \mu} & 12 & 2 fb$^{-1}$  & $0.509 \pm 0.008 \pm 0.012$ & \cite{Aaij:2014bva}  \\
\hline
  \multicolumn{2}{l}{Average} & &  &  \hfagTAUBCnounit
                 &    \\   \hline
% \multicolumn{5}{l}{$^p$ \footnotesize Preliminary.}
\end{tabular}
\end{center}
\end{table}

\mysubsubsection{\Lb and \b-baryon lifetimes}
\labs{taulb}

The first measurements of \b-baryon lifetimes, performed at LEP,
originate from two classes of partially reconstructed decays.
In the first class, decays with an exclusively 
reconstructed \Lc baryon
and a lepton of opposite charge are used. These products are
more likely to occur in the decay of \Lb baryons.
In the second class, more inclusive final states with a baryon
(\particle{p}, \particle{\bar{p}}, $\Lambda$, or $\bar{\Lambda}$) 
and a lepton have been used, and these final states can generally
arise from any \b baryon.  With the large \b-hadron samples available
at the Tevatron and the LHC, the most precise measurements of \b-baryons now
come from fully reconstructed exclusive decays.

The following sources of correlated systematic uncertainties have 
been considered:
experimental time resolution within a given experiment, \b-quark
fragmentation distribution into weakly decaying \b baryons,
\Lb polarization, decay model,
and evaluation of the \b-baryon purity in the selected event samples.
In computing the averages
the central values of the masses are scaled to 
$M(\Lb) = 5620 \pm 2\MeVcc$~\cite{Acosta:2005mq} and
$M(\mbox{\b-baryon}) = 5670 \pm 100\MeVcc$.

For the semi-inclusive lifetime measurements, 
the meaning of decay model
systematic uncertainties
and the correlation of these uncertainties between measurements
are not always clear.
Uncertainties related to the decay model are dominated by
assumptions on the fraction of $n$-body semileptonic decays.
To be conservative it is assumed
that these are 100\%  correlated whenever given as an error.
DELPHI varies the fraction of 4-body decays from 0.0 to 0.3. 
In computing the average, the DELPHI
result is corrected to a value of  $0.2 \pm 0.2$ for this fraction.

Furthermore, in computing the average,
the semileptonic decay results from LEP are corrected for a polarization of 
$-0.45^{+0.19}_{-0.17}$~\cite{Abbaneo:2000ej_mod,*Abbaneo:2001bv_mod_cont} and  a 
\Lb fragmentation parameter
$\langle X_E \rangle =0.70\pm 0.03$~\cite{Buskulic:1995mf}.

%The ALEPH result for $\Lambda_b$ polarisation is -0.23 $pm$ 0.25
%(CERN-PPE/95-156) while the others use -0.47 +- 0.47.
%The corresponding results and error are corrected for the ALEPH measurement.

%\par Considering only the measurements obtained with $\Lambda_c \ell$ correlations
%     and $\Lambda \ell^- \ell^+$ the average is :
% $$ \tau_{\Lambda_b} = 1.24^{+0.08}_{-0.08}~ps$$

%     Considering the measurements obtained with $\Lambda$-lepton correlation
%     and with $p \mu$ correlation (b-baryons Admixture)
%     the average is :
% $$ \tau_{\Lambda_b} = 1.15^{+0.08}_{-0.08}~ps$$

\begin{table}[!t]
\caption{Measurements of the \b-baryon lifetimes.
%Measurements of the \b-baryon and \Lb lifetime.
%The DELPHI and ALEPH $\Xi \ell$ results are not included 
%in the quoted average since the selected data samples
%contain mostly \Xib while 
%the data samples in the other measurements contain mostly \Lb.
}
\labt{lifelb}
\begin{center}
\begin{tabular}{lcccl} 
\hline
Experiment&Method                &Data set& Lifetime (ps) & Ref. \\\hline\hline
ALEPH  &$\Lambda\ell$         & 91--95 &$1.20 \pm 0.08 \pm 0.06$ & \cite{Barate:1997if}\\
DELPHI &$\Lambda\ell\pi$ vtx  & 91--94 &$1.16 \pm 0.20 \pm 0.08$        & \cite{Abreu:1999hu}$^b$\\
DELPHI &$\Lambda\mu$ i.p.     & 91--94 &$1.10^{+0.19}_{-0.17} \pm 0.09$ & \cite{Abreu:1996nt}$^b$ \\
DELPHI &\particle{p\ell}      & 91--94 &$1.19 \pm 0.14 \pm 0.07$        & \cite{Abreu:1999hu}$^b$\\
OPAL   &$\Lambda\ell$ i.p.    & 90--94 &$1.21^{+0.15}_{-0.13} \pm 0.10$ & \cite{Akers:1995ui}$^c$  \\
OPAL   &$\Lambda\ell$ vtx     & 90--94 &$1.15 \pm 0.12 \pm 0.06$        & \cite{Akers:1995ui}$^c$ \\ 
%OS% It does no longer make sense to quote the mean b-baryon lifetime, since this has anyway the 
%OS% same value as the Lambda_b lifetime (the above measurements are old and imprecise)
%OS% \multicolumn{3}{l}{Average of above 19: \hfill mean \b-baryon lifetime $=$} & \hfagTAUBBnounit & \\  
\hline
ALEPH  &$\Lc\ell$             & 91--95 &$1.18^{+0.13}_{-0.12} \pm 0.03$ & \cite{Barate:1997if}$^a$\\
ALEPH  &$\Lambda\ell^-\ell^+$ & 91--95 &$1.30^{+0.26}_{-0.21} \pm 0.04$ & \cite{Barate:1997if}$^a$\\
DELPHI &$\Lc\ell$             & 91--94 &$1.11^{+0.19}_{-0.18} \pm 0.05$ & \cite{Abreu:1999hu}$^b$\\
OPAL   &$\Lc\ell$, $\Lambda\ell^-\ell^+$ 
                                 & 90--95 & $1.29^{+0.24}_{-0.22} \pm 0.06$ & \cite{Ackerstaff:1997qi}\\ 
CDF1   &$\Lc\ell$             & 91--95 &$1.32 \pm 0.15        \pm 0.07$ & \cite{Abe:1996df}\\
CDF2   &$\Lc\pi$              & 02--06 &$1.401 \pm 0.046 \pm 0.035$ & \cite{Aaltonen:2009zn} \\
%%CDF2   &$\jpsi \Lambda$      & 02--09 &$1.537 \pm 0.045 \pm 0.014$ & \cite{Aaltonen:2010pj,*Abulencia:2006dr_mod_cont}\\
CDF2   &$\jpsi \Lambda$      & 02--11 &$1.565 \pm 0.035 \pm 0.020$ & \cite{Aaltonen:2014wfa,*Aaltonen:2014wfa_cont} \\
\dzero &$\Lc\mu$              & 02--06 &$1.290^{+0.119+0.087}_{-0.110-0.091}$ & \cite{Abazov_mod:2007tha} \\
\dzero &$\jpsi \Lambda$      & 02--11 &$1.303 \pm 0.075 \pm 0.035$ & \cite{Abazov:2012iy,*Abazov:2007sf_mod_cont,*Abazov:2004bn_mod_cont} \\
ATLAS  &$\jpsi \Lambda$      & 2011   &$1.449 \pm 0.036 \pm 0.017$ & \cite{Aad:2012sh} \\
CMS    &$\jpsi \Lambda$      & 2011   &$1.503 \pm 0.052 \pm 0.031$ & \cite{Chatrchyan:2013sxa} \\ % 5 fb-1
%%% LHCb   &$\jpsi \Lambda$      & 2010   &$1.353 \pm 0.108 \pm 0.035$ & \cite{LHCb-CONF-2011-001}$^p$ \\
LHCb   &$\jpsi \Lambda$      & 2011   &$1.415 \pm 0.027 \pm 0.006$ & \cite{Aaij:2014owa} \\
LHCb   &$\jpsi pK$           & 11--12 &$1.479 \pm 0.009 \pm 0.010$ & \cite{Aaij:2014zyy,*Aaij:2013oha_cont} \\ % 3 fb-1
\multicolumn{3}{l}{Average of above 13: \hfill \Lb lifetime $=$} & \hfagTAULBnounit & \\
\hline\hline
ALEPH  &$\Xi\ell$             & 90--95 &$1.35^{+0.37+0.15}_{-0.28-0.17}$ & \cite{Buskulic:1996sm}\\
DELPHI &$\Xi\ell$             & 91--93 &$1.5 ^{+0.7}_{-0.4} \pm 0.3$     & \cite{Abreu:1995kt}$^d$ \\
DELPHI &$\Xi\ell$             & 92--95 &$1.45 ^{+0.55}_{-0.43} \pm 0.13$     & \cite{Abdallah:2005cw}$^d$ \\
%OS% It does no longer make sense to quote the mean Xib lifetime, since it would merely be the average 
%OS% of the Xib- and Xib0 lifetimes (the above measurements are old and imprecise)
%OS% \multicolumn{3}{l}{Average of above 7: \hfill mean \Xib lifetime $=$} & \hfagTAUXBnounit & \\
\hline
%%CDF2   &$\jpsi \Xi^-$        & 02--09 &$1.56 ^{+0.27}_{-0.25} \pm 0.02$ & \cite{Aaltonen:2009ny} \\
CDF2   &$\jpsi \Xi^-$        & 02--11 &$1.32 \pm 0.14 \pm 0.02$ & \cite{Aaltonen:2014wfa,*Aaltonen:2014wfa_cont} \\ % full Run2 data set = 9.6 fb-1
LHCb   &$\jpsi \Xi^-$         & 11--12 &$1.55 ^{+0.10}_{-0.09} \pm 0.03$ & \cite{Aaij:2014sia} \\ 
LHCb   &$\Xi_c^0\pi^-$        & 11--12 &$1.599 \pm 0.041 \pm 0.022$ & \cite{Aaij:2014lxa} \\ 
\multicolumn{3}{l}{Average of above 3: \hfill \Xibd lifetime $=$} & \hfagTAUXBDnounit & \\
\hline\hline
LHCb   &$\Xi_c^+\pi^-$        & 11--12 &$1.477 \pm 0.026 \pm 0.019$ & \cite{Aaij:2014esa} \\ 
\multicolumn{3}{l}{Average of above 1: \hfill \Xibu lifetime $=$} & \hfagTAUXBUnounit & \\
\hline\hline
%%CDF2   &$\jpsi \Omega^-$     & 02--09 & $1.13 ^{+0.53}_{-0.40} \pm 0.02$ & \cite{Aaltonen:2009ny} \\
CDF2   &$\jpsi \Omega^-$     & 02--11 & $1.66 ^{+0.53}_{-0.40} \pm 0.02$ & \cite{Aaltonen:2014wfa,*Aaltonen:2014wfa_cont} \\ % full Run2 data set = 9.6 fb-1
LHCb   &$\jpsi \Omega^-$     & 11--12 &$1.54 ^{+0.26}_{-0.21} \pm 0.05$ & \cite{Aaij:2014sia} \\ 
\multicolumn{3}{l}{Average of above 2: \hfill \Omegab lifetime $=$} & \hfagTAUOBnounit & \\
\hline\hline
\multicolumn{5}{l}{$^a$ \footnotesize The combined ALEPH result quoted 
in \cite{Barate:1997if} is $1.21 \pm 0.11$ ps.} \\[-0.5ex]
\multicolumn{5}{l}{$^b$ \footnotesize The combined DELPHI result quoted 
in \cite{Abreu:1999hu} is $1.14 \pm 0.08 \pm 0.04$ ps.} \\[-0.5ex]
\multicolumn{5}{l}{$^c$ \footnotesize The combined OPAL result quoted 
in \cite{Akers:1995ui} is $1.16 \pm 0.11 \pm 0.06$ ps.} \\[-0.5ex]
\multicolumn{5}{l}{$^d$ \footnotesize The combined DELPHI result quoted 
in \cite{Abdallah:2005cw} is $1.48 ^{+0.40}_{-0.31} \pm 0.12$ ps.}
%%%\\[-0.5ex] \multicolumn{5}{l}{$^p$ \footnotesize Preliminary.}
\end{tabular}
\end{center}
\end{table}

Inputs to the averages are given in \Table{lifelb}.
For the \Lb lifetime average, we only include measurements obtained
with inclusive \particle{\Lambda^{\pm}_c \ell^{\mp}}, inclusive
$\Lambda \ell^- \ell^+$, and fully exclusive
final states.
The CDF $\Lb \to \jpsi \Lambda$
lifetime result~\cite{Aaltonen:2014wfa,*Aaltonen:2014wfa_cont} 
is $\hfagNSIGMATAULBCDFTWO\,\sigma$
larger than the world average computed excluding this result. 
It is nonetheless combined with the rest 
without adjustment of input errors.
The world average \Lb lifetime is then
\begin{equation}
\tau(\Lb) = \hfagTAULB \,. 
\end{equation}
% Adding also the measurements with more inclusive baryon final states yields the 
% following world average of \b baryons:
% \begin{equation}
% \langle\tau(\mbox{\b-baryon})\rangle = \hfagTAUBB \,.
% \end{equation}
For the strange \b baryons, we no longer include measurements based on
$\Xi^{\mp} \ell^{\mp}$~\cite{Buskulic:1996sm,Abdallah:2005cw,Abreu:1995kt} 
final states which consist of a mixture of 
$\Xib^0$ and $\Xib^-$ baryons. Instead we only average results obtained with 
fully exclusive modes, and obtain
% \begin{equation}
% \langle\tau(\Xib)\rangle = \hfagTAUXB \,.
% \end{equation}
%old% First measurements of fully reconstructed 
%old% $\Xibd \to \jpsi\Xi^-$ and $\Omegab \to \jpsi\Omega^-$
%old% baryons yield~\cite{Aaltonen:2014wfa,*Aaltonen:2014wfa_cont}
\begin{eqnarray}
\tau(\Xibd) &=& \hfagTAUXBD \,, \\
\tau(\Xibu) &=& \hfagTAUXBU \,, \\
\tau(\Omegab) &=& \hfagTAUOB \,. 
\end{eqnarray}

\mysubsubsection{Summary and comparison with theoretical predictions}
\labs{lifesummary}

Averages of lifetimes of specific \b-hadron species are collected
in \Table{sumlife}.
\begin{table}[t]
\caption{Summary of the lifetime averages for the different \b-hadron species.}
\labt{sumlife}
\begin{center}
\begin{tabular}{lrc} \hline
\multicolumn{2}{l}{\b-hadron species} & Measured lifetime \\ \hline
\Bu &                       & \hfagTAUBU   \\
\Bd &                       & \hfagTAUBD   \\
% \Bs ($\to$ flavour specific) & \hfagTAUBSSL \\
% \Bs ($\to \jpsi\phi$)      & \hfagTAUBSJF \\
\Bs & $1/\Gs =$               & \hfagTAUBSMEANC \\
~~ $\B_{s\rm L}$ & $1/\Gamma_{\rm L}=$  & \hfagTAUBSLCON \\
~~ $\B_{s\rm H}$ & $1/\Gamma_{\rm H}=$  & \hfagTAUBSHCON \\
\Bc     &                   & \hfagTAUBC   \\ 
\Lb     &                   & \hfagTAULB   \\
\Xibd   &                   & \hfagTAUXBD  \\
\Xibu   &                   & \hfagTAUXBU  \\
\Omegab &                   & \hfagTAUOB   \\
\hline
%\multicolumn{2}{l}{\b-hadron mixture}  & \hfagTAUB    \\
%OS% \b-baryon mixture           & \hfagTAUBB   \\
%OS% \Xib mixture                & \hfagTAUXB   \\
%\hline
\end{tabular}
\end{center}
%\end{table}
%\begin{table}[t]
\caption{Measured ratios of \b-hadron lifetimes relative to
the \Bd lifetime and ranges predicted
by theory~\cite{Tarantino:2003qw,*Gabbiani:2003pq,Gabbiani:2004tp}.}
\labt{liferatio}
%
% Predictions for tau(Omega_b-)/tau(B0) < 1.10 
% (quoted in D0 observation paper)
% 
% X. Liu et al., Phys. Rev. D 77, 014031 (2008);
% M. Karliner et al., arXiv:0804.1575;
% E. E. Jenkins, Phys. Rev. D 77, 034012 (2008);
% R. Roncaglia, D. B. Lichtenberg, and E. Predazzi, Phys. Rev. D 52, 1722 (1995);
% N. Mathur, R. Lewis, and R. M. Woloshyn, Phys. Rev. D 66, 014502 (2002);
% D. Ebert, R. N. Faustov, and V. O. Galkin, Phys. Rev. D 72, 034026 (2005);
% T. Ito, M. Matsuda, and Y. Matsui, Prog. Theor. Phys. 99, 271 (1998).
%
\begin{center}
\begin{tabular}{lcc} \hline
Lifetime ratio & Measured value & Predicted range \\ \hline
$\tau(\Bu)/\tau(\Bd)$ & \hfagRTAUBU & 1.04 -- 1.08 \\
%%%% $\tau(\Bs)/\tau(\Bd)^a$ & \hfagRTAUBSSL & 0.99 -- 1.01 \\
$\tau(\Bs)/\tau(\Bd)$ & \hfagRTAUBSMEANC & 0.99 -- 1.01 \\
$\tau(\Lb)/\tau(\Bd)$ & \hfagRTAULB & 0.86 -- 0.95    \\
%%% $\tau(\mbox{\b-baryon})/\tau(\Bd)$  & \hfagRTAUBB & 0.86 -- 0.95 \\
\hline
% \multicolumn{3}{l}{$^a$ \footnotesize 
% Using $\bar{\tau}(\Bs) = 1/\Gs = 2/(\Gamma_{\rm L} + \Gamma_{\rm H})$.  }
\end{tabular}
\end{center}
\end{table}
As described in the introduction to \Sec{lifetimes},
the HQE can be employed to explain the hierarchy of
$\tau(\Bc) \ll \tau(\Lb) < \tau(\Bs) \approx \tau(\Bd) < \tau(\Bu)$,
and used to predict the ratios between lifetimes.
Typical predictions are compared to the measured 
lifetime ratios in \Table{liferatio}.
The prediction of the ratio between the \Bu and \Bd lifetimes,
$1.06 \pm 0.02$~\cite{Tarantino:2003qw,*Gabbiani:2003pq}, 
is in good agreement with experiment. 

The total widths of the \Bs and \Bd mesons
are expected to be very close and differ by at most 
1\%~\cite{Beneke:1996gn,*Keum:1998fd,Gabbiani:2004tp}.
This prediction is consistent with the
experimental ratio $\tau(\Bs)/\tau(\Bd)=\Gd/\Gs$,
which is smaller than 1 by 
% $\hfagRTAUBSMEANCsig\,\sigma$ 
\hfagONEMINUSRTAUBSMEANCpercent. 
% at deviation with respect to the prediction. 

The ratio $\tau(\Lb)/\tau(\Bd)$ has particularly
been the source of theoretical
scrutiny since earlier calculations using the HQE
~\cite{Shifman:1986mx,*Chay:1990da,*Bigi:1992su,*Bigi:1992su_erratum,Voloshin:1999pz,*Guberina:1999bw,*Neubert:1996we,*Bigi:1997fj}
predicted a value larger than 0.90, almost $2\,\sigma$ 
above the world average at the time. 
Many predictions cluster around a most likely central value
of 0.94~\cite{Uraltsev:1996ta,*Pirjol:1998ur,*Colangelo:1996ta,*DiPierro:1999tb}.
More recent calculations
of this ratio that include higher-order effects predict a
lower ratio between the
\Lb and \Bd lifetimes~\cite{Tarantino:2003qw,*Gabbiani:2003pq,Gabbiani:2004tp}
and reduce this difference.
References~\cite{Tarantino:2003qw,*Gabbiani:2003pq,Gabbiani:2004tp} present probability density functions
of their predictions with variation of theoretical inputs, and the
indicated ranges in \Table{liferatio}
are the RMS of the distributions from the most probable values, and for 
$\tau(\Lb)/\tau(\Bd)$, also encompass the earlier theoretical predictions%
~\cite{Shifman:1986mx,*Chay:1990da,*Bigi:1992su,*Bigi:1992su_erratum,Voloshin:1999pz,*Guberina:1999bw,*Neubert:1996we,*Bigi:1997fj,Uraltsev:1996ta,*Pirjol:1998ur,*Colangelo:1996ta,*DiPierro:1999tb}.
% Next sentence added on Feb 18, 2011 (following a comment from A. Lenz)
Note that in contrast to the $B$ mesons, complete NLO QCD
corrections and
fully reliable lattice
determinations of the matrix elements for $\Lb$ are not
yet available.
As already mentioned, the CDF measurement of the $\Lambda_b$ lifetime
in the exclusive decay mode $\jpsi \Lambda$~\cite{Aaltonen:2014wfa,*Aaltonen:2014wfa_cont} 
is significantly 
higher than the world average before inclusion, with a ratio
to the $\tau(\Bd)$ world average of 
$\tau(\Lb)/\tau(\Bd) = 1.012 \pm 0.031$, 
% OS, Apr 27, 2012:
%     CDF2 Lambda_b -> J/psi Lambda lifetime result: 1.537 +-0.045 +-0.014 ps
%     World average of B0 lifetime:                  1.5194 +- 0.0071 ps
%     Ratio tau(Lambda_b CDF2)/tau(B0 world) = 1.012 +- 0.031
%          central value = 1.537/1.5194 = 1.01158
%          error = 1.537/1.5194*sqrt((0.045**2+0.014**2)/1.537**2+(0.0071/1.5194)**2) = 0.031375
%
resulting in continued interest in lifetimes of $b$ baryons.

The lifetimes of the most abundant \b-hadron species are now all known to sub-percent precision. Neglecting the 
contributions of the rarer species (\Bc meson and \b baryons other than the \Lb), one can compute the average 
\b-hadron lifetime from the individual lifetimes and production fractions as 
\begin{equation}
\tau_b = \frac%
{\fBd \tau(\Bd)^2+ \fBu \tau(\Bu)^2+0.5 \fBs \tau(B_{s\rm H})^2+0.5 \fBs \tau(B_{s\rm L})^2+ \fbb \tau(\Lambda_b)^2}%
{\fBd \tau(\Bd)  + \fBu \tau(\Bu)  +0.5 \fBs \tau(B_{s\rm H})  +0.5 \fBs \tau(B_{s\rm L})  + \fbb \tau(\Lambda_b)  } \,.
\end{equation}
Using the lifetimes of \Table{sumlife} and the fractions in $Z$ decays of \Table{fractions},
taking into account the correlations between the fractions (\Table{fractions}) as well as the correlation 
between $\tau(B_{s\rm H})$ and $\tau(B_{s\rm L})$ (\hfagZRHOTAUHTAUL), one obtains
\begin{equation}
\tau_b(Z) = \hfagTAUBZCALC \,.
\end{equation}
This is in very good agreement with (and three times more precise than)
the average of \Eq{TAUBVTX} for the inclusive measurements performed at LEP. 
