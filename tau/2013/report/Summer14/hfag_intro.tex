%\documentclass[12pt]{article}
%\begin{document}

\mysection{Introduction}
\label{sec:intro}

Flavor dynamics is an important element in understanding the nature of
particle physics.  The accurate knowledge of properties of heavy flavor
hadrons, especially $b$ hadrons, plays an essential role for
determining the elements of the Cabibbo-Kobayashi-Maskawa (CKM)
weak-mixing matrix~\cite{Cabibbo:1963yz,Kobayashi:1973fv}. 
The operation of the \belle\ and \babar\ $e^+e^-$ $B$ factory 
experiments led to a large increase in the size of available 
$B$ meson, $D$ hadron and $\tau$ lepton samples, 
enabling dramatic improvement in the accuracies of related measurements.
The CDF and \dzero\ experiments at the Fermilab Tevatron 
have also provided important results in heavy flavor physics,
most notably in the $B^0_s$ sector.
%% on $B$ and $D$ meson
%% decays, most notably the discovery of $B^0_s$-$\Bsb^0$ mixing,
%% and confirmation of $D^0$-$\Dzb$ mixing.
The CERN Large Hadron Collider is now delivering high luminosity, 
enabling the collection of even higher statistics samples of $b$ 
and $c$ hadrons at the ATLAS, CMS, and (especially) LHCb experiments.
 
The Heavy Flavor Averaging Group (HFAG) was formed in 2002 to 
continue the activities of the LEP Heavy Flavor Steering 
group~\cite{Abbaneo:2000ej_mod,*Abbaneo:2001bv_mod_cont}. 
This group was responsible for calculating averages of 
measurements of $b$-flavor related quantities. HFAG has evolved 
since its inception and currently consists of seven subgroups:
% 
\begin{itemize}
\item the ``$B$ Lifetime and Oscillations'' subgroup provides 
averages for $b$-hadron lifetimes, $b$-hadron fractions in 
$\Upsilon(4S)$ decay and $p\bar{p}$ collisions, and various 
parameters governing $B^0$-$\Bzb$ and $B_s^0$-$\Bsb^0$ mixing;

\item the ``Unitarity Triangle Parameters'' subgroup provides
averages for time-dependent $\CP$ asymmetry parameters and 
resulting determinations of the angles of the CKM unitarity triangle;

\item the ``Semileptonic $B$ Decays'' subgroup provides averages
for inclusive and exclusive $B$-decay branching fractions, and
subsequent determinations of the CKM matrix elements 
$|V_{cb}|$ and $|V_{ub}|$;

\item the ``$B$ to Charm Decays'' subgroup provides averages of 
branching fractions for $B$ decays to final states involving open 
charm or charmonium mesons;

\item the ``Rare Decays'' subgroup provides averages of branching 
fractions and $\CP$ asymmetries for charmless, radiative, 
leptonic, and baryonic $B$ meson decays;

\item the ``Charm Physics'' subgroup provides averages of branching 
fractions for $D$ meson hadronic and semileptonic decays, 
%properties of excited $D^{**}$ and $D^{}_{sJ}$ mesons, 
averages of $D^0$-$\Dzb$ mixing and $\CP$ and $T$ violation parameters, 
and an average value for the $D^{}_s$ decay constant~$f^{}_{D_s}$.
The subgroup also documents properties of charm baryons, and upper 
limits for rare and forbidden $D^0$, $D^+_{(s)}$, and $\Lambda_c^+$ 
decays.

\item the ``Tau Physics'' subgroup provides documentation and
averages for a selection of \mtau lepton quantities that most profit
from the adoption of the HFAG prescriptions. In particular, the \mtau
lepton branching fractions, uncertainties and correlations are
obtained from a global fit of the experimental results, and this
information is further elaborated to compute several lepton
universality tests and the CKM matrix element $|V_{us}|$. The \mtau
lepton-flavor-violating decays are documented and, starting with this
edition, combinations of such upper limits are also computed.
\end{itemize}

The ``Lifetime and Oscillations'' and ``Semileptonic'' subgroups were formed from the merger of four LEP working groups.
% with some reorganization, i.e., merging four groups into two. 
The ``Unitary Triangle,'' ``$B$ to Charm Decays,'' and ``Rare Decays''
subgroups were formed to provide averages for new results obtained
from the $B$ factory experiments (and now also from the Fermilab 
Tevatron and CERN LHC experiments).
The ``Charm'' and ``Tau''  subgroups were formed more recently in 
response to the wealth of new data concerning $D$ and $\tau$ decays. 
Subgroup typically include representatives from \belle\ and \babar\ and, 
when relevant, CLEO, CDF, \dzero\ and LHCb. 

This article is an update of the last HFAG preprint,
which used results available at least through the end of 2009~\cite{Asner:2010qj}. 
Here we report world averages using results available at least through
the end of 2011. 
%by winter 2011/12.\footnote{
In some cases results available in the early part of 2012 have been
included.\footnote{
  The precise cut-off date for including results in the averages varies 
  between subgroups.}
% and can depend on the importance/impact of the result.
%Several groups have included all results available before the end of April 2012.
%}
In general, we use all publicly available results that have written documentation. 
These include preliminary results presented at conferences or workshops.
However, we do not use preliminary results that remain unpublished 
for an extended period of time, or for which no publication is planned. 
Close contacts have been established between representatives from
the experiments and members of subgroups that perform averaging 
to ensure that the data are prepared in a form suitable for 
combinations.  

In the case of obtaining a world average for which $\chi^2/\dof > 1$,
where $\dof$ is the number of degrees of freedom in the average
calculation, we do not scale the resulting error, as is presently 
done by the Particle Data Group~\cite{PDG_2010}. 
Rather, 
we examine the systematics of each measurement to better understand them. 
Unless we find possible systematic discrepancies between the measurements, 
we do not apply any additional correction to the calculated error. 
We provide the confidence level of the fit as an indicator for the 
consistency of the measurements included in the average. In case some
special treatment was necessary to calculate an average, or if an
approximation used in an average calculation might not be 
%good enough
sufficiently accurate 
(\eg, assuming Gaussian errors when the likelihood function indicates 
non-Gaussian behavior), we include a warning message.

Chapter~\ref{sec:method} describes the methodology used for calculating
averages. In the averaging procedure, common input parameters used in 
the various analyses are adjusted (rescaled) to common values, and, 
where possible, known correlations are taken into account. 
Chapters~\ref{sec:life_mix}--\ref{sec:tau} present world 
average values from each of the subgroups listed above. 
A brief 
summary of the averages presented is given in Chapter~\ref{sec:summary}.   
A complete listing of the averages and plots,
including updates since this document was prepared,
are also available on the HFAG web site:
\vskip0.15in\hskip0.75in
\vbox{
  \href{http://www.slac.stanford.edu/xorg/hfag}{\tt http://www.slac.stanford.edu/xorg/hfag} 
}


